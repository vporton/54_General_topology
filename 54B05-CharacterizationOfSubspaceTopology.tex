\documentclass[12pt]{article}
\usepackage{pmmeta}
\pmcanonicalname{CharacterizationOfSubspaceTopology}
\pmcreated{2013-03-22 15:40:32}
\pmmodified{2013-03-22 15:40:32}
\pmowner{mps}{409}
\pmmodifier{mps}{409}
\pmtitle{characterization of subspace topology}
\pmrecord{6}{37616}
\pmprivacy{1}
\pmauthor{mps}{409}
\pmtype{Theorem}
\pmcomment{trigger rebuild}
\pmclassification{msc}{54B05}
%\pmkeywords{subspace}
%\pmkeywords{weak topology}
%\pmkeywords{universal property}

% this is the default PlanetMath preamble.  as your knowledge
% of TeX increases, you will probably want to edit this, but
% it should be fine as is for beginners.

% almost certainly you want these
\usepackage{amssymb}
\usepackage{amsmath}
\usepackage{amsfonts}

% used for TeXing text within eps files
%\usepackage{psfrag}
% need this for including graphics (\includegraphics)
%\usepackage{graphicx}
% for neatly defining theorems and propositions
\usepackage{amsthm}
% making logically defined graphics
%%%\usepackage{xypic}

% there are many more packages, add them here as you need them

% define commands here
\newtheorem*{theorem*}{Theorem}
\begin{document}
\PMlinkescapeword{completes}
\begin{theorem*}
Let $X$ be a topological space and $Y\subset X$ any subset.  The subspace topology on $Y$ is the weakest topology making the inclusion map continuous.
\end{theorem*}

\begin{proof}
Let $\mathcal{S}$ denote the subspace topology on $Y$ and $j\colon Y\hookrightarrow X$ denote the inclusion map.

Suppose $\{\mathcal{T}_{\alpha}\mid\alpha\in J\}$ is a family of topologies on $Y$ such that each inclusion map $j_{\alpha}\colon(Y,\mathcal{T}_{\alpha})\hookrightarrow X$ is continuous.  Let $\mathcal{T}$ be the intersection $\bigcap_{\alpha\in J}\mathcal{T}_{\alpha}$.  Observe that $\mathcal{T}$ is also a topology on $Y$.  Let $U$ be open in $X$.  By continuity of $j_{\alpha}$, the set $j_{\alpha}^{-1}(U)=j^{-1}(U)$ is open in each $\mathcal{T}_{\alpha}$; consequently, $j^{-1}(U)$ is also in $\mathcal{T}$.   This shows that there is a weakest topology on $Y$ making inclusion continuous.

We claim that any topology strictly weaker than $\mathcal{S}$ fails to make the inclusion map continuous.  To see this, suppose $\mathcal{S}_0\subsetneq\mathcal{S}$ is a topology on $Y$.  Let $V$ be a set open in $\mathcal{S}$ but not in $\mathcal{S}_0$.  By the definition of subspace topology, $V=U\cap Y$ for some open set $U$ in $X$.  But $j^{-1}(U)=V$, which was specifically chosen not to be in $\mathcal{S_0}$.  Hence $\mathcal{S_0}$ does not make the inclusion map continuous.  This completes the proof.
\end{proof}
%%%%%
%%%%%
\end{document}
