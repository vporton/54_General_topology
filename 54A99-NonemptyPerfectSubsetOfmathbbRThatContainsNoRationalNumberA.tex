\documentclass[12pt]{article}
\usepackage{pmmeta}
\pmcanonicalname{NonemptyPerfectSubsetOfmathbbRThatContainsNoRationalNumberA}
\pmcreated{2013-03-22 15:26:10}
\pmmodified{2013-03-22 15:26:10}
\pmowner{Gorkem}{3644}
\pmmodifier{Gorkem}{3644}
\pmtitle{nonempty perfect subset of $\mathbb{R}$ that contains no rational number, a}
\pmrecord{18}{37283}
\pmprivacy{1}
\pmauthor{Gorkem}{3644}
\pmtype{Example}
\pmcomment{trigger rebuild}
\pmclassification{msc}{54A99}

% this is the default PlanetMath preamble.  as your knowledge
% of TeX increases, you will probably want to edit this, but
% it should be fine as is for beginners.

% almost certainly you want these
\usepackage{amssymb}
\usepackage{amsmath}
\usepackage{amsfonts}

% used for TeXing text within eps files
%\usepackage{psfrag}
% need this for including graphics (\includegraphics)
%\usepackage{graphicx}
% for neatly defining theorems and propositions
%\usepackage{amsthm}
% making logically defined graphics
%%%\usepackage{xypic}

% there are many more packages, add them here as you need them

% define commands here

\newcommand{\disp}{\displaystyle}
\begin{document}
We will construct a nonempty perfect set contained in $\mathbb{R}$ that
contains no rational number.

We will begin with a closed interval, and then, imitating the
construction of Cantor set, we will inductively delete each
rational number in it together with an open interval.  We will do
it in such a way that the end points of the open intervals will
never be deleted afterwards.

Let $E_0 = [b_0, a_0]$ for some irrational numbers $a_0$ and $b_0$,
 with $b_0<a_0$
. Let $\{ q_1, q_2, q_3, \ldots \}$ be an enumeration of the
rational numbers in $[b_0, a_0]$. For each $q_i$, we will define an
open interval $(a_i,b_i)$ and delete it.

Let $a_1$ and $b_1$ be two irrational numbers such that
$b_0<a_1<q_1<b_1<a_0$. Define $E_1 = E_0\backslash (a_1,b_1)$.

Having defined $ E_1,E_2,\ldots,E_n$, $ a_1,a_2,\ldots,a_n $ and $
b_1,b_2,\ldots,b_n $, let's define $a_{n+1}$ and $b_{n+1}$:

If  $\disp q_{n+1}\in \bigcup_{k=1}^n (a_k,b_k)$ then there exists
an $i\leq n$ such that $q_{n+1}\in  (a_i,b_i)$.  Let $a_{n+1}=a_i$
and $b_{n+1}=b_i$.

Otherwise let $a_{n+1}$ and $b_{n+1}$ be two irrational numbers
such that $b_0<a_{n+1}<q_{n+1}<b_{n+1}<a_0$, and which satisfy:
$$
\disp q_{n+1} - a_{n+1} < \min_{i=0,1,2,\ldots,n}\{|q_{n+1} -
b_i|\}$$ and $$\disp b_{n+1} - q_{n+1} <
\min_{i=0,1,2,\ldots,n}\{|a_i - q_{n+1}|\}.
$$



Now define $E_{n+1}=E_n\backslash (a_{n+1},b_{n+1})$.  Note that
by our choice of $a_{n+1}$ and $b_{n+1}$ any of the previous end
points are not removed from $E_n$.

Let $\disp E = \bigcap_{n=1}^\infty E_n$.  $E$ is clearly
nonempty, does not contain any rational number, and also it is
compact, being an intersection of compact sets.

Now let us see that $E$ does not have any isolated points. Let
$x\in E$, and $\epsilon>0$ be given. If $x\neq a_j$ for any $j \in \{0,1,2,\ldots\}$, choose a rational number
$q_k$ such that $x<q_k<x+\epsilon$.  Then $q_k\in(a_k,b_k)$ and
since $x\in E$ we must have $x<a_k$, which means $a_k \in (x,
x+\epsilon)$. Since we know that $a_k\in E $, this shows that $x$ is a limit point.  Otherwise, if $x = a_j$ for some $j$, then choose a $q_k$ such that $x-\epsilon < q_k < x$. Similarly, $q_k \in (a_k, b_k)$ and it follows that $b_k\in (x-\epsilon, x)$.    We have shown that
any point of $E$ is a limit point, hence $E$ is perfect.
%%%%%
%%%%%
\end{document}
