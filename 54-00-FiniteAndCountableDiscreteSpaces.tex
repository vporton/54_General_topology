\documentclass[12pt]{article}
\usepackage{pmmeta}
\pmcanonicalname{FiniteAndCountableDiscreteSpaces}
\pmcreated{2013-03-22 15:17:11}
\pmmodified{2013-03-22 15:17:11}
\pmowner{matte}{1858}
\pmmodifier{matte}{1858}
\pmtitle{finite and countable discrete spaces}
\pmrecord{9}{37077}
\pmprivacy{1}
\pmauthor{matte}{1858}
\pmtype{Theorem}
\pmcomment{trigger rebuild}
\pmclassification{msc}{54-00}

\endmetadata

% this is the default PlanetMath preamble.  as your knowledge
% of TeX increases, you will probably want to edit this, but
% it should be fine as is for beginners.

% almost certainly you want these
\usepackage{amssymb}
\usepackage{amsmath}
\usepackage{amsfonts}
\usepackage{amsthm}

\usepackage{mathrsfs}

% used for TeXing text within eps files
%\usepackage{psfrag}
% need this for including graphics (\includegraphics)
%\usepackage{graphicx}
% for neatly defining theorems and propositions
%
% making logically defined graphics
%%%\usepackage{xypic}

% there are many more packages, add them here as you need them

% define commands here

\newcommand{\sR}[0]{\mathbb{R}}
\newcommand{\sC}[0]{\mathbb{C}}
\newcommand{\sN}[0]{\mathbb{N}}
\newcommand{\sZ}[0]{\mathbb{Z}}

 \usepackage{bbm}
 \newcommand{\Z}{\mathbbmss{Z}}
 \newcommand{\C}{\mathbbmss{C}}
 \newcommand{\F}{\mathbbmss{F}}
 \newcommand{\R}{\mathbbmss{R}}
 \newcommand{\Q}{\mathbbmss{Q}}



\newcommand*{\norm}[1]{\lVert #1 \rVert}
\newcommand*{\abs}[1]{| #1 |}



\newtheorem{thm}{Theorem}
\newtheorem{defn}{Definition}
\newtheorem{prop}{Proposition}
\newtheorem{lemma}{Lemma}
\newtheorem{cor}{Corollary}
\begin{document}
\begin{thm}
Suppose $X\neq \emptyset$ is equipped with the discrete topology.
\begin{enumerate}
\item If $X$ if finite, then $X$ is homeomorphic to $\{1,\ldots, n\}$
  for some $n\ge 1$.
\item If $X$ if countable, then $X$ is homeomorphic to $\Z$.
\end{enumerate}
Here, $\{1,\ldots, n\}$ and $\Z$ are endowed with the discrete topology 
(or, equivalently, the subspace topology from $\R$).
\end{thm}

\begin{proof} The first claim will be proven. If 
  $$
     X=\{a_1,\ldots, a_n\}
  $$
let $\Phi\colon \{1,\ldots, n\} \to X$ be
  $$
    \Phi(i)=a_i,\quad i=1,\ldots, n.
  $$
Since $\Phi$ is a bijection, it is a homeomorphism.

The proof of the second claim is \PMlinkescapetext{similar} to that of the first. 
\end{proof}
%%%%%
%%%%%
\end{document}
