\documentclass[12pt]{article}
\usepackage{pmmeta}
\pmcanonicalname{ProductOfUniformSpaces}
\pmcreated{2013-03-22 16:30:49}
\pmmodified{2013-03-22 16:30:49}
\pmowner{mps}{409}
\pmmodifier{mps}{409}
\pmtitle{product of uniform spaces}
\pmrecord{4}{38691}
\pmprivacy{1}
\pmauthor{mps}{409}
\pmtype{Definition}
\pmcomment{trigger rebuild}
\pmclassification{msc}{54E15}

% this is the default PlanetMath preamble.  as your knowledge
% of TeX increases, you will probably want to edit this, but
% it should be fine as is for beginners.

% almost certainly you want these
\usepackage{amssymb}
\usepackage{amsmath}
\usepackage{amsfonts}

% used for TeXing text within eps files
%\usepackage{psfrag}
% need this for including graphics (\includegraphics)
%\usepackage{graphicx}
% for neatly defining theorems and propositions
\usepackage{amsthm}
% making logically defined graphics
%%%\usepackage{xypic}

% there are many more packages, add them here as you need them

% define commands here
\theoremstyle{definition}
\newtheorem*{definition*}{Definition}
\begin{document}
\begin{definition*}
Let $\{X_{\alpha}\}_{\alpha\in I}$ be a nonempty family of uniform spaces.  The \emph{\PMlinkescapetext{product of the uniform spaces}} is the \PMlinkname{weakest uniformity}{UniformitiesOnASetFormACompleteLattice} on the Cartesian product $X=\prod_{\alpha\in I} X_{\alpha}$ making all the projection maps $\pi_{\alpha}\colon X\to X_{\alpha}$ uniformly continuous.
\end{definition*}

%% still to come - actual constructions
%%%%%
%%%%%
\end{document}
