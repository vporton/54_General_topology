\documentclass[12pt]{article}
\usepackage{pmmeta}
\pmcanonicalname{ACompactMetricSpaceIsSecondCountable}
\pmcreated{2013-03-22 17:00:49}
\pmmodified{2013-03-22 17:00:49}
\pmowner{azdbacks4234}{14155}
\pmmodifier{azdbacks4234}{14155}
\pmtitle{a compact metric space is second countable}
\pmrecord{17}{39296}
\pmprivacy{1}
\pmauthor{azdbacks4234}{14155}
\pmtype{Theorem}
\pmcomment{trigger rebuild}
\pmclassification{msc}{54D70}
%\pmkeywords{compact}
%\pmkeywords{metric space}
%\pmkeywords{metrizable}
%\pmkeywords{open ball}
%\pmkeywords{basis}
%\pmkeywords{open cover}
%\pmkeywords{countable}
\pmrelated{MetricSpace}
\pmrelated{Compact}
\pmrelated{Lindelof}
\pmrelated{Ball}
\pmrelated{basisTopologicalSpace}
\pmrelated{Cover}
\pmrelated{BasisTopologicalSpace}

\endmetadata

%%packages
\usepackage{amssymb}
\usepackage{amsmath}
\usepackage{amsfonts}
\usepackage{amsthm}
%%theorem environments
\theoremstyle{plain}
\newtheorem*{theorem*}{Theorem}
\newtheorem*{lemma*}{Lemma}
\newtheorem*{corollary*}{Corollary}
\newtheorem*{proposition*}{Proposition}
%math operators and commands
\newcommand{\set}[1]{\{#1\}}
\newcommand{\medset}[1]{\left\{#1\right\}}
\newcommand{\bigset}[1]{\bigg\{#1\bigg\}}
\newcommand{\abs}[1]{\vert#1\vert}
\newcommand{\medabs}[1]{\left\vert#1\right\vert}
\newcommand{\bigabs}[1]{\bigg\vert#1\bigg\vert}
\newcommand{\norm}[1]{\Vert#1\Vert}
\newcommand{\mednorm}[1]{\left\Vert#1\right\Vert}
\newcommand{\bignorm}[1]{\bigg\Vert#1\bigg\Vert}
\DeclareMathOperator{\Aut}{Aut}
\DeclareMathOperator{\End}{End}
\DeclareMathOperator{\Inn}{Inn}
\DeclareMathOperator{\supp}{supp}

\begin{document}
\begin{proposition*}
Every compact metric space is second countable.
\end{proposition*}
\begin{proof}
Let $(X,d)$ be a compact metric space, and for each $n\in\mathbb{Z}^+$ define $\mathcal{A}_n=\{B(x,1/n):x\in X\}$, where $B(x,1/n)$ denotes the open ball centered about $x$ of \PMlinkid{radius}{1296} $1/n$. Each such collection is an open cover of the compact space $X$, so for each $n\in\mathbb{Z}^+$ there exists a finite collection $\mathcal{B}_n\subseteq\mathcal{A}_n$ that \PMlinkescapetext{covers} $X$. Put $\mathcal{B}=\bigcup_{n=1}^\infty\mathcal{B}_n$. Being a countable union of finite sets, it follows that $\mathcal{B}$ is countable; we assert that it forms a basis for the metric topology on $X$. The first property of a basis is satisfied trivially, as each set $\mathcal{B}_n$ is an open cover of $X$. For the second property, let $x,x_1,x_2\in X$, $n_1,n_2\in\mathbb{Z}^+$, and suppose $x\in B(x_1,1/n_1)\cap B(x_2,1/n_2)$. Because the sets $B(x_1,1/n_1)$ and $B(x_2,1/n_2)$ are open in the metric topology on $X$, their intersection is also open, so there exists $\epsilon>0$ such that $B(x,\epsilon)\subseteq B(x_1,1/n_1)\cap B(x_2,1/n_2)$. Select $N\in\mathbb{Z}^+$ such that $1/N<\epsilon$. There must exist $x_3\in X$ such that $x\in B(x_3,1/2N)$ (since $\mathcal{B}_{2N}$ is an open cover of $X$). To see that $B(x_3,1/2N)\subseteq B(x_1,1/n_1)\cap B(x_2,1/n_2)$, let $y\in B(x_3,1/2N)$. Then we have
\begin{equation}
d(x,y)\leq d(x,x_3)+d(x_3,y)<\dfrac{1}{2N}+\dfrac{1}{2N}=\dfrac{1}{N}<\epsilon\text{,}
\end{equation} 
so that $y\in B(x,\epsilon)$, from which it follows that $y\in B(x_1,1/n_1)\cap B(x_2,1/n_2)$, hence that $B(x_3,1/2N)\subseteq B(x_1,1/n_1)\cap B(x_2,1/n_2)$. Thus the countable collection $\mathcal{B}$ forms a basis for a topology on $X$; the verification that the topology \PMlinkescapetext{induced} by $\mathcal{B}$ is in fact the metric topology follows by an \PMlinkescapetext{argument similar} to that used to verify the second property of a basis, and completes the proof that $X$ is second countable.
\end{proof}\
It is worth nothing that, because a countable union of countable sets is countable, it would have been sufficient to assume that $(X,d)$ was a Lindel\"{o}f space.
%%%%%
%%%%%
\end{document}
