\documentclass[12pt]{article}
\usepackage{pmmeta}
\pmcanonicalname{Discrete}
\pmcreated{2013-03-22 17:56:49}
\pmmodified{2013-03-22 17:56:49}
\pmowner{lalberti}{18937}
\pmmodifier{lalberti}{18937}
\pmtitle{discrete}
\pmrecord{8}{40445}
\pmprivacy{1}
\pmauthor{lalberti}{18937}
\pmtype{Definition}
\pmcomment{trigger rebuild}
\pmclassification{msc}{54A05}
\pmrelated{Discrete}

% this is the default PlanetMath preamble.  as your knowledge
% of TeX increases, you will probably want to edit this, but
% it should be fine as is for beginners.

% almost certainly you want these
\usepackage{amssymb}
\usepackage{amsmath}
\usepackage{amsfonts}

% used for TeXing text within eps files
%\usepackage{psfrag}
% need this for including graphics (\includegraphics)
%\usepackage{graphicx}
% for neatly defining theorems and propositions
%\usepackage{amsthm}
% making logically defined graphics
%%%\usepackage{xypic}

% there are many more packages, add them here as you need them

% define commands here
\newcommand{\R}{\mathbb{R}}

\begin{document}
This entry aims at highlighting the fact that all uses of the word \emph{discrete} in mathematics are directly related to the core concept of discrete space:
\begin{itemize}
\item A discrete set is a set that, endowed with the topology implied by the context, is a \PMlinkescaptetext{discrete space}. For instance for a subset of $\R^n$ and without information suggesting otherwise, the topology on the set would be assumed the usual topology induced by norms on $\R^n$.
\item A random variable $X$ is discrete if and only if its image space is a discrete set (which by what's just been said means that the image is a discrete topological space for some topology specified by the context). The most common example by far is a random variable taking its values in a enumerated set (e.g. the values of a die, or a set of possible answers to a question in a survey).
\item Discretization of ODEs and PDEs is the process of converting equations on functions on open sets of $\R^n$ (with boundary conditions) into equations on functions on discrete subsets of $\R^n$.
\end{itemize}

%%%%%
%%%%%
\end{document}
