\documentclass[12pt]{article}
\usepackage{pmmeta}
\pmcanonicalname{ExampleOfPseudometricSpace}
\pmcreated{2013-03-22 14:40:24}
\pmmodified{2013-03-22 14:40:24}
\pmowner{mathcam}{2727}
\pmmodifier{mathcam}{2727}
\pmtitle{example of pseudometric space}
\pmrecord{6}{36275}
\pmprivacy{1}
\pmauthor{mathcam}{2727}
\pmtype{Example}
\pmcomment{trigger rebuild}
\pmclassification{msc}{54E35}
\pmrelated{Seminorm}
\pmrelated{VectorSpace}
\pmrelated{MetricSpace}
\pmrelated{Metric}
\pmdefines{trivial pseudometric}

\endmetadata

% this is the default PlanetMath preamble.  as your knowledge
% of TeX increases, you will probably want to edit this, but
% it should be fine as is for beginners.

% almost certainly you want these
\usepackage{amssymb}
\usepackage{amsmath}
\usepackage{amsfonts}
\usepackage{amsthm}

% used for TeXing text within eps files
%\usepackage{psfrag}
% need this for including graphics (\includegraphics)
%\usepackage{graphicx}
% for neatly defining theorems and propositions
%\usepackage{amsthm}
% making logically defined graphics
%%%\usepackage{xypic}

% there are many more packages, add them here as you need them

% define commands here

\newcommand{\mc}{\mathcal}
\newcommand{\mb}{\mathbb}
\newcommand{\mf}{\mathfrak}
\newcommand{\ol}{\overline}
\newcommand{\ra}{\rightarrow}
\newcommand{\la}{\leftarrow}
\newcommand{\La}{\Leftarrow}
\newcommand{\Ra}{\Rightarrow}
\newcommand{\nor}{\vartriangleleft}
\newcommand{\Gal}{\text{Gal}}
\newcommand{\GL}{\text{GL}}
\newcommand{\Z}{\mb{Z}}
\newcommand{\R}{\mb{R}}
\newcommand{\Q}{\mb{Q}}
\newcommand{\C}{\mb{C}}
\newcommand{\<}{\langle}
\renewcommand{\>}{\rangle}
\begin{document}
Let $X=\mathbb{R}^2$ and consider the function $d:X\times X$ to the non-negative real numbers given by 
\begin{align*}
d((x_1,x_2),(y_1,y_2))=|x_1-y_1|.
\end{align*}

Then $d(x,x)=|x_1-x_1|=0$, $d(x,y)=|x_1-y_1|=|y_1-x_1|=d(y,z)$ and the triangle inequality follows from the triangle inequality on $\mathbb{R}^1$, so $(X,d)$ satisfies the defining conditions of a pseudometric space.

Note, however, that this is not an example of a metric space, since we can have two distinct points that are distance 0 from each other, e.g.
\begin{align*}
d((2,3),(2,5))=|2-2|=0.
\end{align*}

Other examples:
\begin{itemize}
\item Let $X$ be a set, $x_0\in X$, and let $F(X)$ be functions $X\to R$.
Then $d(f,g)=|f(x_0)-g(x_0)|$
is a pseudometric on $F(X)$ \cite{willard}. 

\item If $X$ is a vector space and $p$ is a seminorm over $X$, then 
$d(x,y)=p(x-y)$ is a pseudometric on $X$.

\item The \emph{trivial pseudometric} $d(x,y)=0$ for all $x,y\in X$ is a
pseudometric. 
\end{itemize}

\begin{thebibliography}{9}
\bibitem{willard} S. Willard, \emph{General Topology},
Addison-Wesley, Publishing Company, 1970.
\end{thebibliography}
%%%%%
%%%%%
\end{document}
