\documentclass[12pt]{article}
\usepackage{pmmeta}
\pmcanonicalname{Cutpoint}
\pmcreated{2013-03-22 13:56:38}
\pmmodified{2013-03-22 13:56:38}
\pmowner{mathcam}{2727}
\pmmodifier{mathcam}{2727}
\pmtitle{cut-point}
\pmrecord{5}{34706}
\pmprivacy{1}
\pmauthor{mathcam}{2727}
\pmtype{Definition}
\pmcomment{trigger rebuild}
\pmclassification{msc}{54D05}
\pmsynonym{cutpoint}{Cutpoint}

% this is the default PlanetMath preamble.  as your knowledge
% of TeX increases, you will probably want to edit this, but
% it should be fine as is for beginners.

% almost certainly you want these
\usepackage{amssymb}
\usepackage{amsmath}
\usepackage{amsfonts}

% used for TeXing text within eps files
%\usepackage{psfrag}
% need this for including graphics (\includegraphics)
%\usepackage{graphicx}
% for neatly defining theorems and propositions
%\usepackage{amsthm}
% making logically defined graphics
%%%\usepackage{xypic}

% there are many more packages, add them here as you need them

% define commands here

\newcommand{\sR}[0]{\mathbb{R}}
\newcommand{\sC}[0]{\mathbb{C}}
\newcommand{\sN}[0]{\mathbb{N}}
\newcommand{\sZ}[0]{\mathbb{Z}}
\begin{document}
{\bf Theorem}
Suppose $X$ is a connected space and $x$ is a point in $X$.
If $X\setminus \{x\}$ is a disconnected set in $X$, then $x$ is a
{\bf cut-point} of $X$ \cite{jameson, ward}.

\subsubsection{Examples}
\begin{enumerate}
\item Any point of $\sR$ with the usual topology is a cut-point.
\item If $X$ is a normed vector space with $\dim X>1$, then $X$ has
no cut-points \cite{jameson}.
\end{enumerate}

\begin{thebibliography}{9}
 \bibitem{jameson} G.J. Jameson, \emph{Topology and Normed Spaces},
 Chapman and Hall, 1974.
\bibitem{ward} L.E. Ward, \emph{Topology, An Outline for a First Course},
Marcel Dekker, Inc., 1972.
 \end{thebibliography}
%%%%%
%%%%%
\end{document}
