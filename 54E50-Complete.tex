\documentclass[12pt]{article}
\usepackage{pmmeta}
\pmcanonicalname{Complete}
\pmcreated{2013-03-22 11:55:11}
\pmmodified{2013-03-22 11:55:11}
\pmowner{djao}{24}
\pmmodifier{djao}{24}
\pmtitle{complete}
\pmrecord{12}{30603}
\pmprivacy{1}
\pmauthor{djao}{24}
\pmtype{Definition}
\pmcomment{trigger rebuild}
\pmclassification{msc}{54E50}

\endmetadata

\usepackage{amssymb}
\usepackage{amsmath}
\usepackage{amsfonts}
\usepackage{graphicx}
%%%%\usepackage{xypic}
\begin{document}
A metric space $X$ is {\em complete} if every \PMlinkname{Cauchy sequence}{CauchySequence} in $X$ is a convergent sequence.

{\bf Examples:}

\PMlinkescapephrase{Cauchy sequence}

\begin{itemize}
\item The space $\mathbb{Q}$ of rational numbers is not complete: the sequence $3$, $3.1$, $3.14$, $3.141$, $3.1415$, $3.14159$, $3.141592\ldots$ consisting of finite decimals converging to $\pi \in \mathbb{R}$ is a Cauchy sequence in $\mathbb{Q}$ that does not converge in $\mathbb{Q}$.
\item The space $\mathbb{R}$ of real numbers is complete, as it is the completion of $\mathbb{Q}$ with respect to the standard metric (other completions, such as the $p$-adic numbers, are also possible). More generally, the completion of any metric space is a complete metric space.
\item Every Banach space is complete. For example, the $L^p$--space of p-integrable functions is a complete metric space if $p \geq 1$.
\end{itemize}
%%%%%
%%%%%
%%%%%
%%%%%
\end{document}
