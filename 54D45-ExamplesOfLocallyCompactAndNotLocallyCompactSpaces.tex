\documentclass[12pt]{article}
\usepackage{pmmeta}
\pmcanonicalname{ExamplesOfLocallyCompactAndNotLocallyCompactSpaces}
\pmcreated{2013-03-22 12:48:36}
\pmmodified{2013-03-22 12:48:36}
\pmowner{AxelBoldt}{56}
\pmmodifier{AxelBoldt}{56}
\pmtitle{examples of locally compact and not locally compact spaces}
\pmrecord{18}{33129}
\pmprivacy{1}
\pmauthor{AxelBoldt}{56}
\pmtype{Example}
\pmcomment{trigger rebuild}
\pmclassification{msc}{54D45}
\pmrelated{TopologicalSpace}

% this is the default PlanetMath preamble.  as your knowledge
% of TeX increases, you will probably want to edit this, but
% it should be fine as is for beginners.

% almost certainly you want these
\usepackage{amssymb}
\usepackage{amsmath}
\usepackage{amsfonts}

% used for TeXing text within eps files
%\usepackage{psfrag}
% need this for including graphics (\includegraphics)
%\usepackage{graphicx}
% for neatly defining theorems and propositions
%\usepackage{amsthm}
% making logically defined graphics
%%%\usepackage{xypic} 

% there are many more packages, add them here as you need them

% define commands here
\begin{document}
\PMlinkescapeword{rationals}
\PMlinkescapeword{unit}

Examples of locally compact spaces include:

\begin{itemize}

\item The Euclidean spaces $\Bbb{R}^n$ with the standard topology: their local compactness follows from the Heine-Borel theorem. The \PMlinkname{complex plane}{Complex} $\Bbb{C}$ carries the same topology as $\Bbb{R}^2$ and is therefore also locally compact. 

\item All topological manifolds are locally compact since locally they look like Euclidean space.

\item Any closed or open subset of a locally compact space is locally compact. In fact, a subset of a locally compact Hausdorff space $X$ is locally compact if and only if it is the \PMlinkname{difference}{SetDifference} of two closed subsets of $X$ (equivalently: the intersection of an open and a closed subset of $X$).

\item The space of \PMlinkname{$p$-adic rationals}{PAdicIntegers} is homeomorphic to the Cantor set minus one point, and since the Cantor set is compact as a closed bounded subset of $\Bbb{R}$, we see that the $p$-adic rationals are locally compact.

\item Any discrete space is locally compact, since the singletons can serve as compact neighborhoods.

\item The long line is a locally compact topological space.

\item If you take any unbounded totally ordered set and equip it with the left order topology (or right order topology), you get a locally compact space. This space, unlike all the others we have looked at, is not Hausdorff.

\end{itemize}

Examples of spaces which are \emph{not} locally compact include:

\begin{itemize}

\item The rational numbers $\Bbb{Q}$ with the standard topology inherited from $\Bbb{R}$: each of its compact subsets has empty interior.

\item All infinite-dimensional normed vector spaces: a normed vector space is finite-dimensional if and only if its closed unit ball is compact.

\item The subset $X=\{(0,0)\}\cup\{(x,y)\mid x>0\}$ of $\Bbb{R}^2$: no compact 
subset of $X$ contains a neighborhood of $(0,0)$.

\end{itemize}
%%%%%
%%%%%
\end{document}
