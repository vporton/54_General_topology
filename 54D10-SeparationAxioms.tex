\documentclass[12pt]{article}
\usepackage{pmmeta}
\pmcanonicalname{SeparationAxioms}
\pmcreated{2013-03-22 13:28:47}
\pmmodified{2013-03-22 13:28:47}
\pmowner{Koro}{127}
\pmmodifier{Koro}{127}
\pmtitle{separation axioms}
\pmrecord{26}{34050}
\pmprivacy{1}
\pmauthor{Koro}{127}
\pmtype{Definition}
\pmcomment{trigger rebuild}
\pmclassification{msc}{54D10}
\pmclassification{msc}{54D15}
\pmsynonym{separation properties}{SeparationAxioms}
\pmrelated{NormalTopologicalSpace}
\pmrelated{HausdorffSpaceNotCompletelyHausdorff}
\pmrelated{SierpinskiSpace}
\pmrelated{MetricSpacesAreHausdorff}
\pmrelated{ZeroDimensional}
\pmrelated{T2Space}
\pmrelated{RegularSpace}
\pmrelated{T4Space}
\pmdefines{Hausdorff}
\pmdefines{completely Hausdorff}
\pmdefines{normal}
\pmdefines{completely normal}
\pmdefines{regular}
\pmdefines{Tychonoff}
\pmdefines{completely regular}
\pmdefines{perfectly normal}
\pmdefines{Tychonov}
\pmdefines{perfectly $T_4$}

% this is the default PlanetMath preamble.  as your knowledge
% of TeX increases, you will probably want to edit this, but
% it should be fine as is for beginners.

% almost certainly you want these
\usepackage{amssymb}
\usepackage{amsmath}
\usepackage{amsfonts}
\usepackage{mathrsfs}

% used for TeXing text within eps files
%\usepackage{psfrag}
% need this for including graphics (\includegraphics)
%\usepackage{graphicx}
% for neatly defining theorems and propositions
%\usepackage{amsthm}
% making logically defined graphics
%%%\usepackage{xypic}

% there are many more packages, add them here as you need them

% define commands here
\renewcommand{\text}{\textnormal}
\newcommand{\C}{\mathbb{C}}
\newcommand{\R}{\mathbb{R}}
\newcommand{\N}{\mathbb{N}}
\newcommand{\Z}{\mathbb{Z}}
\begin{document}
\PMlinkescapeword{name} \PMlinkescapeword{names} 
\PMlinkescapeword{type} \PMlinkescapeword{types} 
\PMlinkescapeword{order} \PMlinkescapeword{axiom}

The \emph{separation axioms} are additional conditions which may be required to a topological space in order to ensure that some particular types of sets can be
separated by open sets, thus avoiding certain pathological cases.

\begin{center}
\begin{tabular*}{.8\textwidth}{lp{0.65\textwidth}}
\textbf{Axiom}                          & \textbf{Definition} \\
\hline
$T_0$ & given two distinct points, there is an open set containing exactly one of them; \\
\PMlinkname{$T_1$}{T1Space} & given two distinct points, there is a neighborhood of each of them which does not contain the other point;\\
\PMlinkname{$T_2$}{T2Space} & given two distinct points, there are two disjoint open sets each of which contains one of the points;\\
$T_{2\frac{1}{2}}$ & given two distinct points, there are two open sets, each of which contains one of the points, whose closures are disjoint;\\
\PMlinkname{$T_3$}{T3Space} & given a closed set $A$ and a point $x\notin A$, there are two disjoint open sets $U$ and $V$ such that $x\in U$ and $A\subset V$;\\
$T_{3\frac{1}{2}}$ & given a closed set $A$ and a point $x\notin A$, there is an Urysohn function for $A$ and $\{b\}$;\\
$T_4$ & given two disjoint closed sets $A$ and $B$, there are two disjoint open sets $U$ and $V$ such that $A\subset U$ and $B\subset V$;\\
$T_5$ & given two separated sets $A$ and $B$, there are two disjoint open sets $U$ and $V$ such that $A\subset U$ and $B\subset V$.\\
\end{tabular*}
\end{center}

If a topological space satisfies a $T_i$ axiom, it is called a $T_i$-space.
The following table shows other common names for topological spaces with these or other additional separation properties.
\begin{center}
\begin{tabular}{ll}
\textbf{Name}                         & \textbf{Separation properties} \\
\hline
Kolmogorov space                      & $T_0$ \\
Fr\'echet space                         & $T_1$ \\
Hausdorff space                       & $T_2$ \\
Completely Hausdorff space            & $T_{2\frac{1}{2}}$\\
Regular space                         & $T_3$ and $T_0$\\
Tychonoff or completely regular space & $T_{3\frac{1}{2}}$ and $T_0$\\
Normal space                          & $T_4$ and $T_1$\\
Perfectly $T_4$ space                 & $T_4$ and every closed set is a $G_\delta$ (see \PMlinkname{here}{G_deltaSet}) \\
Perfectly normal space                & $T_1$ and perfectly $T_4$\\
Completely normal space               & $T_5$ and $T_1$\\
\end{tabular}
\end{center}

The following implications hold strictly:
\begin{align*}
    (T_2 \text{ and } T_3) &\Rightarrow T_{2\frac{1}{2}}\\
    (T_3 \text{ and } T_4) &\Rightarrow T_{3\frac{1}{2}}\\
    T_{3\frac{1}{2}}       &\Rightarrow T_3\\
    T_5                    &\Rightarrow T_4
\end{align*}

\begin{align*}
\text{Completely normal } &\Rightarrow \text{ normal }\Rightarrow
\text{ completely regular }\\
&\Rightarrow \text{ regular }\Rightarrow T_{2\frac{1}{2}}\Rightarrow T_2\Rightarrow T_1 \Rightarrow T_0
\end{align*}

\textbf{Remark.} Some authors define $T_3$ spaces in the way we defined regular spaces, and $T_4$ spaces in the way we defined normal spaces (and vice-versa); there is no consensus on this issue.
\medskip

\textbf{Bibliography:} \emph{Counterexamples in Topology}, L. A. Steen, J. A. Seebach Jr., Dover Publications Inc. (New York)
%%%%%
%%%%%
\end{document}
