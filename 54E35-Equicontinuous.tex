\documentclass[12pt]{article}
\usepackage{pmmeta}
\pmcanonicalname{Equicontinuous}
\pmcreated{2013-03-22 18:38:10}
\pmmodified{2013-03-22 18:38:10}
\pmowner{asteroid}{17536}
\pmmodifier{asteroid}{17536}
\pmtitle{equicontinuous}
\pmrecord{6}{41376}
\pmprivacy{1}
\pmauthor{asteroid}{17536}
\pmtype{Definition}
\pmcomment{trigger rebuild}
\pmclassification{msc}{54E35}
\pmclassification{msc}{54C35}
\pmsynonym{equicontinuity}{Equicontinuous}
\pmrelated{Equicontinuous2}

\endmetadata

% this is the default PlanetMath preamble.  as your knowledge
% of TeX increases, you will probably want to edit this, but
% it should be fine as is for beginners.

% almost certainly you want these
\usepackage{amssymb}
\usepackage{amsmath}
\usepackage{amsfonts}

% used for TeXing text within eps files
%\usepackage{psfrag}
% need this for including graphics (\includegraphics)
%\usepackage{graphicx}
% for neatly defining theorems and propositions
%\usepackage{amsthm}
% making logically defined graphics
%%%\usepackage{xypic}

% there are many more packages, add them here as you need them

% define commands here

\begin{document}
\section{Definition}

Let $X$ be a topological space, $(Y, d)$ a metric space and $C(X,Y)$ the set of continuous functions $X \to Y$.

Let $\mathcal{F}$ be a subset of $C(X,Y)$. A function $f \in \mathcal{F}$ is continuous at a point $x_0$ when given $\epsilon > 0$ there is a neighbourhood $U$ of $x_0$ such that $d(f(x),f(x_0)) < \epsilon$ for every $x \in U$. When the same neighbourhood $U$ can be chosen for all functions $f \in \mathcal{F}$, the family $\mathcal{F}$ is said to be \emph{equicontinuous}. More precisely:

$\,$

{\bf Definition -} Let $\mathcal{F}$ be a subset of $C(X,Y)$. The set of functions $\mathcal{F}$ is said to be {\bf equicontinuous at} $x_0 \in X$ if for every $\epsilon >0$ there is a neighbourhood $U$ of $x_0$ such that for every $x \in U$ and every $f \in \mathcal{F}$ we have
\begin{align*}
d(f(x),f(x_0)) < \epsilon
\end{align*}

The set $\mathcal{F}$ is said to be {\bf equicontinuous} if it is equicontinuous at every point $x \in X$.

\section{Examples}
\begin{itemize}
\item A finite set of functions in $C(X, Y)$ is always equicontinuous.
\item When $X$ is also a metric space, a family of functions in $C(X,Y)$ that share the same Lipschitz constant is equicontinuous.
\item The family of functions $\{f_n\}_{n \in \mathbb{N}}$, where $f_n:\mathbb{R} \to \mathbb{R}$ is given by $f_n(x):=\arctan (nx)$ is not equicontinuous at $0$.
\end{itemize}

\section{Properties}

\begin{itemize}
\item If a subset $\mathcal{F} \subseteq C(X, Y)$ is totally bounded under the uniform metric, then $\mathcal{F}$ is equicontinuous.
\item Suppose $X$ is compact. If a sequence of functions $\{f_n\}$ in $C(X, \mathbb{R}^k)$ is equibounded and equicontinuous, then the sequence $\{f_n\}$ has a uniformly convergent subsequence. (\PMlinkname{Arzelà's theorem}{AscoliArzelaTheorem})
\item Let $\{f_n\}$ be a sequence of functions in $C(X, Y)$. If $\{f_n\}$ is equicontinuous and converges pointwise to a function $f:X \to Y$, then $f$ is continuous and $\{f_n\}$ converges to $f$ in the compact-open topology.
\end{itemize}

\begin{thebibliography}{9}
\bibitem{munkres} J. Munkres, \emph{Topology} (2nd edition), Prentice Hall, 1999.
\end{thebibliography}
%%%%%
%%%%%
\end{document}
