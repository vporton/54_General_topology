\documentclass[12pt]{article}
\usepackage{pmmeta}
\pmcanonicalname{T4Space}
\pmcreated{2013-03-22 14:42:15}
\pmmodified{2013-03-22 14:42:15}
\pmowner{matte}{1858}
\pmmodifier{matte}{1858}
\pmtitle{$T4$ space}
\pmrecord{5}{36318}
\pmprivacy{1}
\pmauthor{matte}{1858}
\pmtype{Definition}
\pmcomment{trigger rebuild}
\pmclassification{msc}{54D15}
\pmrelated{SeparationAxioms}
\pmrelated{HowIsNormalityAndT4DefinedInBooks}

\endmetadata

% this is the default PlanetMath preamble.  as your knowledge
% of TeX increases, you will probably want to edit this, but
% it should be fine as is for beginners.

% almost certainly you want these
\usepackage{amssymb}
\usepackage{amsmath}
\usepackage{amsfonts}
\usepackage{amsthm}

\usepackage{mathrsfs}

% used for TeXing text within eps files
%\usepackage{psfrag}
% need this for including graphics (\includegraphics)
%\usepackage{graphicx}
% for neatly defining theorems and propositions
%
% making logically defined graphics
%%%\usepackage{xypic}

% there are many more packages, add them here as you need them

% define commands here

\newcommand{\sR}[0]{\mathbb{R}}
\newcommand{\sC}[0]{\mathbb{C}}
\newcommand{\sN}[0]{\mathbb{N}}
\newcommand{\sZ}[0]{\mathbb{Z}}

 \usepackage{bbm}
 \newcommand{\Z}{\mathbbmss{Z}}
 \newcommand{\C}{\mathbbmss{C}}
 \newcommand{\R}{\mathbbmss{R}}
 \newcommand{\Q}{\mathbbmss{Q}}



\newcommand*{\norm}[1]{\lVert #1 \rVert}
\newcommand*{\abs}[1]{| #1 |}



\newtheorem{thm}{Theorem}
\newtheorem{defn}{Definition}
\newtheorem{prop}{Proposition}
\newtheorem{lemma}{Lemma}
\newtheorem{cor}{Corollary}
\begin{document}
\begin{defn} \cite{steen}
Suppose $X$ is a topological space. Further, suppose that
for any two disjoint 
closed sets $A,B\subseteq X$, there are two disjoint open sets 
$U$ and $V$ such that $A\subseteq U$ and $B\subseteq V$. Then we say 
that $X$ is a \emph{$T_4$ space}.
\end{defn}
                        
\subsubsection*{Notes}
It should be pointed out that  there is no standard convention
for separation axioms in topology. The above definition follows
\cite{steen}. However, in some references (e.g. \cite{kelley})
the meaning of $T_4$ and normal are exchanged.                                                     
                                                                                
\begin{thebibliography}{9}
\bibitem{steen} L.A. Steen, J.A.Seebach, Jr.,
\emph{Counterexamples in topology},
Holt, Rinehart and Winston, Inc., 1970.
\bibitem{kelley}
J.L. Kelley, \emph{General Topology}, D. van Nostrand Company, Inc., 1955.
\end{thebibliography}
%%%%%
%%%%%
\end{document}
