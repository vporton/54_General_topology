\documentclass[12pt]{article}
\usepackage{pmmeta}
\pmcanonicalname{NiemytzkiPlane}
\pmcreated{2013-03-22 13:36:53}
\pmmodified{2013-03-22 13:36:53}
\pmowner{PrimeFan}{13766}
\pmmodifier{PrimeFan}{13766}
\pmtitle{Niemytzki plane}
\pmrecord{7}{34247}
\pmprivacy{1}
\pmauthor{PrimeFan}{13766}
\pmtype{Example}
\pmcomment{trigger rebuild}
\pmclassification{msc}{54-00}
\pmclassification{msc}{54G99}
\pmsynonym{Niemytzki space}{NiemytzkiPlane}

\endmetadata

% this is the default PlanetMath preamble.  as your knowledge
% of TeX increases, you will probably want to edit this, but
% it should be fine as is for beginners.

% almost certainly you want these
\usepackage{amssymb}
\usepackage{amsmath}
\usepackage{amsfonts}

% used for TeXing text within eps files
%\usepackage{psfrag}
% need this for including graphics (\includegraphics)
\usepackage{graphicx}
% for neatly defining theorems and propositions
%\usepackage{amsthm}
% making logically defined graphics
%%%\usepackage{xypic}

% there are many more packages, add them here as you need them

% define commands here
\def\sse{\subseteq}
\def\bigtimes{\mathop{\mbox{\Huge $\times$}}}
\def\impl{\Rightarrow}
\def\R{\mathbb{R}}
\begin{document}
Let $\Gamma$ be the Euclidean half plane $\Gamma =\{(x,y)\mid y\ge0\}\sse\R^2$,
with the usual subspace topology. We enrich the topology on $\Gamma$ by throwing
in open sets of the form $\{(x,0)\}\cup B_r(x,r)$, that is an open ball of radius
$r$ around $(x,r)$ together with its point tangent to $\R\times\{0\}$
(Fig.~\ref{nbhd}).
\begin{figure}[h]
\begin{center}
  \includegraphics{nbhd}
  \caption{A new open set in $\Gamma$. It consists of an open
    disc and the point tangent to $y=0$.\label{nbhd}}
\end{center}
\end{figure}
The space $\Gamma$ endowed with the enriched topology is called the
\emph{Niemytzki plane}.

Some miscellaneous properties of the Niemytzki plane are
\begin{itemize}
\item the subspace $\R\times\{0\}$ of $\Gamma$ is discrete, hence
  the only convergent sequences in this subspace are constant ones;
\item it is Hausdorff;
\item it is completely regular;
\item it is not normal.
\end{itemize}
%%%%%
%%%%%
\end{document}
