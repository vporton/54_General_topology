\documentclass[12pt]{article}
\usepackage{pmmeta}
\pmcanonicalname{Completion}
\pmcreated{2013-03-22 12:39:22}
\pmmodified{2013-03-22 12:39:22}
\pmowner{djao}{24}
\pmmodifier{djao}{24}
\pmtitle{completion}
\pmrecord{8}{32923}
\pmprivacy{1}
\pmauthor{djao}{24}
\pmtype{Definition}
\pmcomment{trigger rebuild}
\pmclassification{msc}{54D35}
\pmclassification{msc}{13J10}
\pmclassification{msc}{26-00}

% this is the default PlanetMath preamble.  as your knowledge
% of TeX increases, you will probably want to edit this, but
% it should be fine as is for beginners.

% almost certainly you want these
\usepackage{amssymb}
\usepackage{amsmath}
\usepackage{amsfonts}

% used for TeXing text within eps files
%\usepackage{psfrag}
% need this for including graphics (\includegraphics)
%\usepackage{graphicx}
% for neatly defining theorems and propositions
%\usepackage{amsthm}
% making logically defined graphics
\usepackage[all]{xypic} 

% there are many more packages, add them here as you need them

% define commands here
\newcommand{\p}{{\mathfrak{p}}}
\newcommand{\m}{{\mathfrak{m}}}
\newcommand{\M}{{\mathfrak{M}}}
\renewcommand{\P}{{\mathfrak{P}}}
\newcommand{\C}{\mathbb{C}}
\newcommand{\R}{\mathbb{R}}
\newcommand{\Z}{\mathbb{Z}}
\newcommand{\Q}{\mathbb{Q}}
\newcommand{\N}{\mathbb{N}}
\renewcommand{\H}{\mathcal{H}}
\newcommand{\A}{\mathcal{A}}
\renewcommand{\c}{\mathcal{C}}
\renewcommand{\O}{\mathcal{O}}
\newcommand{\D}{\mathcal{D}}
\newcommand{\lra}{\longrightarrow}
\renewcommand{\div}{\mid}
\newcommand{\res}{\operatorname{res}}
\newcommand{\Spec}{\operatorname{Spec}}
\newcommand{\Gal}{\operatorname{Gal}}
\newcommand{\id}{\operatorname{id}}
\newcommand{\diff}{\operatorname{diff}}
\newcommand{\incl}{\operatorname{incl}}
\newcommand{\Hom}{\operatorname{Hom}}
\renewcommand{\Re}{\operatorname{Re}}
\newcommand{\intersect}{\cap}
\newcommand{\union}{\cup}
\newcommand{\bigintersect}{\bigcap}
\newcommand{\bigunion}{\bigcup}
\newcommand{\ilim}{\,\underset{\longleftarrow}{\lim}\,}
\begin{document}
Let $(X,d)$ be a metric space. Let $\bar{X}$ be the set of all Cauchy
sequences $\{x_n\}_{n \in \N}$ in $X$. Define an equivalence relation
$\sim$ on $\bar{X}$ by setting $\{x_n\} \sim \{y_n\}$ if the
interleave sequence of the sequences $\{x_n\}$ and $\{y_n\}$ is also a
Cauchy sequence. The {\em completion} of $X$ is defined to be the set
$\hat{X}$ of equivalence classes of $\bar{X}$ modulo $\sim$.

The metric $d$ on $X$ extends to a metric on $\hat{X}$ in the
following manner:
$$
d(\{x_n\},\{y_n\}) := \lim_{n \to \infty} d(x_n,y_n),
$$
where $\{x_n\}$ and $\{y_n\}$ are representative Cauchy sequences of
elements in $\hat{X}$. The definition of $\sim$ is tailored so that
the limit in the above definition is well defined, and the fact that these
sequences are Cauchy, together with the fact that $\R$ is complete,
ensures that the limit exists. The space $\hat{X}$ with this metric is of course a complete metric space.

The original metric space $X$ is isometric to the subset of $\hat{X}$ consisting of equivalence classes of constant sequences.

Note the similarity between the construction of $\hat{X}$ and the
construction of $\R$ from $\Q$. The process used here is the same as
that used to construct the real numbers $\R$, except for the minor
detail that one can not use the terminology of metric spaces in the
construction of $\R$ itself because it is necessary to construct $\R$
in the first place before one can define metric spaces.

\section{Metric spaces with richer structure}

If the metric space $X$ has an algebraic structure, then in many
cases this algebraic structure carries through unchanged to $\hat{X}$
simply by applying it one element at a time to sequences in $X$. We
will not attempt to state this principle precisely, but we will
mention the following important instances:

\begin{enumerate}
\item If $(X,\cdot)$ is a topological group, then $\hat{X}$ is also a
  topological group with multiplication defined by
$$
\{x_n\} \cdot \{y_n\} = \{x_n \cdot y_n\}.
$$
\item If $X$ is a topological ring, then addition and multiplication
  extend to $\hat{X}$ and make the completion into a topological ring.
\item If $F$ is a field with a valuation $v$, then the completion of
  $F$ with respect to the metric imposed by $v$ is a topological
  field, denoted $F_v$ and called the completion of $F$ at $v$.
\end{enumerate}

\section{Universal property of completions}

The completion $\hat{X}$ of $X$ satisfies the following universal property: for every uniformly continuous map $f: X \lra Y$ of $X$ into a complete metric space $Y$, there exists a unique lifting of $f$ to a continuous map $\hat{f}: \hat{X} \lra Y$ making the diagram
$$
\xymatrix{
X \ar[rr]^f \ar[dr] & & Y \\
& \hat{X} \ar[ur]_{\hat{f}}
}
$$
commute. Up to isomorphism, the completion of $X$ is the unique metric space satisfying this property. The ability to extend uniformly continuous functions from $X$ to $\hat{X}$ is often the reason why algebraic structures on $X$ extend to $\hat{X}$ as described in the previous section.
%%%%%
%%%%%
\end{document}
