\documentclass[12pt]{article}
\usepackage{pmmeta}
\pmcanonicalname{IntervalsAreConnected}
\pmcreated{2013-03-22 18:32:49}
\pmmodified{2013-03-22 18:32:49}
\pmowner{joking}{16130}
\pmmodifier{joking}{16130}
\pmtitle{intervals are connected}
\pmrecord{5}{41267}
\pmprivacy{1}
\pmauthor{joking}{16130}
\pmtype{Example}
\pmcomment{trigger rebuild}
\pmclassification{msc}{54D05}

\endmetadata

% this is the default PlanetMath preamble.  as your knowledge
% of TeX increases, you will probably want to edit this, but
% it should be fine as is for beginners.

% almost certainly you want these
\usepackage{amssymb}
\usepackage{amsmath}
\usepackage{amsfonts}

% used for TeXing text within eps files
%\usepackage{psfrag}
% need this for including graphics (\includegraphics)
%\usepackage{graphicx}
% for neatly defining theorems and propositions
%\usepackage{amsthm}
% making logically defined graphics
%%%\usepackage{xypic}

% there are many more packages, add them here as you need them

% define commands here

\begin{document}
We wish to show that intervals (with standard topology) are connected. In order to this, we will prove that the space of real numbers $\mathbb{R}$ is connected. First we need a lemma.

Let $(X,d)$ be a metric space. Recall that for $x\in X$ and $r\in\mathbb{R}^{+}$ we have $$B(x,r)=\{y\in X\ |\ d(x,y)<r\}.$$

\textbf{Lemma}. Let $(X,d)$ be a metric space and $R\subset\mathbb{R}^{+}$ such that $R$ is nonempty and bounded. Then for any $x\in X$ we have
$$\bigcup_{r\in R}B(x,r)=B(x,\mathrm{sup}(R)).$$

\textit{Proof}. Assume that $y\in\bigcup_{r\in R}B(x,r)$. Then there is $r_0\in R$ such that $d(x,y)<r_0$ and thus $d(x,y)<\mathrm{sup}(R)$, so $y\in B(x,\mathrm{sup}(R))$.

Now assume that $y\in B(x,\mathrm{sup}(R))$. Then $d(x,y)<\mathrm{sup}(R)$ and it follows (from the definition of supremum) that there is $r_0\in R$ such that $d(x,y)<r_0$ and therefore $y\in B(x,r_0)\subset\bigcup_{r\in R}B(x,r)$, which completes the proof. $\square$


\textbf{Proposition}. The space of real numbers is connected.

\textit{Proof}. Assume that $U,V \subseteq\mathbb{R}$ are open subsets of $\mathbb{R}$ such that $U\cap V = \emptyset$ and $U\cup V = \mathbb{R}$. Furthermore assume that $U \neq \emptyset$ and take any $x_{0} \in U$. Then (since $U$ is open) there is $r_{0}\in\mathbb{R}$ such that the open ball $$B(x_{0},r_{0})=\{ x\in \mathbb{R}\ |\ |x-x_{0}|<r_{0}\} $$
is contained in $U$. Consider the set 
$$R=\{ r\in\mathbb{R}^{+}\ |\ B(x_0,r)\subseteq U\} .$$
Thus $R$ is nonempty.


Assume that $R$ is bounded. Denote by $s=\mathrm{sup}(R)<\infty$. We can apply the lemma:
$$\bigcup_{r\in R} B(x_0,r)=B(x_0,s).$$
Thus (due to the definition of $R$) $B(x_0,s)$ is a maximal open ball (with the center in $x_0$) which is contained in $U$. Now $$B(x_0,s)=(a,b)$$
for some $a,b\in\mathbb{R}$. Since $(a,b)$ is maximal then $a\not\in U$ or $b\not\in U$. Indeed, if both $a\in U$ and $b\in U$, then (since $U$ is open) small neighbourhoods of $a$ and $b$ are also contained in $U$, so $(a-\epsilon,b+\epsilon)$ is contained in $U$ (for some $\epsilon>0$), but $(a,b)$ was maximal. Contradiction.

Without loss of generality we can assume that $b\not\in U$. Then $b\in V$, because $U\cup V=\mathbb{R}$. But then (since $V$ is open) there is $c\in\mathbb{R}$ such that $a<c<b$ and $c\in V$. Thus $U\cap V\neq\emptyset$. Contradiction. Therefore $R$ is unbounded. 


Take any unbounded sequence $(a_n)_{n=1}^{\infty}$ from $R$. Then we have
$$\mathbb{R}=\bigcup_{n=1}^{\infty} B(x_0,a_n)\subseteq U$$
and thus $U=\mathbb{R}$, so $V=\emptyset$. This completes the proof. $\square$

\textbf{Corollary}. For any $a,b\in\mathbb{R}$ such that $a<b$ intervals $(a,b)$, $[a,b)$, $(a,b]$ and $[a,b]$ are connected.

\textit{Proof}. One can easily show that intervals are continous image of $\mathbb{R}$ and therefore intervals are connected.
%%%%%
%%%%%
\end{document}
