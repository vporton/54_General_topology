\documentclass[12pt]{article}
\usepackage{pmmeta}
\pmcanonicalname{ProofOfCauchyConditionForLimitOfFunction}
\pmcreated{2013-03-22 18:59:08}
\pmmodified{2013-03-22 18:59:08}
\pmowner{puff}{4175}
\pmmodifier{puff}{4175}
\pmtitle{proof of Cauchy condition for limit of function}
\pmrecord{8}{41851}
\pmprivacy{1}
\pmauthor{puff}{4175}
\pmtype{Proof}
\pmcomment{trigger rebuild}
\pmclassification{msc}{54E35}
\pmclassification{msc}{26A06}
\pmclassification{msc}{26B12}

% this is the default PlanetMath preamble.  as your knowledge
% of TeX increases, you will probably want to edit this, but
% it should be fine as is for beginners.

% almost certainly you want these
\usepackage{amssymb}
\usepackage{amsmath}
\usepackage{amsfonts}

% used for TeXing text within eps files
%\usepackage{psfrag}
% need this for including graphics (\includegraphics)
%\usepackage{graphicx}
% for neatly defining theorems and propositions
%\usepackage{amsthm}
% making logically defined graphics
%%%\usepackage{xypic}

% there are many more packages, add them here as you need them

% define commands here

\begin{document}
The forward direction is \PMlinkescapetext{simple}. Assume that $\lim_{x \to x_0}f(x) =
L$. Then given $\epsilon$ there is a $\delta$ such that
\begin{equation*}
  |f(u)- L| < \epsilon/2 \text{ when } 0 < |u-x_0| < \delta.
\end{equation*}

Now for $0 < |u - x_0 | < \delta$ and $ 0< |v- x_0 | < \delta$ we have
\begin{equation*}
   |f(u)- L| < \epsilon/2 \text{ and }  |f(v)-L| < \epsilon/2
\end{equation*}
and so

\begin{equation*}
  |f(u)-f(v)| = | f(u)- L -(f(v)-L)| \leq
  |f(u)-L|+|f(v)-L| < \epsilon/2+\epsilon/2 = \epsilon.
\end{equation*}

We prove the reverse by contradiction.
Assume that the condition holds.
Now suppose  that $\lim_{x \to x_0}f(x)$ does not exist. This means that for
any $l$ 
and any $\epsilon$ sufficiently small then for any $\delta>0$ there is
$x_l$ such that $0<|xl-x_0|< \delta~\text{and}~|f(x_l)-l| \geq \epsilon$.
For any such $\epsilon$ choose $u$ such that $0 < |u-x_0| < \delta $ and
put $l=f(v)$ then substituting in  the condition with $u=x_l$ we get
$|f(x_l)-l| < \epsilon$. A contradiction. 

%%%%%
%%%%%
\end{document}
