\documentclass[12pt]{article}
\usepackage{pmmeta}
\pmcanonicalname{InitialTopology}
\pmcreated{2013-03-22 15:30:26}
\pmmodified{2013-03-22 15:30:26}
\pmowner{kompik}{10588}
\pmmodifier{kompik}{10588}
\pmtitle{initial topology}
\pmrecord{11}{37368}
\pmprivacy{1}
\pmauthor{kompik}{10588}
\pmtype{Definition}
\pmcomment{trigger rebuild}
\pmclassification{msc}{54B99}
%\pmkeywords{coarser}
\pmrelated{producttopology}
\pmrelated{subspacetopology}
\pmrelated{ProductTopology}
\pmrelated{IdentificationTopology}
\pmrelated{Coarser}

\endmetadata

% this is the default PlanetMath preamble. as your knowledge
% of TeX increases, you will probably want to edit this, but
% it should be fine as is for beginners.

% almost certainly you want these
\usepackage{amssymb}
\usepackage{amsmath}
\usepackage{amsfonts}
\usepackage{amsthm}

% used for TeXing text within eps files
%\usepackage{psfrag}
% need this for including graphics (\includegraphics)
%\usepackage{graphicx}
% for neatly defining theorems and propositions
%
% making logically defined graphics
%%%\usepackage{xypic}

% there are many more packages, add them here as you need them

% define commands here

\newcommand{\sR}[0]{\mathbb{R}}
\newcommand{\sC}[0]{\mathbb{C}}
\newcommand{\sN}[0]{\mathbb{N}}
\newcommand{\sZ}[0]{\mathbb{Z}}

\usepackage{bbm}
\newcommand{\Z}{\mathbbmss{Z}}
\newcommand{\C}{\mathbbmss{C}}
\newcommand{\R}{\mathbbmss{R}}
\newcommand{\Q}{\mathbbmss{Q}}



\newcommand*{\norm}[1]{\lVert #1 \rVert}
\newcommand*{\abs}[1]{| #1 |}

\newcommand{\Map}[3]{#1\colon#2\to#3}
\begin{document}
Let $X_i$, $i\in I$ be any family of topological spaces.
We say that a topology $\mathcal T$ on $X$ is initial with respect to the family of mappings $\Map {f_i}X{X_i}$, $i\in I$, if $\mathcal T$ is the coarsest topology on $X$ which makes all $f_i$'s continuous.

The initial topology is characterized by the condition that a map $\Map gYX$ is continuous if and only if every $\Map {f_i \circ g}Y{X_i}$ is continuous.

Sets $\mathcal S=\{f_i^{-1}(U): U$ is open in $X_i\}$ form a subbase for the initial topology, their finite intersections form a base.

E.g. the product topology is initial with respect to the \PMlinkname{projections}{GeneralizedCartesianProduct} and a subspace topology is initial with respect to the embedding.

The initial topology is sometimes called topology generated by a family of mappings \cite{engelking}, weak topology \cite{willard} or projective topology. (The \PMlinkescapetext{term} weak topology is used mainly in functional analysis.)

From the viewpoint of category theory, the initial topology is an initial source. (Initial structures, which are a natural generalization of the initial topology, play an important r\^{o}le in topological categories and categorical topology.)


\begin{thebibliography}{9}
\bibitem{ahs}
J.~Ad\'amek, H.~Herrlich, and G.~Strecker, \emph{Abstract and concrete categories}, Wiley,
New York, 1990.

\bibitem{engelking}
R.~Engelking, \emph{General topology}, PWN, Warsaw, 1977.

\bibitem{husek}
M.~Hu\v{s}ek, \emph{Categorical topology}, Encyclopedia of General Topology (K.~P. Hart,
J.-I. Nagata, and J.~E. Vaughan, eds.), Elsevier, 2003, pp.~70--71.

\bibitem{willard}
S.~Willard, \emph{General topology}, Addison-Wesley, Massachussets, 1970.

\bibitem{wikiinit} Wikipedia's entry on \PMlinkexternal{Initial topology}{http://en.wikipedia.org/wiki/Initial_topology}
\end{thebibliography}
%%%%%
%%%%%
\end{document}
