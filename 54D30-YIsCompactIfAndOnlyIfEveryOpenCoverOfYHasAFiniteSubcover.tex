\documentclass[12pt]{article}
\usepackage{pmmeta}
\pmcanonicalname{YIsCompactIfAndOnlyIfEveryOpenCoverOfYHasAFiniteSubcover}
\pmcreated{2013-03-22 13:34:07}
\pmmodified{2013-03-22 13:34:07}
\pmowner{mathcam}{2727}
\pmmodifier{mathcam}{2727}
\pmtitle{$Y$ is compact if and only if every open cover of $Y$ has a finite subcover}
\pmrecord{6}{34179}
\pmprivacy{1}
\pmauthor{mathcam}{2727}
\pmtype{Theorem}
\pmcomment{trigger rebuild}
\pmclassification{msc}{54D30}

\endmetadata

% this is the default PlanetMath preamble.  as your knowledge
% of TeX increases, you will probably want to edit this, but
% it should be fine as is for beginners.

% almost certainly you want these
\usepackage{amssymb}
\usepackage{amsmath}
\usepackage{amsfonts}

% used for TeXing text within eps files
%\usepackage{psfrag}
% need this for including graphics (\includegraphics)
%\usepackage{graphicx}
% for neatly defining theorems and propositions
%\usepackage{amsthm}
% making logically defined graphics
%%%\usepackage{xypic}

% there are many more packages, add them here as you need them

% define commands here
\begin{document}
{\bf Theorem. }\\
Let $X$ be a topological space and $Y$ a subset of $X$. Then the following 
statements are equivalent.
\begin{enumerate}
\item $Y$ is compact as a subset of $X$. 
\item Every open cover of $Y$ (with open sets in $X$) has a finite
subcover.
\end{enumerate}

\emph{Proof.} 
Suppose $Y$ is compact, and $\{U_i\}_{i\in I}$ is an arbitrary 
open cover of $Y$, where $U_i$ are open sets in $X$. Then 
$\{U_i\cap Y\}_{i\in I}$  is a collection of open sets in $Y$ with 
union $Y$. Since $Y$ is compact, there is a finite subset $J\subset I$
such that $Y=\cup_{i\in J} (U_i\cap Y)$. Now 
$Y=(\cup_{i\in J} U_i)\cap Y \subset \cup_{i\in J} U_i$, so
$\{U_i\}_{i\in J}$ is finite open cover of $Y$. 

Conversely, suppose every open cover of $Y$ has a finite subcover, 
and $\{U_i\}_{i\in I}$ is an arbitrary collection of open sets 
(in $Y$) with union $Y$. By the definition of the subspace topology, 
each $U_i$ is of the form $U_i = V_i\cap Y$ for some open set 
$V_i$ in $X$. Now $U_i \subset V_i$, so $\{V_i\}_{i\in I}$ is a cover
of $Y$ by open sets in $X$. By assumption, it has a finite subcover
$\{V_i\}_{i\in J}$. It follows that 
$\{U_i\}_{i\in J}$ covers $Y$, and $Y$ is compact. $\Box$


The above proof follows the proof given in \cite{ikenaga}.
\begin{thebibliography}{9}
 \bibitem{ikenaga}
 B.Ikenaga, \emph{Notes on Topology}, August 16, 2000, available online
 \PMlinkexternal{http://www.millersv.edu/~bikenaga/topology/topnote.html}{http://www.millersv.edu/~bikenaga/topology/topnote.html}.
 \end{thebibliography}
%%%%%
%%%%%
\end{document}
