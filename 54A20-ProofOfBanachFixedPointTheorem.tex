\documentclass[12pt]{article}
\usepackage{pmmeta}
\pmcanonicalname{ProofOfBanachFixedPointTheorem}
\pmcreated{2013-03-22 13:08:34}
\pmmodified{2013-03-22 13:08:34}
\pmowner{asteroid}{17536}
\pmmodifier{asteroid}{17536}
\pmtitle{proof of Banach fixed point theorem}
\pmrecord{5}{33581}
\pmprivacy{1}
\pmauthor{asteroid}{17536}
\pmtype{Proof}
\pmcomment{trigger rebuild}
\pmclassification{msc}{54A20}
\pmclassification{msc}{47H10}
\pmclassification{msc}{54H25}

\endmetadata

% this is the default PlanetMath preamble.  as your knowledge
% of TeX increases, you will probably want to edit this, but
% it should be fine as is for beginners.

% almost certainly you want these
\usepackage{amssymb}
\usepackage{amsmath}
\usepackage{amsfonts}

% used for TeXing text within eps files
%\usepackage{psfrag}
% need this for including graphics (\includegraphics)
%\usepackage{graphicx}
% for neatly defining theorems and propositions
%\usepackage{amsthm}
% making logically defined graphics
%%%\usepackage{xypic}

% there are many more packages, add them here as you need them

% define commands here
\begin{document}
Let $(X,d)$ be a non-empty, complete metric space, and let $T$ be a
contraction mapping on $(X,d)$ with constant $q$.  Pick an arbitrary
$x_0 \in X$, and define the sequence $(x_n)_{n=0}^{\infty}$ by
$x_n:=T^nx_0$.  Let $a:=d(Tx_0,x_0)$.  We first show by induction that
for any $n\ge 0$,
$$
d(T^nx_0,x_0)\le\frac{1-q^n}{1-q} a.
$$
For $n=0$, this is obvious.  For any $n\ge 1$, suppose that
$d(T^{n-1}x_0,x_0)\le\frac{1-q^{n-1}}{1-q}a$. Then
\begin{eqnarray*}
d(T^nx_0,x_0)&\le&d(T^nx_0,T^{n-1}x_0)+d(x_0,T^{n-1}x_0)\\
&\le&q^{n-1}d(Tx_0,x_0)+\frac{1-q^{n-1}}{1-q}a\\
&=&\frac{q^{n-1}-q^n}{1-q}a+\frac{1-q^{n-1}}{1-q}a\\
&=&\frac{1-q^n}{1-q}a
\end{eqnarray*}
by the triangle inequality and repeated application of the property
$d(Tx,Ty)\le qd(x,y)$ of $T$.  By induction, the inequality holds for
all $n \ge 0$.\\
\\
Given any $\epsilon>0$, it is possible to choose a natural number $N$
such that $\frac{q^n}{1-q}a<\epsilon$ for all $n\ge N$, because
$\frac{q^n}{1-q}a\to 0$ as $n\to\infty$.  Now, for any $m,n\ge N$ (we
may assume that $m\ge n$),
\begin{eqnarray*}
d(x_m,x_n)&=&d(T^mx_0,T^nx_0)\\
&\le&q^nd(T^{m-n}x_0,x_0)\\
&\le&q^n\frac{1-q^{m-n}}{1-q}a\\
&<&\frac{q^n}{1-q}a<\epsilon,
\end{eqnarray*}
so the sequence $(x_n)$ is a Cauchy sequence.  Because $(X,d)$ is
complete, this implies that the sequence has a limit in $(X,d)$;
define $x^*$ to be this limit.  We now prove that $x^*$ is a fixed
point of $T$.  Suppose it is not, then $\delta:=d(Tx^*,x^*)>0$.
However, because $(x_n)$ converges to $x^*$,  there is a natural
number $N$ such that $d(x_n,x^*)<\delta/2$ for all $n\ge N$.  Then
\begin{eqnarray*}
d(Tx^*,x^*)&\le&d(Tx^*,x_{N+1})+d(x^*,x_{N+1})\\
&\le&qd(x^*,x_N)+d(x^*,x_{N+1})\\
&<&\delta/2+\delta/2=\delta,
\end{eqnarray*}
contradiction.  So $x^*$ is a fixed point of $T$.  It is also unique.
Suppose there is another fixed point $x'$ of $T$; because $x'\neq
x^*$, $d(x',x^*)>0$.  But then
$$
d(x',x^*)=d(Tx',Tx^*)\le qd(x',x^*)<d(x',x^*),
$$
contradiction.  Therefore, $x^*$ is the unique fixed point of $T$.
%%%%%
%%%%%
\end{document}
