\documentclass[12pt]{article}
\usepackage{pmmeta}
\pmcanonicalname{ApplicationsOfUrysohnsLemmaToLocallyCompactHausdorffSpaces}
\pmcreated{2013-03-22 18:33:31}
\pmmodified{2013-03-22 18:33:31}
\pmowner{azdbacks4234}{14155}
\pmmodifier{azdbacks4234}{14155}
\pmtitle{applications of Urysohn's Lemma to locally compact Hausdorff spaces}
\pmrecord{23}{41281}
\pmprivacy{1}
\pmauthor{azdbacks4234}{14155}
\pmtype{Topic}
\pmcomment{trigger rebuild}
\pmclassification{msc}{54D15}
%\pmkeywords{Urysohn}
%\pmkeywords{Tietze}
%\pmkeywords{continuous}
%\pmkeywords{compact}
%\pmkeywords{locally compact}
%\pmkeywords{compacification}
%\pmkeywords{support}
\pmrelated{UrysohnsLemma}
\pmrelated{TietzeExtensionTheorem}
\pmrelated{VanishAtInfinity}
\pmrelated{SupportOfFunction}
\pmrelated{LocallyCompact}
\pmrelated{T2Space}
\pmrelated{NormalTopologicalSpace}

%packages
\usepackage{amsmath,mathrsfs,amsfonts,amsthm}
%theorem environments
\theoremstyle{plain}
\newtheorem{thm}{Theorem}
\newtheorem{lem}{Lemma}
\newtheorem{cor}{Corollary}
\newtheorem{prop}{Proposition}
%delimiters
\newcommand{\set}[1]{\{#1\}}
\newcommand{\medset}[1]{\big\{#1\big\}}
\newcommand{\bigset}[1]{\bigg\{#1\bigg\}}
\newcommand{\Bigset}[1]{\Bigg\{#1\Bigg\}}
\newcommand{\abs}[1]{\vert#1\vert}
\newcommand{\medabs}[1]{\big\vert#1\big\vert}
\newcommand{\bigabs}[1]{\bigg\vert#1\bigg\vert}
\newcommand{\Bigabs}[1]{\Bigg\vert#1\Bigg\vert}
\newcommand{\norm}[1]{\Vert#1\Vert}
\newcommand{\mednorm}[1]{\big\Vert#1\big\Vert}
\newcommand{\bignorm}[1]{\bigg\Vert#1\bigg\Vert}
\newcommand{\Bignorm}[1]{\Bigg\Vert#1\Bigg\Vert}
\newcommand{\vbrack}[1]{\langle#1\rangle}
\newcommand{\medvbrack}[1]{\big\langle#1\big\rangle}
\newcommand{\bigvbrack}[1]{\bigg\langle#1\bigg\rangle}
\newcommand{\Bigvbrack}[1]{\Bigg\langle#1\Bigg\rangle}
\newcommand{\sbrack}[1]{[#1]}
\newcommand{\medsbrack}[1]{\big[#1\big]}
\newcommand{\bigsbrack}[1]{\bigg[#1\bigg]}
\newcommand{\Bigsbrack}[1]{\Bigg[#1\Bigg]}
%operators
\DeclareMathOperator{\Hom}{Hom}
\DeclareMathOperator{\Tor}{Tor}
\DeclareMathOperator{\Ext}{Ext}
\DeclareMathOperator{\Aut}{Aut}
\DeclareMathOperator{\End}{End}
\DeclareMathOperator{\Inn}{Inn}
\DeclareMathOperator{\lcm}{lcm}
\DeclareMathOperator{\ord}{ord}
\DeclareMathOperator{\rank}{rank}
\DeclareMathOperator{\tr}{tr}
\DeclareMathOperator{\Mat}{Mat}
\DeclareMathOperator{\Gal}{Gal}
\DeclareMathOperator{\GL}{GL}
\DeclareMathOperator{\SL}{SL}
\DeclareMathOperator{\SO}{SO}
\DeclareMathOperator{\ann}{ann}
\DeclareMathOperator{\im}{im}
\DeclareMathOperator{\Char}{char}
\DeclareMathOperator{\Spec}{Spec}
\DeclareMathOperator{\supp}{supp}
\DeclareMathOperator{\diam}{diam}
\DeclareMathOperator{\Ind}{Ind}
\DeclareMathOperator{\vol}{vol}

\begin{document}
Let $X$ be a locally compact Hausdorff space (LCH space) and $X^*$ its one-point compactification. We employ the following notation:
\begin{itemize}
\item $C(X)$ denotes the set of continuous complex functions on $X$;
\item $C_b(X)$ denotes the set of continuous and bounded complex functions on $X$;
\item $C_0(X)$ denotes the set of continuous complex functions on $X$ which vanish at infinity;
\item $C_c(X)$ denotes the set of continuous complex functions on $X$ with compact support
\end{itemize}
Note that we have $C_c(X)\subseteq C_0(X)\subseteq C_b(X)\subseteq C(X)$, and that when we replace $X$ with $X^*$ (in general, when $X$ is compact), these four classes of functions coincide.\\\\
Now, while Urysohn's Lemma does not directly apply to $X$ (since $X$ need not in general be normal), it does apply to $X^*$, for being compact Hausdorff, $X^*$ \emph{is} necessarily normal. One may therefore indirectly apply Urysohn's Lemma to $X$ by way of $X^*$ to obtain various results asserting the existence of certain continuous functions on $X$ with prescribed properties. The following results and their proofs illustrate this technique and are frequently useful in analysis. 
\begin{prop}
If $K\subseteq U\subseteq X$ with $K$ compact and $U$ open, then there exists an open subset $V$ of $X$ with compact closure such that $K\subseteq V\subseteq\overline{V}\subseteq U$.
\end{prop}
\begin{proof}
Since $K$ is a compact subset of the Hausdorff space $X^*$, it is closed, and because $X$ is open in $X^*$, $U$ is as well. Therefore, by normality, there exists an open subset $V$ of $X^*$ such that $K\subseteq V\subseteq\overline{V}\subseteq U$ (note that the closure of $V$ in $X^*$ coincides with that of $V$ in $X$, since the former set is contained in $X$ and the latter set is equal to the former intersected with $X$). As $\overline{V}$ is closed in $X^*$, it is compact, and because $V$ is open in $X^*$ and $V\subseteq X$, $V$ is open in $X$. Thus $V$ possesses the desired properties.
\end{proof}
\begin{cor}
For each $x\in X$ and each open subset $U$ of $X$ containing $x$, there exists an open subset $V$ of $X$ with compact closure such that $x\in V$ and $\overline{V}\subseteq U$.
\end{cor}
\begin{proof}
Take $K=\set{x}$ in the preceding proposition. 
\end{proof}
\begin{thm}
(Urysohn's Lemma for LCH Spaces)
If $K\subseteq U\subseteq X$ with $K$ compact and $U$ open, then there exists $f\in C_c(X)$ such that $0\leq f\leq 1$, $f\vert_K\equiv 1$, and $\supp f\subseteq U$.
\end{thm}
\begin{proof}
By the first Proposition, there exists an open subset $V$ of $X$ with compact closure such that $K\subseteq V\subseteq\overline{V}\subseteq U$; since $K$ and $X^*-V$ are disjoint closed subsets of the normal space $X^*$, Urysohn's Lemma furnishes $g\in C(X^*)$ such that $0\leq g\leq 1$, $g\vert_K\equiv 1$, and $g\vert_{X^*-V}\equiv 0$. Put $f=g\vert_X$. Then $f\in C(X)$, $0\leq f\leq 1$, and $f\vert_K\equiv 1$. Moreover, $f$ vanishes outside $\overline{V}$ because $g$ does, so $\set{x\in X:f(x)\neq 0}\subseteq\overline{V}\subseteq U$; since $\overline{V}$ is compact, and consequently closed, the last inclusion gives $\supp f\subseteq\overline{V}\subseteq U$ and $f\in C_c(X)$.
\end{proof}
\begin{thm}
(Tietze Extension Theorem for LCH Spaces)
If $K\subseteq X$ is compact and $f\in C(K)$ is real, then there exists a real $g\in C_c(X)$ extending $f$.
\end{thm}
\begin{cor}
$C_0(X)$ is the uniform closure of $C_c(X)$ in $C_b(X)$.
\end{cor}
\begin{proof}
We first show that $C_0(X)$ is closed in $C_b(X)$. To this end, assume that $(f_n)_{n=1}^\infty$ is a uniformly convergent sequence of functions in $C_0(X)$ with limit $f$ and let $\epsilon>0$ be given. Select $N\in\mathbb{Z}^+$ such that $\norm{f-f_N}_\infty<\epsilon/2$, and select a compact subset $K$ of $X$ such that $\abs{f_N}<\epsilon/2$ for $x\in X-K$. We then have, for all such $x$,
\begin{equation*}
\abs{f(x)}=\abs{f(x)-f_N(x)+f_N(x)}\leq\abs{f(x)-f_N(x)}+\abs{f_N(x)}\leq\norm{f-f_N}_\infty+\abs{f_N(x)}<
\epsilon\text{.}
\end{equation*}
Thus $f$ vanishes at infinity; since the uniform limit of continuous functions is continuous, we obtain $f\in C_0(X)$, whence $C_0(X)$ is closed. It remains to establish the density of $C_c(X)$ in $C_0(X)$. Given $f\in C_0(X)$ and $\epsilon>0$, select a compact subset $K$ of $X$ such that $\abs{f(x)}<\epsilon/2$ for $x\in X-K$. By Theorem 1, there exists $g\in C_c(X)$ with range in $[0,1]$ satisfying $g\vert_K\equiv 1$. The function $h=fg$ is continuous and supported inside $\supp g$, hence lies in $C_c(X)$; moreover, if $x\in K$, then we have $\abs{f(x)-h(x)}=\abs{f(x)-f(x)}=0$, while if $x\notin K$, then
\begin{equation*}
\abs{f(x)-h(x)}=\abs{f(x)-f(x)g(x)}=\abs{f(x)}\abs{1-g(x)}\leq\abs{f(x)}<\dfrac{\epsilon}{2}\text{.}
\end{equation*}
It follows that $\norm{f-h}_\infty<\epsilon$, hence that $f\in\overline{C_c(X)}$, completing the proof.
\end{proof}

%%%%%
%%%%%
\end{document}
