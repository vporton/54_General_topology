\documentclass[12pt]{article}
\usepackage{pmmeta}
\pmcanonicalname{Sheaf}
\pmcreated{2013-03-22 12:37:05}
\pmmodified{2013-03-22 12:37:05}
\pmowner{djao}{24}
\pmmodifier{djao}{24}
\pmtitle{sheaf}
\pmrecord{14}{32878}
\pmprivacy{1}
\pmauthor{djao}{24}
\pmtype{Definition}
\pmcomment{trigger rebuild}
\pmclassification{msc}{54B40}
\pmclassification{msc}{18F20}
\pmclassification{msc}{14F05}
\pmrelated{Stalk}
\pmrelated{PrimeSpectrum}
\pmrelated{Sheafification2}
\pmrelated{Sheaf2}
\pmdefines{presheaf}
\pmdefines{section}
\pmdefines{morphism of sheaves}
\pmdefines{isomorphism of sheaves}
\pmdefines{sheaf isomorphism}

\endmetadata

% this is the default PlanetMath preamble.  as your knowledge
% of TeX increases, you will probably want to edit this, but
% it should be fine as is for beginners.

% almost certainly you want these
\usepackage{amssymb}
\usepackage{amsmath}
\usepackage{amsfonts}
\usepackage{amsthm}

% used for TeXing text within eps files
%\usepackage{psfrag}
% need this for including graphics (\includegraphics)
%\usepackage{graphicx}
% for neatly defining theorems and propositions
%\usepackage{amsthm}
% making logically defined graphics
\usepackage[all]{xypic}

% there are many more packages, add them here as you need them

% define commands here

\newcommand{\p}{{\mathfrak{p}}}
\newcommand{\C}{\mathbb{C}}
\newcommand{\R}{\mathbb{R}}
\renewcommand{\H}{\mathcal{H}}
\newcommand{\A}{\mathcal{A}}
\renewcommand{\c}{\mathcal{C}}
\renewcommand{\O}{\mathcal{O}}
\newcommand{\D}{\mathcal{D}}
\newcommand{\lra}{\longrightarrow}
\newcommand{\res}{\operatorname{res}}
\newcommand{\id}{\operatorname{id}}
\newcommand{\diff}{\operatorname{diff}}
\newcommand{\incl}{\operatorname{incl}}
\newcommand{\Hom}{\operatorname{Hom}}
\begin{document}
\newtheorem{theorem}{Theorem}
\newtheorem{proposition}[theorem]{Proposition}
\newtheorem{lemma}[theorem]{Lemma}
\newtheorem{corollary}[theorem]{Corollary}

\theoremstyle{definition}
\newtheorem{definition}[theorem]{Definition}
\newtheorem{example}[theorem]{Example}

\section{Presheaves}

Let $X$ be a topological space and let $\A$ be a category. A
\emph{presheaf} on $X$ with values in $\A$ is a contravariant functor
$F$ from the category $\mathcal{B}$ whose objects are open sets in $X$ and whose morphisms are inclusion mappings of open sets of $X$, to
the category $\A$.

As this definition may be less than helpful to many readers, we offer
the following equivalent (but longer) definition. A \emph{presheaf} $F$
on $X$ consists of the following data:
\begin{enumerate}
\item An object $F(U)$ in $\A$, for each open set $U \subset X$
\item A morphism $\res_{V,U}\colon F(V) \to F(U)$ for each pair of open
sets $U \subset V$ in $X$ (called the \emph{restriction morphism}), such
that:
\begin{enumerate}
\item For every open set $U \subset X$, the morphism $\res_{U,U}$ is
the identity morphism.
\item For any open sets $U \subset V \subset W$ in $X$, the diagram
$$
\xymatrix{
F(W) \ar@/^1pc/[rr]^{\res_{W,U}} \ar[r]_{\res_{W,V}} & F(V)
\ar[r]_{\res_{V,U}} & F(U)
}
$$
commutes.
\end{enumerate}
\end{enumerate}
If the object $F(U)$ of $\A$ is a set, its elements are called \emph{sections} of $U$.

\section{Morphisms of Presheaves}

Let $f\colon X \to Y$ be a continuous map of topological spaces. Suppose
$F_X$ is a presheaf on $X$, and $G_Y$ is a presheaf on $Y$ (with $F_X$
and $G_Y$ both having values in $\A$). We define a \emph{morphism of
presheaves} $\phi$ from $G_Y$ to $F_X$, relative to $f$, to be a
collection of morphisms $\phi_U\colon G_Y(U) \to F_X(f^{-1}(U))$ in $\A$,
one for every open set $U \subset Y$, such that the diagram
$$
\xymatrix{
G_Y(V) \ar[r]^-{\phi_V} \ar[d]_{\res_{V,U}} & F_X(f^{-1}(V))
\ar[d]^{\res_{f^{-1}(V),f^{-1}(U)}} \\
G_Y(U) \ar[r]_-{\phi_U} & F_X(f^{-1}(U))
}
$$
commutes, for each pair of open sets $U \subset V$ in $Y$.

Alternatively, a morphism of presheaves can be regarded as a natural
transformation from $G_Y$ to $F_Y$, where $F_Y$ is the presheaf on $Y$
given by $F_Y(U) := F_X(f^{-1}(U))$.  In the special case that $f$ is the
identity map $\id\colon X \to X$, we omit mention of the map $f$, and speak of
$\phi$ as simply a morphism of presheaves on $X$.

Form the category whose objects are presheaves
on $X$ and whose morphisms are morphisms of presheaves on $X$. Then an
\emph{isomorphism} of presheaves $\phi$ on $X$ is a morphism of
presheaves on $X$ which is an isomorphism in this category; that is,
there exists a morphism $\phi^{-1}$ whose composition with $\phi$ both
ways is the identity morphism.

More generally, if $f\colon X \to Y$ is any homeomorphism of topological
spaces, a morphism of presheaves $\phi$ relative to $f$ is an \emph{isomorphism} if it admits a two--sided inverse morphism of presheaves
$\phi^{-1}$ relative to $f^{-1}$.

\section{Sheaves}

We now assume that the category $\A$ is a concrete category. A \emph{sheaf} is a presheaf $F$ on $X$, with values in $\A$, such that for
every open set $U \subset X$, and every open cover $\{U_i\}$ of
$U$, the following two conditions hold:
\begin{enumerate}
\item Any two elements $f_1, f_2 \in F(U)$ which have identical
restrictions to each $U_i$ are equal. That is, if $\res_{U,U_i} f_1 =
\res_{U,U_i} f_2$ for every $i$, then $f_1 = f_2$.
\item Any collection of elements $f_i \in F(U_i)$ that have common
restrictions can be realized as the collective restrictions of a
single element of $F(U)$. That is, if $\res_{U_i, U_i \cap U_j} f_i =
\res_{U_j, U_i \cap U_j} f_j$ for every $i$ and $j$, then there exists
an element $f \in F(U)$ such that $\res_{U,U_i} f = f_i$ for all $i$.
\end{enumerate}

\section{Sheaves in abelian categories}

If $\A$ is a concrete abelian category, then a presheaf $F$ is a sheaf
if and only if for every open subset $U$ of $X$, the sequence
\begin{equation}\label{exact}
\xymatrix{
0 \ar[r] & F(U) \ar[r]^-{\incl} & \prod_i F(U_i) \ar[r]^-{\diff} &
\prod_{i,j} F(U_i \cap U_j)
}
\end{equation}
is an exact sequence of morphisms in $\A$ for every open cover
$\{U_i\}$ of $U$ in $X$. This diagram requires some explanation,
because we owe the reader a definition of the morphisms $\incl$ and
$\diff$. We start with $\incl$ (short for ``inclusion''). The
restriction morphisms $F(U) \to F(U_i)$ induce a morphism
$$
F(U) \to \prod_i F(U_i)
$$
to the categorical direct product $\prod_i F(U_i)$, which we define
to be $\incl$. The map $\diff$ (called ``difference'') is defined as
follows. For each $U_i$, form the morphism
$$
\alpha_i\colon F(U_i) \to \prod_j F(U_i \cap U_j).
$$
By the universal properties of categorical direct product, there
exists a unique morphism
$$
\alpha\colon \prod_i F(U_i) \to \prod_i \prod_j F(U_i \cap U_j)
$$
such that $\pi_i \alpha = \alpha_i \pi_i$ for all $i$, where $\pi_i$
is projection onto the $i^\text{th}$ factor. In a similar manner, form
the morphism
$$
\beta\colon \prod_j F(U_j) \to \prod_j \prod_i F(U_i \cap U_j).
$$
Then $\alpha$ and $\beta$ are both elements of the set
$$
\Hom\left(\prod_i F(U_i), \prod_{i,j} F(U_i \cap U_j)\right),
$$
which is an abelian group since
$\A$ is an abelian category. Take the difference $\alpha - \beta$ in
this group, and define this morphism to be $\diff$.

Note that exactness of the sequence~\eqref{exact} is an element free
condition, and therefore makes sense for any abelian category $\A$,
even if $\A$ is not concrete. Accordingly, for any abelian category
$\A$, we define a sheaf to be a presheaf $F$ for which the
sequence~\eqref{exact} is always exact.

\section{Examples}

It's high time that we give some examples of sheaves and
presheaves. We begin with some of the standard ones.

\begin{example}
If $F$ is a presheaf on $X$, and $U \subset X$ is an open subset, then
one can define a presheaf $F|_U$ on $U$ by restricting the functor $F$
to the subcategory of open sets of $X$ in $U$ and inclusion
morphisms. In other words, for open subsets of $U$, define $F|_U$ to
be exactly what $F$ was, and ignore open subsets of $X$ that are not
open subsets of $U$. The resulting presheaf is called, for obvious
reasons, the \emph{restriction presheaf} of $F$ to $U$, or the \emph{restriction sheaf} if $F$ was a sheaf to begin with.
\end{example}

\begin{example}
For any topological space $X$, let $\c_X$ be the presheaf on $X$, with
values in the category of rings, given by
\begin{itemize}
\item $\c_X(U) := $ the ring of continuous real--valued functions $U
\to \R$,
\item $\res_{V,U} f := $ the restriction of $f$ to $U$, for every
element $f\colon V \to \R$ of $\c_X(V)$ and every subset $U$ of $V$.
\end{itemize}
Then $\c_X$ is actually a sheaf of rings, because continuous functions
are uniquely specified by their values on an open cover. The sheaf
$\c_X$ is called the \emph{sheaf of continuous real--valued functions}
on $X$.
\end{example}

\begin{example}
Let $X$ be a smooth differentiable manifold. Let $\D_X$ be the
presheaf on $X$, with values in the category of real vector spaces,
defined by setting $\D_X(U)$ to be the space of smooth real--valued
functions on $U$, for each open set $U$, and with the restriction
morphism given by restriction of functions as before. Then $\D_X$ is a
sheaf as well, called the \emph{sheaf of smooth real--valued functions}
on $X$.

Much more surprising is that the construct $\D_X$ can actually be used
to {\bf define} the concept of smooth manifold! That is, one can
define a smooth manifold to be a locally Euclidean $n$--dimensional
second countable topological space $X$, together with a sheaf $F$,
such that there exists an open cover $\{U_i\}$ of $X$ where:
\begin{quotation}
For every $i$, there exists a homeomorphism $f_i\colon U_i \to \R^n$ and
an isomorphism of sheaves $\phi_i\colon \D_{\R^n} \to F|_{U_i}$ relative
to $f_i$.
\end{quotation}
The idea here is that not only does every smooth manifold $X$ have a
sheaf $\D_X$ of smooth functions, but specifying this sheaf of smooth
functions is sufficient to fully describe the smooth manifold
structure on $X$. While this phenomenon may seem little more than a
toy curiousity for differential geometry, it arises in full force in
the field of algebraic geometry where the coordinate functions are
often unwieldy and algebraic structures in many cases can only be
satisfactorily described by way of sheaves and schemes.
\end{example}

\begin{example}
Similarly, for a complex analytic manifold $X$, one can form the sheaf
$\H_X$ of holomorphic functions by setting $\H_X(U)$ equal to the
complex vector space of $\C$--valued holomorphic functions on $U$,
with the restriction morphism being restriction of functions as
before.
\end{example}

\begin{example}
The algebraic geometry analogue of the sheaf $\D_X$ of differential
geometry is the prime spectrum $\operatorname{Spec}(R)$ of a
commutative ring $R$. However, the construction of the sheaf
$\operatorname{Spec}(R)$ is beyond the scope of this discussion and
merits a separate article.
\end{example}

\begin{example}
For an example of a presheaf that is not a sheaf, consider the
presheaf $F$ on $X$, with values in the category of real vector
spaces, whose sections on $U$ are locally constant real--valued
functions on $U$ modulo constant functions on $U$. Then every section
$f \in F(U)$ is locally zero in some fine enough open cover $\{U_i\}$
(it is enough to take a cover where each $U_i$ is connected), whereas
$f$ may be nonzero if $U$ is not connected.
\end{example}

We conclude with some interesting examples of morphisms of
sheaves, chosen to illustrate the unifying power of the language of
schemes across various diverse branches of mathematics.

\begin{enumerate}
\item For any continuous function $f\colon X \to Y$, the map $\phi_U\colon
\c_Y(U) \to \c_X(f^{-1}(U))$ given by $\phi_U(g) := gf$ defines a
morphisms of sheaves from $\c_Y$ to $\c_X$ with respect to $f$.
\item For any continuous function $f\colon X \to Y$ of smooth
differentiable manifolds, the map given by $\phi_U(g) := gf$ has the
property
$$
g \in \D_Y(U) \implies \phi_U(g) \in \D_X(f^{-1}(U))
$$
if and only if $f$ is a smooth function.
\item For any continuous function $f\colon X \to Y$ of complex analytic
manifolds, the map given by $\phi_U(g) := gf$ has the property
$$
g \in \H_Y(U) \implies \phi_U(g) \in \H_X(f^{-1}(U))
$$
if and only if $f$ is a holomorphic function.
\item For any Zariski continuous function $f\colon X \to Y$ of algebraic
varieties over a field $k$, the map given by $\phi_U(g) := gf$ has the
property
$$
g \in \O_Y(U) \implies \phi_U(g) \in \O_X(f^{-1}(U))
$$
if and only if $f$ is a regular function. Here $\O_X$ denotes the
sheaf of $k$--valued regular functions on the algebraic variety $X$.
\end{enumerate}

\begin{thebibliography}{9}
\bibitem{mumford} David Mumford, \emph{The Red Book of Varieties and
Schemes, Second Expanded Edition}, Springer--Verlag, 1999 (LNM {\bf
1358}).
\bibitem{weibel} Charles Weibel, \emph{An Introduction to Homological
Algebra}, Cambridge University Press, 1994.
\end{thebibliography}

%%%%%
%%%%%
\end{document}
