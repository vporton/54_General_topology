\documentclass[12pt]{article}
\usepackage{pmmeta}
\pmcanonicalname{FullyT4}
\pmcreated{2013-03-22 17:09:43}
\pmmodified{2013-03-22 17:09:43}
\pmowner{Mathprof}{13753}
\pmmodifier{Mathprof}{13753}
\pmtitle{fully T4}
\pmrecord{7}{39473}
\pmprivacy{1}
\pmauthor{Mathprof}{13753}
\pmtype{Definition}
\pmcomment{trigger rebuild}
\pmclassification{msc}{54D15}
\pmdefines{fully normal}

% this is the default PlanetMath preamble.  as your knowledge
% of TeX increases, you will probably want to edit this, but
% it should be fine as is for beginners.

% almost certainly you want these
\usepackage{amssymb}
\usepackage{amsmath}
\usepackage{amsfonts}

% used for TeXing text within eps files
%\usepackage{psfrag}
% need this for including graphics (\includegraphics)
%\usepackage{graphicx}
% for neatly defining theorems and propositions
%\usepackage{amsthm}
% making logically defined graphics
%%%\usepackage{xypic}

% there are many more packages, add them here as you need them

% define commands here

\begin{document}
A topological space $X$ is said to be \emph{fully} $T_4$ if every open cover of $X$ 
has star refinement.

A topological space is said to be \emph{fully normal} if it is a $T_1$ space and
is fully $T_4$.

For example, every pseudometric space is fully $T_4$.

We have the following implications:

Lindel\"of $T_3 \Rightarrow$ paracompact and $T_3 \Rightarrow$ fully $T_4 \Rightarrow T_4 \Rightarrow$ uniformizable $\Rightarrow T_3$ ,

and

fully normal $\Leftrightarrow$ paracompact regular.

%%%%%
%%%%%
\end{document}
