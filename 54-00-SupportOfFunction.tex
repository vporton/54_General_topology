\documentclass[12pt]{article}
\usepackage{pmmeta}
\pmcanonicalname{SupportOfFunction}
\pmcreated{2013-03-22 13:46:10}
\pmmodified{2013-03-22 13:46:10}
\pmowner{matte}{1858}
\pmmodifier{matte}{1858}
\pmtitle{support of function}
\pmrecord{16}{34475}
\pmprivacy{1}
\pmauthor{matte}{1858}
\pmtype{Definition}
\pmcomment{trigger rebuild}
\pmclassification{msc}{54-00}
\pmsynonym{support}{SupportOfFunction}
\pmsynonym{carrier}{SupportOfFunction}
\pmrelated{ZeroOfAFunction}
\pmrelated{ApplicationsOfUrysohnsLemmaToLocallyCompactHausdorffSpaces}

\endmetadata

% this is the default PlanetMath preamble.  as your knowledge
% of TeX increases, you will probably want to edit this, but
% it should be fine as is for beginners.

% almost certainly you want these
\usepackage{amssymb}
\usepackage{amsmath}
\usepackage{amsfonts}

% used for TeXing text within eps files
%\usepackage{psfrag}
% need this for including graphics (\includegraphics)
%\usepackage{graphicx}
% for neatly defining theorems and propositions
%\usepackage{amsthm}
% making logically defined graphics
%%%\usepackage{xypic}

% there are many more packages, add them here as you need them

% define commands here

\newcommand{\sR}[0]{\mathbb{R}}
\newcommand{\sC}[0]{\mathbb{C}}
\newcommand{\sN}[0]{\mathbb{N}}
\newcommand{\sZ}[0]{\mathbb{Z}}
\begin{document}
\PMlinkescapeword{words}
\PMlinkescapeword{support}
\newcommand{\supp}[0]{\operatorname{supp}}

{\bf Definition}
Suppose $X$ is a topological space, and $f\colon X\to \sC$ is a function.
Then the \emph{support} of $f$ (written as $\supp f$), is the set
$$ 
  \supp f = \overline{\{x\in X\mid f(x)\neq 0\}}.
$$
In other words, $\supp f$ is the closure of the set where $f$
does not vanish.

\subsubsection*{Properties}
Let $f\colon X\to \sC$ be a function. 
\begin{enumerate}
\item $\supp f$ is closed.
\item If $x\notin \supp f$, then $f(x)=0$. 
\item If $\supp f = \emptyset$, then $f=0$. 
\item If $\chi\colon X\to \sC$ is such that $\chi = 1$ on $\supp f$, then
$f=\chi f$. 
\item If $f,g\colon X\to \sC$ are functions, then we have
\begin{eqnarray*}
\supp (fg) &\subset & \supp f \cap \supp g, \\
\supp (f+g) &\subset & \supp f \cup \supp g.
\end{eqnarray*}
\item If $Y$ is another topological space, and $\Psi\colon Y\to X$ is a 
homeomorphism, then 
$$ 
  \supp (f\circ \Psi) = \Psi^{-1}(\supp f).
$$
\end{enumerate}
%%%%%
%%%%%
\end{document}
