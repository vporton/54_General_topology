\documentclass[12pt]{article}
\usepackage{pmmeta}
\pmcanonicalname{TopicEntryOnRealNumbers}
\pmcreated{2013-03-22 15:40:38}
\pmmodified{2013-03-22 15:40:38}
\pmowner{matte}{1858}
\pmmodifier{matte}{1858}
\pmtitle{topic entry on real numbers}
\pmrecord{22}{37618}
\pmprivacy{1}
\pmauthor{matte}{1858}
\pmtype{Topic}
\pmcomment{trigger rebuild}
\pmclassification{msc}{54C30}
\pmclassification{msc}{26-00}
\pmclassification{msc}{12D99}
\pmrelated{GelfandTornheimTheorem}
\pmrelated{PositivityInOrderedRing}
\pmrelated{TopicEntryOnRationalNumbers}

\endmetadata

% this is the default PlanetMath preamble.  as your knowledge
% of TeX increases, you will probably want to edit this, but
% it should be fine as is for beginners.

% almost certainly you want these
\usepackage{amssymb}
\usepackage{amsmath}
\usepackage{amsfonts}
\usepackage{amsthm}

\usepackage{mathrsfs}

% used for TeXing text within eps files
%\usepackage{psfrag}
% need this for including graphics (\includegraphics)
%\usepackage{graphicx}
% for neatly defining theorems and propositions
%
% making logically defined graphics
%%%\usepackage{xypic}

% there are many more packages, add them here as you need them

% define commands here

\newcommand{\sR}[0]{\mathbb{R}}
\newcommand{\sC}[0]{\mathbb{C}}
\newcommand{\sN}[0]{\mathbb{N}}
\newcommand{\sZ}[0]{\mathbb{Z}}

 \usepackage{bbm}
 \newcommand{\Z}{\mathbbmss{Z}}
 \newcommand{\C}{\mathbbmss{C}}
 \newcommand{\F}{\mathbbmss{F}}
 \newcommand{\R}{\mathbbmss{R}}
 \newcommand{\Q}{\mathbbmss{Q}}



\newcommand*{\norm}[1]{\lVert #1 \rVert}
\newcommand*{\abs}[1]{| #1 |}



\newtheorem{thm}{Theorem}
\newtheorem{defn}{Definition}
\newtheorem{prop}{Proposition}
\newtheorem{lemma}{Lemma}
\newtheorem{cor}{Corollary}
\begin{document}
\subsubsection*{Introduction}
The real number system may be conceived as an attempt to fill in the
gaps in the rational number system.  These gaps first became apparent
in connection with the Pythagorean theorem, which requires one to extract
a square root in order to find the third side of a right triangle two of
whose sides are known.  Hypossos, a student of Pythagoras, showed that there
is no rational number whose square is exactly 2.  In particular, this means
that there is no rational number which may be used to describe the length
of the diagonal of a square the length of whose sides is rational.  This
result ruined the philosophical program of Pythagoras, which was to describe
everything in terms of whole numbers (or ratios of whole numbers) and, 
according to legend, resulted in Hypossos drowning himself.  Eventually,
geometers reconciled themselves to the existence of irrational magnitudes
and Eudoxos devised his method of exhaustion which allowed one to prove
results about irrational magnitudes by considerations of rational magnitudes
which are smaller and larger than the the irrational magnitude in question.

Centuries later, Descartes showed how it is systematically possible to
reduce questions of geometry to algebra.  This brought up the issue of
irrational numbers again --- if one is going to reformulate everything
in terms of algebra, then one cannot have recourse to defining magnitudes
geometrically, but have to find some sort of number which can adequately 
represent things like the hypotenuse of a square with rational sides.
At first, such problems of logical consistency were swept under the rug,
but eventually mathematicians realized that their subject needed to be
put on a firm logical foundation.  In particular, Dedekind solved this 
difficulty by defining the real numbers as a certain type of partition of
the set of rational numbers which he termed a cut and defining operations
on these numbers, such as addition, subtraction, multiplication, and division
in terms of operations on these partitions.

\subsubsection*{Index of entries on real numbers}
The below list presents entries on real numbers in an order 
suitable for studying the subject. 

\begin{enumerate}
\item Rational numbers
\item Axiomatic definition of the real numbers. 
\item Constructions of real numbers (advanced):
   \begin{enumerate}
   \item Dedekind cuts
   \item \PMlinkname{Cauchy sequences}{RealNumber}
   \item \PMlinkname{Characterization of real numbers}{EveryOrderedFieldWithTheLeastUpperBoundPropertyIsIsomorphicToTheRealNumbers}
   \item \PMlinkname{Reals not isomorphic to $p$-adic numbers}{NonIsomorphicCompletionsOfMathbbQ}
   \end{enumerate}
\item commensurable numbers
\item positive
\item \PMlinkname{Inequalities for real numbers}{InequalityForRealNumbers}
\item index of inequalities
\item rational numbers are real numbers 
\item interval
\item nested interval theorem
\item \PMlinkname{Real numbers are uncountable}{CantorsDiagonalArgument}

\item Archimedean property
\item Operations for real numbers
   \begin{enumerate}
   \item infimum and supremum for real numbers
   \item minimal and maximal number
   \item absolute value
   \item square root
   \item fraction power
   \end{enumerate}
\item \PMlinkname{Topic entry on algebraic and transcendental numbers}{TheoryOfAlgebraicNumbers}
   \begin{enumerate}
   \item \PMlinkname{Irrational number}{Irrational}
   \item Transcendental number
   \item \PMlinkname{Algebraic number}{AlgebraicNumber}
   \end{enumerate}
\item Particular real numbers
 \begin{enumerate}
 \item natural log base
 \item pi
 \item Mascheroni constant
 \item golden ratio
 \end{enumerate}

\end{enumerate}

\subsubsection*{Generalizations}
There are many generalizations of real numbers. These include
the complex numbers, quaternions, extended real numbers, 
\PMlinkname{hyperreal numbers}{Hyperreal},
and surreal numbers.\, Of course the field $\mathbb{R}$ has many other field extensions, e.g. the field $\mathbb{R}(x)$ of the rational functions.
%%%%%
%%%%%
\end{document}
