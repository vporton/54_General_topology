\documentclass[12pt]{article}
\usepackage{pmmeta}
\pmcanonicalname{KrullDimension}
\pmcreated{2013-03-22 12:03:27}
\pmmodified{2013-03-22 12:03:27}
\pmowner{mathcam}{2727}
\pmmodifier{mathcam}{2727}
\pmtitle{Krull dimension}
\pmrecord{8}{31107}
\pmprivacy{1}
\pmauthor{mathcam}{2727}
\pmtype{Definition}
\pmcomment{trigger rebuild}
\pmclassification{msc}{54-00}
\pmsynonym{dimension (Krull)}{KrullDimension}
\pmrelated{HeightOfAPrimeIdeal}
\pmrelated{Dimension3}

\endmetadata

\usepackage{amssymb}
\usepackage{amsmath}
\usepackage{amsfonts}
\usepackage{graphicx}
%%%\usepackage{xypic}
\begin{document}
If $R$ is a ring, the \emph{Krull dimension} (or simply dimension) of $R$, $\dim R$ is the supremum of all integers $n$ such that there is an increasing sequence of prime ideals $\mathfrak{p}_0 \subsetneq \cdots \subsetneq \mathfrak{p}_n$ of length $n$ in $R$.

If $X$ is a topological space, the Krull dimension (or simply dimension) of $X$, $\dim X$ is the supremum of all integers $n$ such that there is a decreasing sequence of irreducible closed subsets $F_0 \supsetneq \cdots \supsetneq F_n$ of $X$.
%%%%%
%%%%%
%%%%%
\end{document}
