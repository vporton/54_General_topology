\documentclass[12pt]{article}
\usepackage{pmmeta}
\pmcanonicalname{AlexandrovOnepointCompactification}
\pmcreated{2013-03-22 13:47:54}
\pmmodified{2013-03-22 13:47:54}
\pmowner{yark}{2760}
\pmmodifier{yark}{2760}
\pmtitle{Alexandrov one-point compactification}
\pmrecord{9}{34515}
\pmprivacy{1}
\pmauthor{yark}{2760}
\pmtype{Definition}
\pmcomment{trigger rebuild}
\pmclassification{msc}{54D35}
\pmsynonym{one-point compactification}{AlexandrovOnepointCompactification}
\pmsynonym{Alexandroff one-point compactification}{AlexandrovOnepointCompactification}
\pmsynonym{Aleksandrov one-point compactification}{AlexandrovOnepointCompactification}
\pmsynonym{Alexandrov compactification}{AlexandrovOnepointCompactification}
\pmsynonym{Aleksandrov compactification}{AlexandrovOnepointCompactification}
\pmsynonym{Alexandroff compactification}{AlexandrovOnepointCompactification}
%\pmkeywords{compactification}
\pmrelated{Compactification}

\endmetadata


\begin{document}
\PMlinkescapeword{open}

The \emph{Alexandrov one-point compactification} of a non-compact topological space $X$ is obtained by adjoining a new point $\infty$ and defining the topology on $X\cup\{\infty\}$ to consist of the open sets of $X$ together with the sets of the form $U\cup\{\infty\}$, where $U$ is an open subset of $X$ with compact complement.

With this topology, $X\cup\{\infty\}$ is always compact.
Furthermore, it is Hausdorff if and only if $X$ is Hausdorff and locally compact.
%%%%%
%%%%%
\end{document}
