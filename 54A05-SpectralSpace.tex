\documentclass[12pt]{article}
\usepackage{pmmeta}
\pmcanonicalname{SpectralSpace}
\pmcreated{2013-03-22 16:22:32}
\pmmodified{2013-03-22 16:22:32}
\pmowner{jocaps}{12118}
\pmmodifier{jocaps}{12118}
\pmtitle{spectral space}
\pmrecord{9}{38518}
\pmprivacy{1}
\pmauthor{jocaps}{12118}
\pmtype{Definition}
\pmcomment{trigger rebuild}
\pmclassification{msc}{54A05}
%\pmkeywords{generic points}
%\pmkeywords{generization}
%\pmkeywords{specialization}
%\pmkeywords{closed points}

% this is the default PlanetMath preamble.  as your knowledge
% of TeX increases, you will probably want to edit this, but
% it should be fine as is for beginners.

% almost certainly you want these
\usepackage{amssymb}
\usepackage{amsmath}
\usepackage{amsfonts}

% used for TeXing text within eps files
%\usepackage{psfrag}
% need this for including graphics (\includegraphics)
%\usepackage{graphicx}
% for neatly defining theorems and propositions
%\usepackage{amsthm}
% making logically defined graphics
%%%\usepackage{xypic}

% there are many more packages, add them here as you need them

% define commands here

\begin{document}
A topological space is called \emph{spectral} if 
\begin{itemize}
\item it is compact, 
\item Kolmogorov (also called \PMlinkname{$T_0$}{T0}), 
\item compactness is preserved upon finite intersection of open compact sets, and 
\item any nonempty irreducible subspace of it contains a generic point
\end{itemize}

In his thesis, Mel Hochster showed that for any spectral space there is commutative unitary ring whose prime spectrum is homeomorphic to the spectral space.

\begin{thebibliography}{99}

\bibitem{hochster}
\textbf{M. Hochster},
"Prime Ideal Structure in Commutative Rings", 
Transactions of American Mathematical Society, Aug. 1969, vol. 142, p. 43-60

\end{thebibliography}
%%%%%
%%%%%
\end{document}
