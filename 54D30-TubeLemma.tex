\documentclass[12pt]{article}
\usepackage{pmmeta}
\pmcanonicalname{TubeLemma}
\pmcreated{2013-03-22 17:25:39}
\pmmodified{2013-03-22 17:25:39}
\pmowner{asteroid}{17536}
\pmmodifier{asteroid}{17536}
\pmtitle{tube lemma}
\pmrecord{7}{39802}
\pmprivacy{1}
\pmauthor{asteroid}{17536}
\pmtype{Theorem}
\pmcomment{trigger rebuild}
\pmclassification{msc}{54D30}

% this is the default PlanetMath preamble.  as your knowledge
% of TeX increases, you will probably want to edit this, but
% it should be fine as is for beginners.

% almost certainly you want these
\usepackage{amssymb}
\usepackage{amsmath}
\usepackage{amsfonts}

% used for TeXing text within eps files
%\usepackage{psfrag}
% need this for including graphics (\includegraphics)
%\usepackage{graphicx}
% for neatly defining theorems and propositions
%\usepackage{amsthm}
% making logically defined graphics
%%%\usepackage{xypic}

% there are many more packages, add them here as you need them

% define commands here

\begin{document}
{\bf Tube lemma -} Let $X$ and $Y$ be topological spaces such that $Y$ is compact. If $N$ is an open set of $X \times Y$ containing a "slice" $x_0 \times Y$, then $N$ contains some "tube" $W \times Y$, where $W$ is a neighborhood of $x_0$ in $X$.

{\bf Proof :} $N$ is a union of basis elements $U \times V$, with $U$ and $V$ open sets in $X$ and $Y$ respect. Since $x_0 \times Y$ is compact (it is homeomorphic to $Y$), only a finite number $U_1 \times V_1, \dots, U_n \times V_n$ of such basis elements cover $x_0 \times Y$.

We may assume that each of the basis elements $U_i \times V_i$ actually intersects $x_0 \times Y$, since otherwise we could discard it from the finite collection and still have a covering of $x_0 \times Y$.

Define $W:=U_1 \cap \dots \cap U_n$. The set $W$ is open and contains $x_0$ because each $U_i \times V_i$ intersects $x_0 \times Y$ by the previous remark.

We now claim that $W \times Y \subseteq N$. Let $(x, y)$ be a point in $W \times Y$. The point $(x_0, y)$ is in some $U_i \times V_i$ and so $y \in V_i$. We also know that $x \in W = U_1 \cap \dots \cap U_n \subseteq U_i$.

Therefore $(x, y) \in U_i \times V_i \subseteq N$ as desired. $\square$

\begin{thebibliography}{9}
\bibitem{munkres} J. Munkres, \emph{Topology} (2nd edition), Prentice Hall, 1999.
\end{thebibliography}
%%%%%
%%%%%
\end{document}
