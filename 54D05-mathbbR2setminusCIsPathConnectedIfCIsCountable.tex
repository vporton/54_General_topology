\documentclass[12pt]{article}
\usepackage{pmmeta}
\pmcanonicalname{mathbbR2setminusCIsPathConnectedIfCIsCountable}
\pmcreated{2013-03-22 16:09:11}
\pmmodified{2013-03-22 16:09:11}
\pmowner{silverfish}{6603}
\pmmodifier{silverfish}{6603}
\pmtitle{$\mathbb{R}^2 \setminus C$ is path connected if $C$ is countable}
\pmrecord{5}{38234}
\pmprivacy{1}
\pmauthor{silverfish}{6603}
\pmtype{Theorem}
\pmcomment{trigger rebuild}
\pmclassification{msc}{54D05}

\endmetadata

% this is the default PlanetMath preamble.  as your knowledge
% of TeX increases, you will probably want to edit this, but
% it should be fine as is for beginners.

% almost certainly you want these
\usepackage{amssymb}
\usepackage{amsmath}
\usepackage{amsfonts}

% used for TeXing text within eps files
%\usepackage{psfrag}
% need this for including graphics (\includegraphics)
%\usepackage{graphicx}
% for neatly defining theorems and propositions
\usepackage{amsthm}
% making logically defined graphics
%%%\usepackage{xypic}

\newtheorem{theo}{Theorem}
\newtheorem{coro}{Corollary}

% there are many more packages, add them here as you need them

% define commands here

\begin{document}
\begin{theo} Let $C$ be a countable subset of $\mathbb{R}^2$.  Then $\mathbb{R}^2\setminus C$ is path connected.\end{theo}

We use $\mathbb{R}^2$ simply as an example; an analogous proof will work for any $\mathbb{R}^n, n>1$.

\begin{proof}
Fix a point $P$ not in $C$.  The strategy of the proof is to construct a path $p_x$ from any $x \in \mathbb{R}^2\setminus C$ to $P$. If we can do this then for any $d, d' \in \mathbb{R}^2\setminus C$ we may construct a path from $d$ to $d'$ by first following $p_d$ and then following $p_{d'}$ in reverse.

Fix $x \in \mathbb{R}^2\setminus C$, and consider the set of all (straight) lines through $x$.  There are uncountably many of these and they meet in the single point $x$, so not all of them contain a point of $C$.  Choose one that doesn't and move along it:  your distance from $P$ takes on uncountably many values, and hence at some point this distance $r$ from $P$ is not shared by any point of $C$.  The whole of the circle with radius $r$, centre $P$, lies in $\mathbb{R}^2\setminus C$ so we may move around it freely.

Consider all lines through $P$:  these all intersect this circle, and there are uncountably many of them so we may choose one, say $L$, that contains no point of $C$.  Moving around the circle until we meet $L$ and then following it inwards completes our path form $x$ to $P$.
\end{proof} 

\begin{coro} Let $f: \mathbb{R}^2 \rightarrow \mathbb{R}$ be continuous and onto.  Then $f^{-1} (0)$ is uncountable.  \end{coro}

\begin{proof} Suppose that $f^{-1} (0)$ is countable.  $\mathbb{R}^2$ can be written as the disjoint union
\[ f^{-1} (0) \cup f^{-1} ((-\infty, 0)) \cup f^{-1} ((0, \infty)) \]
where the last two sets are open (as $f$ is continuous), non-empty (as $f$ is onto) and disjoint.  Since pathwise connected is the same as connected for Hausdorff spaces, we have that $\mathbb{R}^2 \setminus  f^{-1} (0)$ is not path connected, contradicting the theorem.
\end{proof}
%%%%%
%%%%%
\end{document}
