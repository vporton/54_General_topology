\documentclass[12pt]{article}
\usepackage{pmmeta}
\pmcanonicalname{NoncommutativeTopology}
\pmcreated{2013-03-22 17:40:18}
\pmmodified{2013-03-22 17:40:18}
\pmowner{asteroid}{17536}
\pmmodifier{asteroid}{17536}
\pmtitle{noncommutative topology}
\pmrecord{14}{40110}
\pmprivacy{1}
\pmauthor{asteroid}{17536}
\pmtype{Topic}
\pmcomment{trigger rebuild}
\pmclassification{msc}{54A99}
\pmclassification{msc}{46L85}
\pmclassification{msc}{46L05}
\pmrelated{GelfandTransform}
\pmrelated{NoncommutativeGeometry}
\pmdefines{noncommutative topology dictionary}
\pmdefines{Noncommutative Geometry}

\endmetadata

% this is the default PlanetMath preamble.  as your knowledge
% of TeX increases, you will probably want to edit this, but
% it should be fine as is for beginners.

% almost certainly you want these
\usepackage{amssymb}
\usepackage{amsmath}
\usepackage{amsfonts}

% used for TeXing text within eps files
%\usepackage{psfrag}
% need this for including graphics (\includegraphics)
%\usepackage{graphicx}
% for neatly defining theorems and propositions
%\usepackage{amsthm}
% making logically defined graphics
%%%\usepackage{xypic}

% there are many more packages, add them here as you need them

% define commands here

\begin{document}
\PMlinkescapeword{properties}
\PMlinkescapeword{theory}

\section{Noncommutative Topology}
Noncommutative topology is basically the theory of \PMlinkname{$C^*$-algebras}{CAlgebra}. But why the name noncommutative topology then?

It turns out that commutative $C^*$-algebras and 
\PMlinkname{locally compact Hausdorff spaces}{LocallyCompactHausdorffSpace} are one and the same "thing" (this will be explained further ahead). Every commutative $C^*$-algebra corresponds to a locally compact Hausdorff space and vice-versa and there is a correspondence between topological properties of spaces and $C^*$-algebraic properties (see the noncommutative topology dictionary below).

The $C^*$-algebraic properties and concepts are of course present in noncommutative $C^*$-algebras too. Thus, although noncommutative $C^*$-algebras cannot be associated with "standard" topological spaces, all the topological/$C^*$ concepts are present. For this reason, this \PMlinkescapetext{area} of mathematics was given the name "noncommutative topology".

In this \PMlinkescapetext{way}, noncommutative topology can be seen as "topology, but without spaces".

\section{The Commutative Case}
Given a locally compact Hausdorff space $X$, all of its topological properties are encoded in $C_0(X)$, the algebra of complex-valued continuous functions in $X$ that vanish at \PMlinkescapetext{infinity}. Notice that $C_0(X)$ is a commutative $C^*$-algebra.

Conversely, given a commutative $C^*$-algebra $\mathcal{A}$, the Gelfand transform provides an isomorphism between $\mathcal{A}$ and $C_0(X)$, for a suitable locally compact Hausdorff space $X$.

Furthermore, there is an \PMlinkname{equivalence}{EquivalenceOfCategories} between the category of locally compact Hausdorff spaces and the category of commutative $C^*$-algebras. This is the content of the Gelfand-Naimark theorem.

This equivalence of categories is one of the reasons for saying that locally compact Hausdorff spaces and commutative $C^*$-algebras are the same thing. The other reason is the correspondence between topological and $C^*$-algebraic properties, present in the following dictionary.

\section{Noncommutative Topology Dictionary}

We only provide a short list of easy-to-state concepts. Some correspondences of properties are technical and could not be easily stated here. Some of them originate new \PMlinkescapetext{branches} of "noncommutative mathematics", such as noncommutative measure theory.
$\,$

\begin{tabular}{|l|l|}
{\bf Topological properties and concepts} & {\bf $C^*$-algebraic properties and concepts}\\
topological space & $C^*$-algebra\\
proper map & *-homomorphism\\
homeomorphism & *-isomorphism\\
open subset & ideal\\
closed subset & \PMlinkname{quotient}{QuotientRing}\\
compact space & algebra with unit\\
compactification & unitization \\
one-point compactification & \PMlinkname{minimal unitization}{Unitization}\\
\PMlinkname{Stone-Cech compactification}{StoneVCechCompactification} & \PMlinkescapetext{maximal} unitization\\
second countable & separable\\
connected & projectionless\\
connected components and topological sums & projections\\
complement of singleton & maximal ideal\\
Radon measure & \PMlinkescapetext{positive linear functional}\\
\end{tabular}

\subsection{Remarks:}

1. Noncommutative topology can be considered as part of \PMlinkexternal{Nonabelian Algebraic Topology (NAAT)}{http://aux.planetphysics.us/files/books/167/Anatv1.pdf}.

2.A specialized form of noncommutative topology is generally known as 
\PMlinkname{\em Noncommutative Geometry}{NoncommutativeGeometry} and has been introduced and developed by Professor Alain Connes (Field Medialist in 1982 and Crafoord Prize in 2001).

%%%%%
%%%%%
\end{document}
