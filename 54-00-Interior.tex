\documentclass[12pt]{article}
\usepackage{pmmeta}
\pmcanonicalname{Interior}
\pmcreated{2013-03-22 12:48:20}
\pmmodified{2013-03-22 12:48:20}
\pmowner{yark}{2760}
\pmmodifier{yark}{2760}
\pmtitle{interior}
\pmrecord{19}{33123}
\pmprivacy{1}
\pmauthor{yark}{2760}
\pmtype{Definition}
\pmcomment{trigger rebuild}
\pmclassification{msc}{54-00}
%\pmkeywords{topology}
\pmrelated{Complement}
\pmrelated{Closure}
\pmrelated{BoundaryInTopology}
\pmdefines{exterior}

\endmetadata

\usepackage{amssymb}
\usepackage{amsmath}
\usepackage{amsfonts}

\def\int{\operatorname{int}}
\def\emptyset{\varnothing}
\begin{document}
\PMlinkescapeword{entire}
\PMlinkescapeword{implies}
\PMlinkescapeword{mean}
\PMlinkescapeword{properties}

Let $A$ be a subset of a topological space $X$.

The union of all open sets contained in $A$
is defined to be the \emph{interior} of $A$.
Equivalently, one could define the interior
of $A$ to the be the largest open set contained in $A$.

In this entry we denote the interior of $A$ by $\int(A)$.
Another common notation is $A^\circ$.

The \emph{exterior} of $A$ is defined as 
the union of all open sets whose intersection with $A$ is empty.
That is, the exterior of $A$ is the interior of the complement of $A$.

The interior of a set enjoys many special properties,
some of which are listed below:
\begin{enumerate}
\item $\int(A)\subseteq A$
\item $\int(A)$ is open
\item $\int(\int(A))=\int(A)$
\item $\int(X)=X$
\item $\int(\emptyset)=\emptyset$
\item $A$ is open if and only if $A=\int(A)$
\item $\overline{A^\complement}=(\int(A))^\complement$
\item $\overline{A}^\complement = \int(A^\complement)$
\item $A\subseteq B$ implies that $\int(A)\subseteq \int(B)$
\item $\int(A)=A\setminus \partial A$,
      where $\partial A$ is the boundary of $A$
\item $X=\int(A)\cup \partial A \cup \int(A^\complement)$
\end{enumerate}

\begin{thebibliography}{9}
\bibitem{willard} S. Willard, \emph{General Topology},
Addison-Wesley Publishing Company, 1970.
\end{thebibliography}
%%%%%
%%%%%
\end{document}
