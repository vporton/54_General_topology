\documentclass[12pt]{article}
\usepackage{pmmeta}
\pmcanonicalname{UltrametricSpace}
\pmcreated{2013-03-22 14:55:28}
\pmmodified{2013-03-22 14:55:28}
\pmowner{pahio}{2872}
\pmmodifier{pahio}{2872}
\pmtitle{ultrametric space}
\pmrecord{6}{36612}
\pmprivacy{1}
\pmauthor{pahio}{2872}
\pmtype{Definition}
\pmcomment{trigger rebuild}
\pmclassification{msc}{54E35}
\pmrelated{UltrametricTriangleInequality}
\pmrelated{Ultrametric}

\endmetadata

% this is the default PlanetMath preamble.  as your knowledge
% of TeX increases, you will probably want to edit this, but
% it should be fine as is for beginners.

% almost certainly you want these
\usepackage{amssymb}
\usepackage{amsmath}
\usepackage{amsfonts}

% used for TeXing text within eps files
%\usepackage{psfrag}
% need this for including graphics (\includegraphics)
%\usepackage{graphicx}
% for neatly defining theorems and propositions
%\usepackage{amsthm}
% making logically defined graphics
%%%\usepackage{xypic}

% there are many more packages, add them here as you need them

% define commands here
\begin{document}
The metric space \,$(X,\,d)$\, is called an {\em ultrametric space}, if its metric $d$ is an ultrametric, i.e. if
     $$d(x,\,z) \leqq \max \{d(x,\,y),\,d(y,\,z)\} \quad \forall x,\,y,\,z\in X.$$

\textbf{Example.} \,The field $\mathbb{Q}$ together with any of its $p$-adic metrics
                     $$d_p(x,\,y) = |x-y|_p,$$
where \,$|\cdot|_p$\, is the \PMlinkname{$p$-adic valuation}{PAdicValuation} of $\mathbb{Q}$,\, forms an ultrametric space.
%%%%%
%%%%%
\end{document}
