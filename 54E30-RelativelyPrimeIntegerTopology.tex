\documentclass[12pt]{article}
\usepackage{pmmeta}
\pmcanonicalname{RelativelyPrimeIntegerTopology}
\pmcreated{2013-03-22 14:42:07}
\pmmodified{2013-03-22 14:42:07}
\pmowner{mathcam}{2727}
\pmmodifier{mathcam}{2727}
\pmtitle{relatively prime integer topology}
\pmrecord{7}{36314}
\pmprivacy{1}
\pmauthor{mathcam}{2727}
\pmtype{Definition}
\pmcomment{trigger rebuild}
\pmclassification{msc}{54E30}
\pmdefines{prime integer topology}

% this is the default PlanetMath preamble.  as your knowledge
% of TeX increases, you will probably want to edit this, but
% it should be fine as is for beginners.

% almost certainly you want these
\usepackage{amssymb}
\usepackage{amsmath}
\usepackage{amsfonts}
\usepackage{amsthm}

% used for TeXing text within eps files
%\usepackage{psfrag}
% need this for including graphics (\includegraphics)
%\usepackage{graphicx}
% for neatly defining theorems and propositions
%\usepackage{amsthm}
% making logically defined graphics
%%%\usepackage{xypic}

% there are many more packages, add them here as you need them

% define commands here

\newcommand{\mc}{\mathcal}
\newcommand{\mb}{\mathbb}
\newcommand{\mf}{\mathfrak}
\newcommand{\ol}{\overline}
\newcommand{\ra}{\rightarrow}
\newcommand{\la}{\leftarrow}
\newcommand{\La}{\Leftarrow}
\newcommand{\Ra}{\Rightarrow}
\newcommand{\nor}{\vartriangleleft}
\newcommand{\Gal}{\text{Gal}}
\newcommand{\GL}{\text{GL}}
\newcommand{\Z}{\mb{Z}}
\newcommand{\R}{\mb{R}}
\newcommand{\Q}{\mb{Q}}
\newcommand{\C}{\mb{C}}
\newcommand{\<}{\langle}
\renewcommand{\>}{\rangle}
\begin{document}
Let $X$ be the set of strictly positive integers.  The \emph{relatively prime integer topology} on $X$ is the topology determined by a basis consisting of the sets 
\begin{align*}
U(a,b)=\{ax+b\mid x\in X\}
\end{align*}
for any $a$ and $b$ are relatively prime integers.  That this does indeed form a basis is found in \PMlinkname{this entry.}{HausdorffSpaceNotCompletelyHausdorff}

Equipped with this topology, $X$ is \PMlinkname{$T_0$}{T0Space}, \PMlinkname{$T_1$}{T1Space},and \PMlinkname{$T_2$}{T2Space}, but satisfies none of the higher separation axioms (and hence meet very few compactness criteria).

We can define a coarser topology on $X$ by considering the subbasis of the above basis consisting of all $U(a,b)$ with $a$ being a prime.  This is called the \emph{prime integer topology} on $\Z^+$.

\begin{thebibliography}{9}
\bibitem{steen} L.A. Steen, J.A.Seebach, Jr.,
\emph{Counterexamples in topology},
Holt, Rinehart and Winston, Inc., 1970.
\end{thebibliography}
%%%%%
%%%%%
\end{document}
