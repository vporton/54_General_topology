\documentclass[12pt]{article}
\usepackage{pmmeta}
\pmcanonicalname{PointAndACompactSetInAHausdorffSpaceHaveDisjointOpenNeighborhoods}
\pmcreated{2013-03-22 13:34:27}
\pmmodified{2013-03-22 13:34:27}
\pmowner{drini}{3}
\pmmodifier{drini}{3}
\pmtitle{point and a compact set in a Hausdorff space have  disjoint open neighborhoods.}
\pmrecord{13}{34193}
\pmprivacy{1}
\pmauthor{drini}{3}
\pmtype{Theorem}
\pmcomment{trigger rebuild}
\pmclassification{msc}{54D30}
\pmclassification{msc}{54D10}

\endmetadata

% this is the default PlanetMath preamble.  as your knowledge
% of TeX increases, you will probably want to edit this, but
% it should be fine as is for beginners.

% almost certainly you want these
\usepackage{amssymb}
\usepackage{amsmath}
\usepackage{amsfonts}
\usepackage{amsthm}

% used for TeXing text within eps files
%\usepackage{psfrag}
% need this for including graphics (\includegraphics)
%\usepackage{graphicx}
% for neatly defining theorems and propositions
%\usepackage{amsthm}
% making logically defined graphics
%%%\usepackage{xypic}

% there are many more packages, add them here as you need them

% define commands here

\newtheorem*{thm}{Theorem}
\begin{document}
\PMlinkescapeword{satisfy}
\begin{thm} Let $X$ be a Hausdorff space, let $A$ be a compact
non-empty set in $X$, and let $y$ a point in the complement of $A$. 
Then there exist disjoint open sets $U$ and $V$ in $X$ such that 
$A\subset U$ and $y\in V$. 
\end{thm}

\begin{proof} 
First we use the fact that $X$ is a Hausdorff space. 
Thus, for all $x\in A$ there exist
disjoint open sets $U_x$ and $V_x$ such that $x\in U_x$ and $y\in V_x$. 
Then $\{U_x\}_{x\in A}$ is an open cover for $A$. 
Using \PMlinkname{this characterization of compactness}{YIsCompactIfAndOnlyIfEveryOpenCoverOfYHasAFiniteSubcover}, 
it follows that there
exist a finite set $A_0\subset A$ such that $\{U_x\}_{x\in A_0}$
is a finite open cover for $A$. 
Let us define
\begin{eqnarray*}
U= \bigcup_{x\in A_0} U_x,\,\,\,\,\,&\,&\,\,\,\,\, V= \bigcap_{x\in A_0} V_x.
\end{eqnarray*}
Next we show  that these sets satisfy the given conditions 
for $U$ and $V$. First, it is clear
that $U$ and $V$ are open. We also have that $A\subset U$ and $y\in V$. 
To see that $U$ and $V$ are disjoint, suppose $z\in U$. Then
$z\in U_x$ for some $x\in A_0$. 
Since $U_x$ and $V_x$ are disjoint, $z$ can not be in $V_x$, and 
consequently $z$ can not be in $V$.
\end{proof}

The above result and proof follows \cite{kelley} (Chapter 5, Theorem 7) or \cite{singer} (page 27).

\begin{thebibliography}{9}
\bibitem{kelley} J.L. Kelley, \emph{General Topology},
D. van Nostrand Company, Inc., 1955.
\bibitem{singer} I.M. Singer, J.A.Thorpe,
\emph{Lecture Notes on Elementary Topology and Geometry},
Springer-Verlag, 1967.
\end{thebibliography}
%%%%%
%%%%%
\end{document}
