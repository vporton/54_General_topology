\documentclass[12pt]{article}
\usepackage{pmmeta}
\pmcanonicalname{DenseInitself}
\pmcreated{2013-03-22 14:38:29}
\pmmodified{2013-03-22 14:38:29}
\pmowner{rspuzio}{6075}
\pmmodifier{rspuzio}{6075}
\pmtitle{dense in-itself}
\pmrecord{4}{36228}
\pmprivacy{1}
\pmauthor{rspuzio}{6075}
\pmtype{Definition}
\pmcomment{trigger rebuild}
\pmclassification{msc}{54A99}
\pmrelated{ScatteredSpace}

% this is the default PlanetMath preamble.  as your knowledge
% of TeX increases, you will probably want to edit this, but
% it should be fine as is for beginners.

% almost certainly you want these
\usepackage{amssymb}
\usepackage{amsmath}
\usepackage{amsfonts}

% used for TeXing text within eps files
%\usepackage{psfrag}
% need this for including graphics (\includegraphics)
%\usepackage{graphicx}
% for neatly defining theorems and propositions
%\usepackage{amsthm}
% making logically defined graphics
%%%\usepackage{xypic}

% there are many more packages, add them here as you need them

% define commands here
\begin{document}
A subset $A$ of a topological space is said to be \emph{dense-in-itself} if $A$ contains no isolated points.

Note that if the subset $A$ is also a closed set, then $A$ will be a perfect set.  Conversely, every perfect set is dense-in-itself.

A simple example of a set which is dense-in-itself but not closed (and hence not a perfect set) is the subset of irrational numbers.  This set is dense-in-itself because every neighborhood of an irrational number $x$ contains at least one other irrational number $y \ne x$.  On the other hand, this set of irrationals is not closed because every rational number lies in its closure.

For similar reasons, the set of rational numbers is also dense-in-itself but not closed.
%%%%%
%%%%%
\end{document}
