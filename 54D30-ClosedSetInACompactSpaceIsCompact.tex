\documentclass[12pt]{article}
\usepackage{pmmeta}
\pmcanonicalname{ClosedSetInACompactSpaceIsCompact}
\pmcreated{2013-03-22 13:34:02}
\pmmodified{2013-03-22 13:34:02}
\pmowner{mathcam}{2727}
\pmmodifier{mathcam}{2727}
\pmtitle{closed set in a compact space is compact}
\pmrecord{9}{34177}
\pmprivacy{1}
\pmauthor{mathcam}{2727}
\pmtype{Proof}
\pmcomment{trigger rebuild}
\pmclassification{msc}{54D30}
\pmrelated{ClosedSubsetsOfACompactSetAreCompact}

% this is the default PlanetMath preamble.  as your knowledge
% of TeX increases, you will probably want to edit this, but
% it should be fine as is for beginners.

% almost certainly you want these
\usepackage{amssymb}
\usepackage{amsmath}
\usepackage{amsfonts}

% used for TeXing text within eps files
%\usepackage{psfrag}
% need this for including graphics (\includegraphics)
%\usepackage{graphicx}
% for neatly defining theorems and propositions
%\usepackage{amsthm}
% making logically defined graphics
%%%\usepackage{xypic}

% there are many more packages, add them here as you need them

% define commands here
\begin{document}
\emph{Proof.} Let $A$ be a closed set in a compact space $X$. 
To show that $A$ is compact, we show that an arbitrary open cover has
a finite subcover. For this purpose, suppose
$\{U_i\}_{i\in I}$ be an arbitrary open cover for $A$. 
Since $A$ is closed, the complement of $A$, 
which we denote by $A^c$, is open. 
Hence
$A^c$ and $\{U_i\}_{i\in I}$ together form an open cover for $X$. 
Since $X$ is compact, this cover has a finite subcover that
covers $X$. Let $D$ be this subcover. 
Either $A^c$ is part of $D$ or $A^c$ is not. 
In any case, $D\backslash\{A^c\}$ is a finite open cover
for $A$, and $D\backslash\{A^c\}$
is a subcover of $\{U_i\}_{i\in I}$. The claim follows. $\Box$
%%%%%
%%%%%
\end{document}
