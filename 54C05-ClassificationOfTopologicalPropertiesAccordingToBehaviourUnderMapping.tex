\documentclass[12pt]{article}
\usepackage{pmmeta}
\pmcanonicalname{ClassificationOfTopologicalPropertiesAccordingToBehaviourUnderMapping}
\pmcreated{2013-03-22 14:38:04}
\pmmodified{2013-03-22 14:38:04}
\pmowner{rspuzio}{6075}
\pmmodifier{rspuzio}{6075}
\pmtitle{classification of topological properties according to behaviour under mapping}
\pmrecord{15}{36217}
\pmprivacy{1}
\pmauthor{rspuzio}{6075}
\pmtype{Definition}
\pmcomment{trigger rebuild}
\pmclassification{msc}{54C05}
\pmdefines{hereditary}
\pmdefines{hereditarily}
\pmdefines{weakly hereditary}
\pmdefines{continuous}
\pmdefines{open}
\pmdefines{closed invariant}

\endmetadata

% this is the default PlanetMath preamble.  as your knowledge
% of TeX increases, you will probably want to edit this, but
% it should be fine as is for beginners.

% almost certainly you want these
\usepackage{amssymb}
\usepackage{amsmath}
\usepackage{amsfonts}

% used for TeXing text within eps files
%\usepackage{psfrag}
% need this for including graphics (\includegraphics)
%\usepackage{graphicx}
% for neatly defining theorems and propositions
%\usepackage{amsthm}
% making logically defined graphics
%%%\usepackage{xypic}

% there are many more packages, add them here as you need them

% define commands here
\begin{document}
Topological properties may be classified by their behaviour with respect to mappings.  The basis of such a classification is the following question:  Given two topological spaces $X$ and $Y$ and a continuous map $f \colon X \to Y$, can one infer that one of the spaces has a certain topological property from the fact that the other space has this property? 

A trivial case of this question may be disposed of.  If $f$ is a homeomorphism, then the spaces $X$ and $Y$ cannot be distinguished using only the techniques of topology, and hence both spaces will have exactly the same topological properties.

To obtain a non-trivial classification, we must consider more general maps.  Since every map may be expressed as the composition of an inclusion and a surjection, it is natural to consider the cases where $f$ is an inclusion and where it is a surjection.

In the case of an inclusion, we can define the following classifications:

A property of a topological space is called {\bf hereditary} if it is the case that whenever a space has that property, every subspace of that space also has the same property.

A property of a topological space is called {\bf weakly hereditary} if it is the case that whenever a space has that property, every \emph{closed} subspace of that space also has the same property.

In the case of a surjection, we can define the following classifications:

A property of a topological space is called {\bf continuous} if it is the case that, whenever a space has this property, the images of this space under all continuous mapping also have the same property.

A property of a topological space is called {\bf open} if it is the case that, whenever a space has this property, the images of this space under all open continuous mappings also have the same property.

A property of a topological space is called {\bf closed invariant} if it is the case that, whenever a space has this property, the images of this space under all closed continuous mapping also have the same property.
%%%%%
%%%%%
\end{document}
