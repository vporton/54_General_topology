\documentclass[12pt]{article}
\usepackage{pmmeta}
\pmcanonicalname{MetricEquivalence}
\pmcreated{2013-03-22 19:23:11}
\pmmodified{2013-03-22 19:23:11}
\pmowner{CWoo}{3771}
\pmmodifier{CWoo}{3771}
\pmtitle{metric equivalence}
\pmrecord{6}{42341}
\pmprivacy{1}
\pmauthor{CWoo}{3771}
\pmtype{Definition}
\pmcomment{trigger rebuild}
\pmclassification{msc}{54E35}
\pmdefines{equivalent}

\endmetadata

\usepackage{amssymb,amscd}
\usepackage{amsmath}
\usepackage{amsfonts}
\usepackage{mathrsfs}

% used for TeXing text within eps files
%\usepackage{psfrag}
% need this for including graphics (\includegraphics)
%\usepackage{graphicx}
% for neatly defining theorems and propositions
\usepackage{amsthm}
% making logically defined graphics
%%\usepackage{xypic}
\usepackage{pst-plot}

% define commands here
\newcommand*{\abs}[1]{\left\lvert #1\right\rvert}
\newtheorem{prop}{Proposition}
\newtheorem{thm}{Theorem}
\newtheorem{ex}{Example}
\newcommand{\real}{\mathbb{R}}
\newcommand{\pdiff}[2]{\frac{\partial #1}{\partial #2}}
\newcommand{\mpdiff}[3]{\frac{\partial^#1 #2}{\partial #3^#1}}

\begin{document}
Let $X$ be a set equipped with two metrics $\rho$ and $\sigma$.  We say that $\rho$ is \emph{equivalent} to $\sigma$ (on $X$) if the identity map on $X$, is a homeomorphism between the metric topology on $X$ induced by $\rho$ and the metric topology on $X$ induced by $\sigma$.

For example, if $(X,\rho)$ is a metric space, then the function $\sigma: X\to \mathbb{R}$ defined by $$\sigma(x,y):=\frac{\rho(x,y)}{1+\rho(x,y)}$$
is a metric on $X$ that is equivalent to $\rho$.  This shows that every metric is equivalent to a bounded metric.

%%%%%
%%%%%
\end{document}
