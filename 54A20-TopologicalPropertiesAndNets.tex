\documentclass[12pt]{article}
\usepackage{pmmeta}
\pmcanonicalname{TopologicalPropertiesAndNets}
\pmcreated{2013-03-22 18:38:04}
\pmmodified{2013-03-22 18:38:04}
\pmowner{asteroid}{17536}
\pmmodifier{asteroid}{17536}
\pmtitle{topological properties and nets}
\pmrecord{6}{41374}
\pmprivacy{1}
\pmauthor{asteroid}{17536}
\pmtype{Feature}
\pmcomment{trigger rebuild}
\pmclassification{msc}{54A20}

% this is the default PlanetMath preamble.  as your knowledge
% of TeX increases, you will probably want to edit this, but
% it should be fine as is for beginners.

% almost certainly you want these
\usepackage{amssymb}
\usepackage{amsmath}
\usepackage{amsfonts}

% used for TeXing text within eps files
%\usepackage{psfrag}
% need this for including graphics (\includegraphics)
%\usepackage{graphicx}
% for neatly defining theorems and propositions
%\usepackage{amsthm}
% making logically defined graphics
%%%\usepackage{xypic}

% there are many more packages, add them here as you need them

% define commands here

\begin{document}
\PMlinkescapeword{Theorem}
\PMlinkescapeword{properties}
\PMlinkescapeword{converge}
\PMlinkescapeword{converges}
\PMlinkescapeword{convergent}
\PMlinkescapephrase{locally compact Hausdorff space}

Many topological properties and concepts can be translated in \PMlinkescapetext{terms} of convergence of nets. In this entry we give a list of this correspondence of properties. For detailed proofs please follow the \PMlinkescapetext{links}.

\section{Closure}

Let $X$ be a topological space and $Y \subseteq X$ a subset. A point $x \in X$ is in the closure of $Y$ if and only there exists a net in $Y$ \PMlinkname{converging}{Net} to $x$.

\begin{itemize}
\item For a detailed proof, \PMlinkname{click here}{NetsAndClosuresOfSubspaces}.
\end{itemize}

\section{Closed}

Let $X$ be a topological space. A subset $Y \subseteq X$ is closed if and only if every convergent net in $Y$ converges to a point in $Y$.

\section{Limit point}

Let $X$ be a topological space and $Y \subseteq X$ a subset. A point $x \in X$ is a limit point of $Y$ if and only if there is a net in $Y$ converging to $x$ that is not eventually constant.

\section{Hausdorff}

A topological space $X$ is Hausdorff if and only if every convergent net in $X$ has a unique limit.

\section{Compact}

A topological space $X$ is compact if and only if every net in $X$ has a convergent subnet.

\begin{itemize}
\item For a detailed proof, \PMlinkname{click here}{CompactnessAndConvergentSubnets}.
\end{itemize}

\section{Continuous}

Let $X$ and $Y$ be topological spaces. A function $f:X \to Y$ is continuous at a point $x \in X$ if and only if for every net $(x_i)_{i \in I}$ in $X$ converging to $x$, the net $(f(x_i))_{i \in I}$ converges to $f(x)$.

\begin{itemize}
\item For a detailed proof, \PMlinkname{click here}{ContinuityAndConvergentNets}.
\end{itemize}

\section{Open map}

Let $f:X \longrightarrow Y$ be a surjective map between the topological spaces $X$ and $Y$. Then $f$ is an open mapping if and only if given a net $\{y_i\}_{i \in I} \subset Y$ such that $y_i \longrightarrow y$, then for every $x \in f^{-1}(\{y\})$ there exists a subnet $\{y_{i_j}\}_{j \in J}$ that \PMlinkescapetext{lifts} to a net $\{x_{i_j}\}_{j \in J} \subset X$ such that $x_{i_j} \longrightarrow x$. By "\PMlinkescapetext{lift}" we \PMlinkescapetext{mean} that $\{x_{i_j}\}_{j \in J}$ is such that $ f(x_{i_j}) = y_{i_j}$.

\begin{itemize}
\item For a detailed proof, \PMlinkname{click here}{ContinuousSurjectiveOpenMapsInTermsOfNets}.
\end{itemize}

\section{Initial topology}

Let $X$ be a set, $\{X_{\alpha}\}_{\alpha \in \mathcal{A}}$ a family of topological spaces and $f_{\alpha}: X \to X_{\alpha}$ a family of functions. 

A net $(x_i)_{i \in I}$ in $X$ converges to a point $x$ in the initial topology of $X$ (relatively to the mappings $f_{\alpha}$) if and only if for each $\alpha \in \mathcal{A}$, $(f_{\alpha}(x_i))_{i \in I}$ converges to $f_{\alpha}(x)$.

\subsubsection{Particular case: subspace topology}

Let $X$ be a toplogical space and $Y \subseteq X$ a subset. A net $(y_i)_{i \in I}$ in $Y$ converges to $y \in Y$ in the subspace topology if and only if $(y_i)_{i \in I}$ converges to $y$ in the topology of $X$.

\subsubsection{Particular case: product topology}

Let $\{X_{\alpha}\}_{\alpha \in \mathcal{A}}$ be a collection of topological spaces and $X := \prod_{\alpha} X_{\alpha}$ their Cartesian product. A net $(x_i)_{i \in I}$ in $X$ converges to $x$ in the product topology if and only if every coordinate $(x_i^{\alpha})_{i \in I}$ converges to $x^{\alpha}$.

\section{Compact-open topology}

Let $X$ be a locally compact Hausdorff space, $Y$ a topological space and $C(X, Y)$ the set of continuous functions from $X$ to $Y$. A net $(f_i)_{i \in I}$ in $C(X, Y)$ converges to $f$ in the compact-open topology if and only if whenever a net $(x_i)_{i \in I}$ in $X$, indexed by the same directed set $I$, converges to $x \in X$, we also have that $(f_i(x_i))_{i \in I}$ converges to $f(x)$.

%%%%%
%%%%%
\end{document}
