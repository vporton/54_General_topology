\documentclass[12pt]{article}
\usepackage{pmmeta}
\pmcanonicalname{ZeroDimensional}
\pmcreated{2013-03-22 14:41:05}
\pmmodified{2013-03-22 14:41:05}
\pmowner{matte}{1858}
\pmmodifier{matte}{1858}
\pmtitle{zero dimensional}
\pmrecord{9}{36291}
\pmprivacy{1}
\pmauthor{matte}{1858}
\pmtype{Definition}
\pmcomment{trigger rebuild}
\pmclassification{msc}{54-00}
\pmsynonym{zero-dimensional}{ZeroDimensional}
\pmrelated{SeparationAxioms}

\endmetadata

% this is the default PlanetMath preamble.  as your knowledge
% of TeX increases, you will probably want to edit this, but
% it should be fine as is for beginners.

% almost certainly you want these
\usepackage{amssymb}
\usepackage{amsmath}
\usepackage{amsfonts}
\usepackage{amsthm}

\usepackage{mathrsfs}

% used for TeXing text within eps files
%\usepackage{psfrag}
% need this for including graphics (\includegraphics)
%\usepackage{graphicx}
% for neatly defining theorems and propositions
%
% making logically defined graphics
%%%\usepackage{xypic}

% there are many more packages, add them here as you need them

% define commands here

\newcommand{\sR}[0]{\mathbb{R}}
\newcommand{\sC}[0]{\mathbb{C}}
\newcommand{\sN}[0]{\mathbb{N}}
\newcommand{\sZ}[0]{\mathbb{Z}}

 \usepackage{bbm}
 \newcommand{\Z}{\mathbbmss{Z}}
 \newcommand{\C}{\mathbbmss{C}}
 \newcommand{\R}{\mathbbmss{R}}
 \newcommand{\Q}{\mathbbmss{Q}}



\newcommand*{\norm}[1]{\lVert #1 \rVert}
\newcommand*{\abs}[1]{| #1 |}



\newtheorem{thm}{Theorem}
\newtheorem{defn}{Definition}
\newtheorem{prop}{Proposition}
\newtheorem{lemma}{Lemma}
\newtheorem{cor}{Corollary}
\begin{document}
\begin{defn} \cite{steen, willard}
Suppose $X$ is a topological space. If $X$ has a basis consising of
clopen sets, then $X$ is said to be \PMlinkescapetext{\emph{zero dimensional}}. 
\end{defn}

Examples of zero-dimensional spaces are: the set $\mathbb{Q}$ of rational numbers (with subspace topology induced from the usual metric topology on $\mathbb{R}$, the set of real numbers), the Cantor space, as well as the Sorgenfrey line.

The concepts of zero-dimentionality and total disconnectedness are closely related.  Indeed, every zero-dimentional \PMlinkname{$T_1$ space}{T1Space} is totally disconnected.  Furthermore, if a topological space is locally compact and Hausdorff, then the notions of zero-dimentionality and total disconnectedness are equivalent.


\begin{thebibliography}{9}
\bibitem{steen} L.A. Steen, J.A.Seebach, Jr.,
\emph{Counterexamples in topology},
Holt, Rinehart and Winston, Inc., 1970.
\bibitem{willard} S. Willard, \emph{General Topology},
Addison-Wesley, Publishing Company, 1970.
\end{thebibliography}
%%%%%
%%%%%
\end{document}
