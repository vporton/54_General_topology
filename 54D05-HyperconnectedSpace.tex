\documentclass[12pt]{article}
\usepackage{pmmeta}
\pmcanonicalname{HyperconnectedSpace}
\pmcreated{2013-03-22 14:20:30}
\pmmodified{2013-03-22 14:20:30}
\pmowner{yark}{2760}
\pmmodifier{yark}{2760}
\pmtitle{hyperconnected space}
\pmrecord{10}{35813}
\pmprivacy{1}
\pmauthor{yark}{2760}
\pmtype{Definition}
\pmcomment{trigger rebuild}
\pmclassification{msc}{54D05}
\pmsynonym{hyper-connected space}{HyperconnectedSpace}
\pmrelated{UltraconnectedSpace}
\pmrelated{IrreducibleClosedSet}
\pmdefines{hyperconnected}
\pmdefines{hyper-connected}

\usepackage{amssymb}
\usepackage{amsmath}
\usepackage{amsfonts}
\begin{document}
\PMlinkescapeword{example}

A topological space $X$ is said to be \emph{hyperconnected} if no pair of nonempty open sets of $X$ is disjoint (or, equivalently, if $X$ is not the union of two proper closed sets).
Hyperconnected spaces are sometimes known as \PMlinkname{irreducible sets}{IrreducibleClosedSet}.

All hyperconnected spaces are connected, locally connected, and pseudocompact.

Any infinite set with the cofinite topology is an example of a hyperconnected space.
Similarly, any uncountable set with the cocountable topology is hyperconnected.
Affine spaces and projectives spaces over an infinite field, when endowed with the Zariski topology, are also hyperconnected.
%%%%%
%%%%%
\end{document}
