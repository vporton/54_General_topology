\documentclass[12pt]{article}
\usepackage{pmmeta}
\pmcanonicalname{RealTree}
\pmcreated{2013-03-22 15:16:55}
\pmmodified{2013-03-22 15:16:55}
\pmowner{GrafZahl}{9234}
\pmmodifier{GrafZahl}{9234}
\pmtitle{real tree}
\pmrecord{10}{37071}
\pmprivacy{1}
\pmauthor{GrafZahl}{9234}
\pmtype{Definition}
\pmcomment{trigger rebuild}
\pmclassification{msc}{54E99}
\pmclassification{msc}{54E40}
\pmsynonym{$\mathbb{R}$-tree}{RealTree}
\pmrelated{MetricSpace}
\pmrelated{Arc}
\pmrelated{Curve}
\pmrelated{SNCFMetric}
\pmrelated{Isometry}
\pmrelated{FreeGroup}
\pmrelated{HyperbolicGroup}

% this is the default PlanetMath preamble.  as your knowledge
% of TeX increases, you will probably want to edit this, but
% it should be fine as is for beginners.

% almost certainly you want these
\usepackage{amssymb}
\usepackage{amsmath}
\usepackage{amsfonts}
\usepackage[latin1]{inputenc}

% used for TeXing text within eps files
%\usepackage{psfrag}
% need this for including graphics (\includegraphics)
%\usepackage{graphicx}
% for neatly defining theorems and propositions
%\usepackage{amsthm}
% making logically defined graphics
%%%\usepackage{xypic}

% there are many more packages, add them here as you need them

% define commands here
\newcommand{\Bigcup}{\bigcup\limits}
\newcommand{\Prod}{\prod\limits}
\newcommand{\Sum}{\sum\limits}
\newcommand{\mbb}{\mathbb}
\newcommand{\mbf}{\mathbf}
\newcommand{\mc}{\mathcal}
\newcommand{\mmm}[9]{\left(\begin{array}{rrr}#1&#2&#3\\#4&#5&#6\\#7&#8&#9\end{array}\right)}
\newcommand{\ol}{\overline}

% Math Operators/functions
\DeclareMathOperator{\Frob}{Frob}
\DeclareMathOperator{\cwe}{cwe}
\DeclareMathOperator{\we}{we}
\DeclareMathOperator{\wt}{wt}
\begin{document}
A metric space $X$ is said to be a \emph{real tree} or
\emph{$\mbb{R}$-tree}, if for each $x,y\in X$ there is a unique arc
from $x$ to $y$, and furthermore this arc is an isometric \PMlinkid{embedding}{429}.

Every real tree is a hyperbolic metric space; moreover, every real tree is 0 hyperbolic.

The Cayley graph of any free group is considered to be a real tree.  Note that its graph is a tree in the graph theoretic sense.  To make it a real tree, we view the edges as \PMlinkname{isometric}{Isometric} to the line segment $[0,1]$ under a (surjective) \PMlinkname{isometry}{Isometry} and attach the edges to the tree.  The resulting 1-complex is then a locally finite real tree.  Because of this result, every free group is a hyperbolic group.
%%%%%
%%%%%
\end{document}
