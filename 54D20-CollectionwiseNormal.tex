\documentclass[12pt]{article}
\usepackage{pmmeta}
\pmcanonicalname{CollectionwiseNormal}
\pmcreated{2013-03-22 14:49:54}
\pmmodified{2013-03-22 14:49:54}
\pmowner{mathcam}{2727}
\pmmodifier{mathcam}{2727}
\pmtitle{collectionwise normal}
\pmrecord{5}{36497}
\pmprivacy{1}
\pmauthor{mathcam}{2727}
\pmtype{Definition}
\pmcomment{trigger rebuild}
\pmclassification{msc}{54D20}
\pmdefines{countably collectionwise normal}

\endmetadata

% this is the default PlanetMath preamble.  as your knowledge
% of TeX increases, you will probably want to edit this, but
% it should be fine as is for beginners.

% almost certainly you want these
\usepackage{amssymb}
\usepackage{amsmath}
\usepackage{amsfonts}
\usepackage{amsthm}

% used for TeXing text within eps files
%\usepackage{psfrag}
% need this for including graphics (\includegraphics)
%\usepackage{graphicx}
% for neatly defining theorems and propositions
%\usepackage{amsthm}
% making logically defined graphics
%%%\usepackage{xypic}

% there are many more packages, add them here as you need them

% define commands here

\newcommand{\mc}{\mathcal}
\newcommand{\mb}{\mathbb}
\newcommand{\mf}{\mathfrak}
\newcommand{\ol}{\overline}
\newcommand{\ra}{\rightarrow}
\newcommand{\la}{\leftarrow}
\newcommand{\La}{\Leftarrow}
\newcommand{\Ra}{\Rightarrow}
\newcommand{\nor}{\vartriangleleft}
\newcommand{\Gal}{\text{Gal}}
\newcommand{\GL}{\text{GL}}
\newcommand{\Z}{\mb{Z}}
\newcommand{\R}{\mb{R}}
\newcommand{\Q}{\mb{Q}}
\newcommand{\C}{\mb{C}}
\newcommand{\<}{\langle}
\renewcommand{\>}{\rangle}
\begin{document}
A Hausdorff topological space $X$ is called \emph{collectionwise normal} if any discrete collection of sets $\{U_i\}$ in $X$ can be covered by a pairwise-disjoint collection of open sets $\{V_j\}$ such that each $V_j$ covers just one $U_i$.  This is equivalent to requiring the same property for any discrete collection of closed sets.

A Hausdorff topological space $X$ is called \emph{countably collectionwise normal} if any countable discrete collection of sets $\{U_i\}$ in $X$ can be covered by a pairwise-disjoint collection of open sets $\{V_j\}$ such that each $V_j$ covers just one $U_i$.  This is equivalent to requiring the same property for any countable discrete collection of closed sets.

Any metrizable space is collectionwise normal.

\begin{thebibliography}{9}
\bibitem{a}
Steen, Lynn Arthur and Seebach, J. Arthur, \emph{Counterexamples in Topology}, Dover Books, 1995.
\end{thebibliography}
%%%%%
%%%%%
\end{document}
