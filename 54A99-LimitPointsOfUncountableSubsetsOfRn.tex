\documentclass[12pt]{article}
\usepackage{pmmeta}
\pmcanonicalname{LimitPointsOfUncountableSubsetsOfRn}
\pmcreated{2013-03-22 19:07:57}
\pmmodified{2013-03-22 19:07:57}
\pmowner{joking}{16130}
\pmmodifier{joking}{16130}
\pmtitle{limit points of uncountable subsets of R^n}
\pmrecord{6}{42028}
\pmprivacy{1}
\pmauthor{joking}{16130}
\pmtype{Theorem}
\pmcomment{trigger rebuild}
\pmclassification{msc}{54A99}

\endmetadata

% this is the default PlanetMath preamble.  as your knowledge
% of TeX increases, you will probably want to edit this, but
% it should be fine as is for beginners.

% almost certainly you want these
\usepackage{amssymb}
\usepackage{amsmath}
\usepackage{amsfonts}

% used for TeXing text within eps files
%\usepackage{psfrag}
% need this for including graphics (\includegraphics)
%\usepackage{graphicx}
% for neatly defining theorems and propositions
%\usepackage{amsthm}
% making logically defined graphics
%%%\usepackage{xypic}

% there are many more packages, add them here as you need them

% define commands here

\begin{document}
\textbf{Proposition.} Let $\mathbb{R}^n$ be an $n$-dimensional, real normed space and let $A\subseteq \mathbb{R}^n$. If $A$ is uncountable, then there exists limit point of $A$ in $\mathbb{R}^n$.

\textit{Proof.} For any $k\in\mathbb{N}$ let
$$\mathbb{B}_{k}=\{v\in\mathbb{R}^n\ |\ ||v||\leq k\},$$
i.e. $\mathbb{B}_k$ is a closed ball centered in $0$ with radius $k$. Assume, that for any $k$ the set 
$$V_k=\mathbb{B}_k\cap A$$
is finite. Then $\bigcup V_k=A$ would be at most countable. Contradiction, since $A$ is uncountable. Thus, there exists $k_0\in\mathbb{N}$ such that $V_{k_0}$ is infinite. But $V_{k_0}\subseteq\mathbb{B}_{k_0}$ and since $\mathbb{B}_{k_0}$ is compact (and $V_{k_0}$ is infinite), then there exists limit point of $V_{k_0}$ in $\mathbb{R}^n$. This completes the proof. $\square$

\textbf{Corollary.} If $A\subseteq\mathbb{R}^n$ is uncountable, then there exist infinitely many limit points of $A$ in $\mathbb{R}^n$.

\textit{Proof.} Assume, that there are finitely many limit points of $A$, namely $x_1,\ldots,x_k\in\mathbb{R}^n$. For $\varepsilon >0$ define
$$A_{\varepsilon}=\{v\in\mathbb{R}^n\ |\ \forall_{i}\ ||v-x_i||>\varepsilon\}.$$
Briefly speaking, $A_{\varepsilon}$ is a complement of a union of closed balls centered at $x_i$ with radii $\varepsilon$. Of course $A_{\varepsilon}\neq\emptyset$ since there are finitely many limit points. Let
$$V_{\varepsilon}=A\cap A_{\varepsilon}.$$
Assume, that $V_{\varepsilon}$ is countable for every $\varepsilon$. Then
$$A\subseteq \bigcup_{n\in\mathbb{N}}V_{\frac{1}{n}}\cup\{x_1,\ldots,x_k\}$$
would be at most countable (of course under assumption of Axiom of Choice). Contradiction. Thus, there is $\gamma > 0$ such that $V_{\gamma}$ is uncountable. Then (due to proposition) there is a limit point $x'\in\mathbb{R}^n$ of $V_{\gamma}$. Note, that 
$$x'\in \overline{V_{\gamma}}\subseteq V_{\gamma'}$$

for some $0 < \gamma' < \gamma $. Thus $x'$ is different from any $x_i$. Contradiction, since $x'$ is also a limit point of $A$. $\square$
%%%%%
%%%%%
\end{document}
