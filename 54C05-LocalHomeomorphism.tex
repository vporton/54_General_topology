\documentclass[12pt]{article}
\usepackage{pmmeta}
\pmcanonicalname{LocalHomeomorphism}
\pmcreated{2013-03-22 18:53:47}
\pmmodified{2013-03-22 18:53:47}
\pmowner{joking}{16130}
\pmmodifier{joking}{16130}
\pmtitle{local homeomorphism}
\pmrecord{4}{41743}
\pmprivacy{1}
\pmauthor{joking}{16130}
\pmtype{Definition}
\pmcomment{trigger rebuild}
\pmclassification{msc}{54C05}

% this is the default PlanetMath preamble.  as your knowledge
% of TeX increases, you will probably want to edit this, but
% it should be fine as is for beginners.

% almost certainly you want these
\usepackage{amssymb}
\usepackage{amsmath}
\usepackage{amsfonts}

% used for TeXing text within eps files
%\usepackage{psfrag}
% need this for including graphics (\includegraphics)
%\usepackage{graphicx}
% for neatly defining theorems and propositions
%\usepackage{amsthm}
% making logically defined graphics
%%%\usepackage{xypic}

% there are many more packages, add them here as you need them

% define commands here

\begin{document}
\textbf{Definition.} Let $X$ and $Y$ be topological spaces. Continuous map $f:X\to Y$ is said to be \textit{locally invertible in $x\in X$} iff there exist open subsets $U\subseteq X$ and $V\subseteq Y$ such that $x\in U$, $f(x)\in V$ and the restriction $$f:U\to V$$ is a homeomorphism. If $f$ is locally invertible in every point of $X$, then $f$ is called a \textit{local homeomorphism}.

\textbf{Examples.} Of course every homeomorphism is a local homeomorphism, but the converse is not true. For example, let $f:\mathbb{C}\to\mathbb{C}$ be an exponential function, i.e. $f(z)=e^z$. Then $f$ is a local homeomorphism, but it is not a homeorphism (indeed, $f(z)=f(z+2\pi i)$ for any $z\in\mathbb{C}$).

One of the most important theorem of differential calculus (i.e. inverse function theorem) states, that if $f:M\to N$ is a $C^1$-map between $C^1$-manifolds such that $T_{x}f:T_{x}M\to T_{f(x)}N$ is a linear isomorphism for a given $x\in M$, then $f$ is locally invertible in $x$ (in this case the local inverse is even a $C^1$-map).
%%%%%
%%%%%
\end{document}
