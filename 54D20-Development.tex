\documentclass[12pt]{article}
\usepackage{pmmeta}
\pmcanonicalname{Development}
\pmcreated{2013-03-22 14:49:49}
\pmmodified{2013-03-22 14:49:49}
\pmowner{mathcam}{2727}
\pmmodifier{mathcam}{2727}
\pmtitle{development}
\pmrecord{6}{36495}
\pmprivacy{1}
\pmauthor{mathcam}{2727}
\pmtype{Definition}
\pmcomment{trigger rebuild}
\pmclassification{msc}{54D20}
\pmdefines{developable}
\pmdefines{nested development}
\pmdefines{Vickery's theorem}

\endmetadata

% this is the default PlanetMath preamble.  as your knowledge
% of TeX increases, you will probably want to edit this, but
% it should be fine as is for beginners.

% almost certainly you want these
\usepackage{amssymb}
\usepackage{amsmath}
\usepackage{amsfonts}
\usepackage{amsthm}

% used for TeXing text within eps files
%\usepackage{psfrag}
% need this for including graphics (\includegraphics)
%\usepackage{graphicx}
% for neatly defining theorems and propositions
%\usepackage{amsthm}
% making logically defined graphics
%%%\usepackage{xypic}

% there are many more packages, add them here as you need them

% define commands here

\newcommand{\mc}{\mathcal}
\newcommand{\mb}{\mathbb}
\newcommand{\mf}{\mathfrak}
\newcommand{\ol}{\overline}
\newcommand{\ra}{\rightarrow}
\newcommand{\la}{\leftarrow}
\newcommand{\La}{\Leftarrow}
\newcommand{\Ra}{\Rightarrow}
\newcommand{\nor}{\vartriangleleft}
\newcommand{\Gal}{\text{Gal}}
\newcommand{\GL}{\text{GL}}
\newcommand{\Z}{\mb{Z}}
\newcommand{\R}{\mb{R}}
\newcommand{\Q}{\mb{Q}}
\newcommand{\C}{\mb{C}}
\newcommand{\<}{\langle}
\renewcommand{\>}{\rangle}
\begin{document}
Let $X$ be a topological space.  A \emph{development} for $X$ is a countable collection $F_1, F_2, \ldots$ of open coverings of $X$ such that for any closed subset $C$ of $X$ and any point $p$ in the complement of $C$, there exists a cover $F_j$ such that no element of $F_j$ which contains $p$ intersects $C$.  A space with a development is called \emph{developable}.

A development $F_1, F_2,\ldots$ such that $F_i\subset F_{i+1}$ for all $i$ is called a \emph{nested development}.  A theorem from Vickery states that every developable space in fact has a nested development.

\begin{thebibliography}{9}
\bibitem{a}
Steen, Lynn Arthur and Seebach, J. Arthur, \emph{Counterexamples in Topology}, Dover Books, 1995.
\end{thebibliography}
%%%%%
%%%%%
\end{document}
