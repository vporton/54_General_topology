\documentclass[12pt]{article}
\usepackage{pmmeta}
\pmcanonicalname{ProductTopology}
\pmcreated{2013-03-22 12:47:09}
\pmmodified{2013-03-22 12:47:09}
\pmowner{CWoo}{3771}
\pmmodifier{CWoo}{3771}
\pmtitle{product topology}
\pmrecord{38}{33100}
\pmprivacy{1}
\pmauthor{CWoo}{3771}
\pmtype{Definition}
\pmcomment{trigger rebuild}
\pmclassification{msc}{54B10}
\pmsynonym{Tychonoff product topology}{ProductTopology}
\pmrelated{BoxTopology}
\pmrelated{GeneralizedCartesianProduct}
\pmrelated{ASpaceMathnormalXIsHausdorffIfAndOnlyIfDeltaXIsClosed}
\pmrelated{InitialTopology}
\pmdefines{product}

\def\T{{\mathcal T}}
\def\closure{\overline}

\newtheorem{theorem}{Theorem}
\begin{document}
\PMlinkescapeword{component}
\PMlinkescapeword{components}
\PMlinkescapeword{continuous}
\PMlinkescapeword{equivalent}
\PMlinkescapeword{induced}
\PMlinkescapeword{restricted}
\PMlinkescapeword{term}
\PMlinkescapeword{terms}
\PMlinkescapeword{theorem}
\PMlinkescapeword{top}
\PMlinkescapeword{tychonoff}

\section*{Definition}

Let $((X_\alpha,\T_\alpha))_{\alpha\in A}$
be a family of topological spaces, and let $Y$ be the 
\PMlinkname{Cartesian product}{GeneralizedCartesianProduct} 
of the sets $X_\alpha$, that is
\[
  Y = \prod_{\alpha\in A} X_\alpha.
\]
Recall that an element $y\in Y$ is a function
$y\colon A\to \bigcup_{\alpha\in A} X_\alpha$ 
such that $y(\alpha) \in X_\alpha$ for each $\alpha \in A$,
and that for each $\alpha\in A$
the projection map $\pi_\alpha\colon Y\to X_\alpha$
is defined by $\pi_\alpha(y) = y(\alpha)$ for each $y\in Y$.

The (\emph{Tychonoff}) \emph{product topology} $\T$ for $Y$
is defined to be the initial topology with respect to the projection maps;
that is,
$\T$ is the smallest topology such that each $\pi_\alpha$ is \PMlinkname{continuous}{Continuous}.

\section*{Subbase}

If $U\subseteq X_\alpha$ is open,
then $\pi_\alpha^{-1}(U)$ is an open set in $Y$.
Note that this is the set of all elements of $Y$ 
in which the $\alpha$ component is restricted to $U$
and all other components are unrestricted.
The open sets of $Y$ are the unions of finite intersections of such sets.
That is,
\[
  \{\, \pi_\alpha^{-1}(U) \mid \alpha\in A\hbox{ and }U\in\T_\alpha \,\}
\]
is a subbase for the topology on $Y$.

\section*{Theorems}

The following theorems assume the product topology on
$\prod_{\alpha\in A}X_\alpha$.
Notation is as above.

\begin{theorem}
Let $Z$ be a topological space
and let $f\colon Z\to\prod_{\alpha\in A}X_\alpha$ be a function.
Then $f$ is continuous if and only if $\pi_\alpha\circ f$ is continuous
for each $\alpha\in A$.
\end{theorem}

\begin{theorem}
The product topology on $\prod_{\alpha\in A}X_\alpha$
is the topology induced by the subbase
\[
  \{ \pi_\alpha^{-1}(U)\mid \alpha\in A\mbox{ and }U\in \T_\alpha \}.
\]
\end{theorem}

\begin{theorem}
The product topology on $\prod_{\alpha\in A}X_\alpha$
is the topology induced by the base
\[
  \biggl\{ \bigcap_{\alpha\in F} \pi_\alpha^{-1}(U_\alpha)
  \,\biggm|\,
  F\mbox{ is a finite subset of }A
  \mbox{ and }U_\alpha\in\T_\alpha\mbox{ for each }\alpha\in F\biggr\}.
\]
\end{theorem}

\begin{theorem}
A net $(x_i)_{i \in I}$ in $\prod_{\alpha\in A}X_{\alpha}$ converges to $x$ if and only if each coordinate $(x_i^{\alpha})_{i \in I}$ converges to $x^{\alpha}$ in $X_{\alpha}$.
\end{theorem}

\begin{theorem}
Each projection map $\pi_\alpha\colon\prod_{\alpha\in A}X_\alpha\to X_\alpha$
is continuous and \PMlinkname{open}{OpenMapping}.
\end{theorem}

\begin{theorem}
For each $\alpha\in A$, let $A_\alpha\subseteq X_\alpha$.
Then
\[
  \closure{\prod_{\alpha\in A}A_\alpha}=\prod_{\alpha\in A}\closure{A_\alpha}.
\]
In particular, any product of closed sets is closed.
\end{theorem}

\begin{theorem}
{\PMlinkescapetext{\rm} (Tychonoff's Theorem)}
If each $X_\alpha$ is compact, then $\prod_{\alpha\in A}X_\alpha$ is compact.
\end{theorem}

\section*{Comparison with box topology}

There is another well-known way to topologize $Y$, namely the box topology.
The product topology is a subset of the box topology;
if $A$ is finite, then the two topologies are the same.

The product topology is generally more useful than the box topology.
The main reason for this can be expressed in terms of category theory:
the product topology is the topology of the
\PMlinkname{direct categorical product}{CategoricalDirectProduct} 
in the category \textbf{Top} (see Theorem 1 above).

\begin{thebibliography}{9}
\bibitem{kelley} J.~L.~Kelley, \emph{General Topology},
 D.~van Nostrand Company, Inc., 1955.
\bibitem{munkres} J.~Munkres, \emph{Topology} (2nd edition),
 Prentice Hall, 1999.
\end{thebibliography}
%%%%%
%%%%%
\end{document}
