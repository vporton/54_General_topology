\documentclass[12pt]{article}
\usepackage{pmmeta}
\pmcanonicalname{TheSphereIsIndecomposableAsATopologicalSpace}
\pmcreated{2013-03-22 18:31:45}
\pmmodified{2013-03-22 18:31:45}
\pmowner{joking}{16130}
\pmmodifier{joking}{16130}
\pmtitle{the sphere is indecomposable as a topological space}
\pmrecord{8}{41234}
\pmprivacy{1}
\pmauthor{joking}{16130}
\pmtype{Theorem}
\pmcomment{trigger rebuild}
\pmclassification{msc}{54F99}

\endmetadata

% this is the default PlanetMath preamble.  as your knowledge
% of TeX increases, you will probably want to edit this, but
% it should be fine as is for beginners.

% almost certainly you want these
\usepackage{amssymb}
\usepackage{amsmath}
\usepackage{amsfonts}

% used for TeXing text within eps files
%\usepackage{psfrag}
% need this for including graphics (\includegraphics)
%\usepackage{graphicx}
% for neatly defining theorems and propositions
%\usepackage{amsthm}
% making logically defined graphics
%%%\usepackage{xypic}

% there are many more packages, add them here as you need them

% define commands here

\begin{document}
\textbf{Proposition}. If for any topological spaces $X$ and $Y$ the $n$-dimensional sphere $\mathbb{S}^{n}$ is homeomorphic to $X\times Y$, then either $X$ has exactly one point or $Y$ has exactly one point.

\textit{Proof}. Recall that the homotopy group functor is additive, i.e. $\pi_{n}(X\times Y)\simeq\pi_{n}(X)\oplus\pi_{n}(Y)$. Assume that $\mathbb{S}^{n}$ is homeomorphic to $X\times Y$. Now $\pi_{n}(\mathbb{S}^{n})\simeq\mathbb{Z}$ and thus we have:
$$\mathbb{Z}\simeq\pi_{n}(\mathbb{S}^{n})\simeq\pi_{n}(X\times Y)\simeq\pi_{n}(X)\oplus\pi_{n}(Y).$$
Since $\mathbb{Z}$ is an indecomposable group, then either $\pi_{n}(X)\simeq 0$ or $\pi_{n}(Y)\simeq 0$.

Assume that $\pi_{n}(Y)\simeq 0$. Consider the map $p:X\times Y\to Y$ such that $p(x,y)=y$. Since $X\times Y$ is homeomorphic to $\mathbb{S}^{n}$ and $\pi_{n}(Y)\simeq 0$, then $p$ is homotopic to some constant map. Let $y_0\in Y$ and $H:\mathrm{I}\times X\times Y\to Y$ be such that 
$$H(0,x,y)=p(x,y)=y;$$
$$H(1,x,y)=y_0.$$
Consider the map $F:\mathrm{I}\times X\times Y\to X\times Y$ defined by the formula
$$F(t,x,y)=(x,H(t,x,y)).$$
Note that $F(0,x,y)=(x,y)$ and $F(1,x,y)=(x,y_{0})$ and thus $X\times\{y_0\}$ is a deformation retract of $X\times Y$. But $X\times Y$ is a sphere and spheres do not have proper deformation retracts (please see \PMlinkname{this entry}{EveryMapIntoSphereWhichIsNotOntoIsNullhomotopic} for more details). Therefore $X\times\{y_{0}\}=X\times Y$, so $Y=\{y_{0}\}$ has exactly one point. $\square$
%%%%%
%%%%%
\end{document}
