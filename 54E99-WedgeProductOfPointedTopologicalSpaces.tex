\documentclass[12pt]{article}
\usepackage{pmmeta}
\pmcanonicalname{WedgeProductOfPointedTopologicalSpaces}
\pmcreated{2013-03-22 18:49:09}
\pmmodified{2013-03-22 18:49:09}
\pmowner{MichaelMcCliment}{20205}
\pmmodifier{MichaelMcCliment}{20205}
\pmtitle{wedge product of pointed topological spaces}
\pmrecord{5}{41621}
\pmprivacy{1}
\pmauthor{MichaelMcCliment}{20205}
\pmtype{Definition}
\pmcomment{trigger rebuild}
\pmclassification{msc}{54E99}
\pmsynonym{wedge}{WedgeProductOfPointedTopologicalSpaces}
\pmsynonym{wedge product}{WedgeProductOfPointedTopologicalSpaces}
\pmrelated{QuotientSpace}
\pmrelated{CategoryOfPointedTopologicalSpaces}

% this is the default PlanetMath preamble.  as your knowledge
% of TeX increases, you will probably want to edit this, but
% it should be fine as is for beginners.

% almost certainly you want these
\usepackage{amssymb}
\usepackage{amsmath}
\usepackage{amsfonts}

% used for TeXing text within eps files
%\usepackage{psfrag}
% need this for including graphics (\includegraphics)
%\usepackage{graphicx}
% for neatly defining theorems and propositions
%\usepackage{amsthm}
% making logically defined graphics
%%%\usepackage{xypic}

% there are many more packages, add them here as you need them

% define commands here

\begin{document}
{\bf Definition.} Let $\{(X_i,x_i)\}_{i\in I}$ be a finite family of disjoint pointed topological spaces. The \emph{wedge product} of these spaces is

$$\bigvee_{i\in I} X_i = \left(\bigcup_{i\in I} X_i\right) / \{x_i: i\in I\}.$$

This can be generalized to arbitrary families of pointed topological spaces, although this may require that the topology on $\bigcup_{i\in I} X_i$ satisfy a coherence condition (see \cite{Munkres}).

\begin{thebibliography} {9}
\bibitem{Munkres} Munkres, J. R. (2000). \emph{Topology} (2nd. ed.). Upper Saddle River, NJ: Prentice Hall.
\bibitem{Prasolov} Prasolov, V. V. (2004). \emph{Elements of combinatorial and differential topology}. Providence, RI: American Mathematical Society.
\bibitem{Shick} Shick, P. L. (2007). \emph{Topology: Point-set and geometric}. Hoboken, NJ: John Wiley \& Sons.
\end{thebibliography}





%%%%%
%%%%%
\end{document}
