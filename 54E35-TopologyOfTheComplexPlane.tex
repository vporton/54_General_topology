\documentclass[12pt]{article}
\usepackage{pmmeta}
\pmcanonicalname{TopologyOfTheComplexPlane}
\pmcreated{2013-03-22 13:38:40}
\pmmodified{2013-03-22 13:38:40}
\pmowner{matte}{1858}
\pmmodifier{matte}{1858}
\pmtitle{topology of the complex plane}
\pmrecord{8}{34295}
\pmprivacy{1}
\pmauthor{matte}{1858}
\pmtype{Definition}
\pmcomment{trigger rebuild}
\pmclassification{msc}{54E35}
\pmclassification{msc}{30-00}
\pmrelated{IdentityTheorem}
\pmrelated{PlacesOfHolomorphicFunction}
\pmdefines{open disk}
\pmdefines{accumulation point}
\pmdefines{interior point}
\pmdefines{open}
\pmdefines{closed}
\pmdefines{bounded}
\pmdefines{compact}

\endmetadata

% this is the default PlanetMath preamble.  as your knowledge
% of TeX increases, you will probably want to edit this, but
% it should be fine as is for beginners.

% almost certainly you want these
\usepackage{amssymb}
\usepackage{amsmath}
\usepackage{amsfonts}

% used for TeXing text within eps files
%\usepackage{psfrag}
% need this for including graphics (\includegraphics)
%\usepackage{graphicx}
% for neatly defining theorems and propositions
%\usepackage{amsthm}
% making logically defined graphics
%%%\usepackage{xypic}

% there are many more packages, add them here as you need them

% define commands here

\newcommand{\sR}[0]{\mathbb{R}}
\newcommand{\sC}[0]{\mathbb{C}}
\newcommand{\sN}[0]{\mathbb{N}}
\newcommand{\sZ}[0]{\mathbb{Z}}
\begin{document}
The usual topology for the complex plane $\sC$
is the topology induced by the metric 
$$d(x,\,y) := |x\!-\!y|$$
for\, $x,\,y \in \sC$.
Here, $|\cdot|$ is the \PMlinkname{complex modulus}{ModulusOfComplexNumber}.

If we identify $\sR^2$ and $\sC$, it is clear that the above
topology coincides with topology induced by the Euclidean metric on $\sR^2$.

Some basic topological concepts for $\sC$:
\begin{enumerate}
\item The open balls 
$$B_r(\zeta) \;=\; \{z\in\sC\,\vdots\; |z\!-\!\zeta| < r\}$$
are often called \emph{open disks}.
\item A point $\zeta$ is an \emph{accumulation point} of a subset $A$ of $\sC$, if any open disk $B_r(\zeta)$ contains at least one point of $A$ distinct from $\zeta$.
\item A point $\zeta$ is an \emph{interior point} of the set $A$, if there exists an open disk $B_r(\zeta)$ which is contained in $A$.
\item A set $A$ is \emph{open}, if each of its points is an interior point of $A$.
\item A set $A$ is \emph{closed}, if all its accumulation points belong to $A$.
\item A set $A$ is \emph{bounded}, if there is an open disk $B_r(\zeta)$ containing $A$.
\item A set $A$ is \emph{compact}, if it is closed and bounded.
\end{enumerate}
%%%%%
%%%%%
\end{document}
