\documentclass[12pt]{article}
\usepackage{pmmeta}
\pmcanonicalname{FiniteIntersectionProperty}
\pmcreated{2013-03-22 13:34:05}
\pmmodified{2013-03-22 13:34:05}
\pmowner{azdbacks4234}{14155}
\pmmodifier{azdbacks4234}{14155}
\pmtitle{finite intersection property}
\pmrecord{17}{34178}
\pmprivacy{1}
\pmauthor{azdbacks4234}{14155}
\pmtype{Definition}
\pmcomment{trigger rebuild}
\pmclassification{msc}{54D30}
\pmsynonym{finite intersection condition}{FiniteIntersectionProperty}
\pmsynonym{f.i.c.}{FiniteIntersectionProperty}
\pmsynonym{f.i.p.}{FiniteIntersectionProperty}
%\pmkeywords{compact}
%\pmkeywords{intersection}
%\pmkeywords{finite}
\pmrelated{Compact}
\pmrelated{Intersection}
\pmrelated{Finite}
\pmdefines{finite intersection property}

%%packages
\usepackage{amssymb}
\usepackage{amsmath}
\usepackage{amsfonts}
\usepackage{amsthm}
%%theorem environments
\theoremstyle{plain}
\newtheorem*{theorem*}{Theorem}
\newtheorem*{lemma*}{Lemma}
\newtheorem*{corollary*}{Corollary}
\newtheorem*{proposition*}{Proposition}
%math operators and commands
\newcommand{\set}[1]{\{#1\}}
\newcommand{\medset}[1]{\left\{#1\right\}}
\newcommand{\bigset}[1]{\bigg\{#1\bigg\}}
\newcommand{\abs}[1]{\vert#1\vert}
\newcommand{\medabs}[1]{\left\vert#1\right\vert}
\newcommand{\bigabs}[1]{\bigg\vert#1\bigg\vert}
\newcommand{\norm}[1]{\Vert#1\Vert}
\newcommand{\mednorm}[1]{\left\Vert#1\right\Vert}
\newcommand{\bignorm}[1]{\bigg\Vert#1\bigg\Vert}
\DeclareMathOperator{\Aut}{Aut}
\DeclareMathOperator{\End}{End}
\DeclareMathOperator{\Inn}{Inn}
\DeclareMathOperator{\supp}{supp}

\begin{document}
A collection $\mathcal{A}=\set{A_\alpha}_{\alpha\in I}$ of subsets of a set $X$ is said to have the \emph{finite intersection property}, abbreviated f.i.p., if every finite subcollection $\set{A_1,A_2,\ldots,A_n}$ of $\mathcal{A}$ satisifes $\bigcap_{i=1}^nA_i\neq\emptyset$. 

The finite intersection property is most often used to give the following \PMlinkid{equivalent}{3769} condition for the \PMlinkid{compactness}{503} of a topological space (a proof of which may be found \PMlinkid{here}{4181}):

\begin{proposition*}
A topological space $X$ is compact if and only if for every collection $\mathcal{C}=\set{C_\alpha}_{\alpha\in J}$ of closed subsets of $X$ having the finite intersection property, $\bigcap_{\alpha\in J}C_\alpha\neq\emptyset$.
\end{proposition*}

An important special case of the preceding \PMlinkescapeword{proposition} is that in which $\mathcal{C}$ is a countable collection of non-empty nested sets, i.e., when we have
\begin{equation*}
C_1\supset C_2\supset C_3\supset\cdots\text{.}
\end{equation*}
In this case, $\mathcal{C}$ automatically has the finite intersection property, and if each $C_i$ is a closed subset of a compact topological space, then, by the proposition, $\bigcap_{i=1}^\infty C_i\neq\emptyset$.

The f.i.p. characterization of \PMlinkescapetext{compactness} may be used to prove a general result on the uncountability of certain compact Hausdorff spaces, and is also used in a proof of Tychonoff's Theorem.


\begin{thebibliography}{1}
\bibitem{Munkres}
J. Munkres, \emph{Topology}, 2nd ed. Prentice Hall, 1975.
\end{thebibliography}


 
%%%%%
%%%%%
\end{document}
