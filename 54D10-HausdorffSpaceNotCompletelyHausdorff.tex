\documentclass[12pt]{article}
\usepackage{pmmeta}
\pmcanonicalname{HausdorffSpaceNotCompletelyHausdorff}
\pmcreated{2013-03-22 14:16:05}
\pmmodified{2013-03-22 14:16:05}
\pmowner{drini}{3}
\pmmodifier{drini}{3}
\pmtitle{Hausdorff space not completely Hausdorff}
\pmrecord{21}{35718}
\pmprivacy{1}
\pmauthor{drini}{3}
\pmtype{Example}
\pmcomment{trigger rebuild}
\pmclassification{msc}{54D10}
\pmsynonym{$T_2$ space not $T_{2\frac12}$}{HausdorffSpaceNotCompletelyHausdorff}
\pmsynonym{example of a Hausdorff space that is not completely Hausdorff}{HausdorffSpaceNotCompletelyHausdorff}
\pmrelated{CompletelyHausdorff}
\pmrelated{SeparationAxioms}
\pmrelated{FrechetSpace}
\pmrelated{RegularSpace}
\pmrelated{FurstenbergsProofOfTheInfinitudeOfPrimes}
\pmrelated{SeparationAxioms}
\pmrelated{T2Space}

%%%\usepackage{xypic} 
\usepackage{bbm}
\newcommand{\Z}{\mathbbmss{Z}}
\newcommand{\B}{\mathbbmss{B}}
\newcommand{\R}{\mathbbmss{R}}
\newcommand{\Q}{\mathbbmss{Q}}
\newcommand{\lcm}{\mathrm{lcm}}
\newcommand{\mathbb}[1]{\mathbbmss{#1}}
\newcommand{\figura}[1]{\begin{center}\includegraphics{#1}\end{center}}
\newcommand{\figuraex}[2]{\begin{center}\includegraphics[#2]{#1}\end{center}}
\newtheorem{dfn}{Definition}
\begin{document}
On the set $\Z^+$ of strictly 
positive integers, let $a$ and $b$ be two different integers $b\neq0$ and consider the set
\[ S(a,b)=\{a+kb\in \Z^+ \colon k\in \Z\}\]
such set is the infinite arithmetic progression of positive integers with difference $b$ and containing $a$.
The collection of all $S(a,b)$ sets is a basis for a topology on $\Z^+$. We will use a coarser topology induced by the following basis:
\[\mathbb{B}=\{S(a,b): \gcd(a,b)=1\}\]

\subsubsection*{The collection $\B$ is basis for a topology on $\Z^+$}
We first prove such collection is a basis. 
Suppose $x\in S(a,b)\cap S(c,d)$. By Euclid's algorithm we have $S(a,b)=S(x,b)$ and $S(c,d)=S(x,d)$ and 
\[ 
x\in S(x,bd)\subset S(x,d)\cap S(c,d)
\]
besides, since $\gcd(x,b)=1$ and $\gcd(x,d)=1$ then $\gcd(x,bd)=1$ so $x$ and $bd$ are coprimes and $S(x,bd)\in \mathbb{B}$. This concludes the proof that $\B$ is indeed a basis for a topology on $\Z^+$.

\subsubsection*{The topology on $\Z^+$ induced by $\B$ is Hausdorff}
Let $m,n$ integers two different integers.
We need to show that there are open disjoint neighborhoods $U_m$ and $U_n$ such that $m\in U_m$ and $n\in U_n$, but it suffices to show the existence of disjoint basic open sets containing $m$ and $n$.

Taking $d=|m-n|$,  we can find an integer $t$ such that $t>d$ and such that
$\gcd(m,t)=\gcd(n,t)=1$. A way to accomplish this is to take any multiple of $mn$ greater than $d$ and add $1$.

The basic open sets $S(m,t)$ and $S(n,t)$ are disjoint, because they have common elements if and only if the diophantine equation $m+tx = n+ty$ has solutions. But it cannot have since $t(x-y) = n-m$ implies that $t$ divides $n-m$ but $t>|n-m|$ makes it impossible.

We conclude that $S(m,t)\cap S(n,t)=\emptyset$ and this means that $\Z^+$ becomes a Hausdorff space with the given topology.

\subsubsection*{Some properties of $\overline{S(a,b)}$}
We need to determine first some facts about $\overline{S(a,b)}$. in order to take an example, consider $S(3,5)$ first. Notice that if we had considered the former topology (where in $S(a,b)$, $a$ and $b$ didn't have to be coprime) the complement of $S(3,5)$ would have been $S(4,5)\cup S(5,5)\cup S(6,5)\cup S(7,5)$ which is open, and so $S(3,5)$ would have been closed. In general, in the finer topology, all basic sets were both open and closed. However, this is not true in our coarser topology (for instance $S(5,5)$ is not open). 

The key fact to prove $\Z^+$ is not a completely Hausdorff space is: given any $S(a,b)$, then $b\Z^+ = \{n \in \Z^+ : b \mbox{ divides }n \}$ is a subset of $\overline{S(a,b)}$.

Indeed, any basic open set containing $bk$ is of the form $S(bk,t)$ with $t,bk$ coprimes. This means $\gcd(t,b)=1$. Now $S(bk,t)$ and $S(a,b)$ have common terms if an only if $bk+tx = a+by$ for some integers $x,y$. But that diophantine equation can be rewritten as
\[ tx - by = a-bk\]
and it always has solutions because $1=\gcd(t,b)$ divides $a-bk$.

This also proves  $S(a,b)\neq \overline{S(a,b)}$, because $b$ is not in $S(a,b)$ but it is on the closure.

\subsubsection*{The topology on $\Z^+$ induced by $\B$ is not completely Hausdorff}
We will use the closed-neighborhood sense for completely Hausdorff, which will also imply the topology is not completely Hausdorff in the functional sense.

Let $m,n$ different positive integers. Since $\B$ is a basis, for any two disjoint neighborhoods $U_m, U_n$ we can find basic sets $S(m,a)$ and $S(n,b)$ such that
\[m\in S(m,a)\subseteq U_m,\qquad n\in S(n,b)\subseteq U_n\]
and thus 
\[S(m,a)\cap S(n,b)=\emptyset.\]

But then $g=ab$ is both a multiple of $a$ and $b$ so it must be in $\overline{S(m,a)}$ and $\overline{S(n,b)}$. This means
\[\overline{S(m,a)}\cap\overline{S(n,b)}\neq\emptyset\]
and thus $\overline{U_m}\cap \overline{U_n}\neq\emptyset$.

This proves the topology under consideration is not completely 
Hausdorff (under both usual meanings).
%%%%%
%%%%%
\end{document}
