\documentclass[12pt]{article}
\usepackage{pmmeta}
\pmcanonicalname{DiagonalEmbedding}
\pmcreated{2013-03-22 14:20:41}
\pmmodified{2013-03-22 14:20:41}
\pmowner{waj}{4416}
\pmmodifier{waj}{4416}
\pmtitle{diagonal embedding}
\pmrecord{8}{35817}
\pmprivacy{1}
\pmauthor{waj}{4416}
\pmtype{Definition}
\pmcomment{trigger rebuild}
\pmclassification{msc}{54B10}
\pmclassification{msc}{18A05}
\pmsynonym{diagonal map}{DiagonalEmbedding}
\pmrelated{ASpaceMathnormalXIsHausdorffIfAndOnlyIfDeltaXIsClosed}

% this is the default PlanetMath preamble.  as your knowledge
% of TeX increases, you will probably want to edit this, but
% it should be fine as is for beginners.

% almost certainly you want these
\usepackage{amssymb}
\usepackage{amsmath}
\usepackage{amsfonts}

% used for TeXing text within eps files
%\usepackage{psfrag}
% need this for including graphics (\includegraphics)
%\usepackage{graphicx}
% for neatly defining theorems and propositions
%\usepackage{amsthm}
% making logically defined graphics
%%%\usepackage{xypic}

% there are many more packages, add them here as you need them

% define commands here
\begin{document}
\PMlinkescapeword{structure}
\PMlinkescapeword{information}
\PMlinkescapeword{algebraic}
\PMlinkescapeword{compact}

Given a topological space $X$, the \emph{diagonal embedding}, or \emph{diagonal map} of $X$ into $X\times X$ (with the product topology) is the map

\[x\stackrel{\Delta}{\longmapsto}(x,x) .\]

$X$ is homeomorphic to the image of $\Delta$ (which is why we use the word ``embedding{}'').


We can perform the same construction with objects other than topological spaces: for instance, there's a diagonal map $\Delta\colon G\to G\times G$, from a group into its direct sum with itself, given by the same \PMlinkescapetext{formula}.  It's sensible to call this an embedding, too, since $\Delta$ is a monomorphism.

We could also imagine a diagonal map into an n-fold product given by

\[x\stackrel{\Delta_n}{\longmapsto}(x,x,\ldots ,x).\]

\subsubsection*{Why call it the diagonal map?}

Picture $\mathbb{R}$.  Its diagonal embedding into the Cartesian plane $\mathbb{R}\times\mathbb{R}$ is the diagonal line $y=x$.

\subsubsection*{What's it good for?}

Sometimes we can use information about the product space $X\times X$ together with the diagonal embedding to get back information about $X$.  For instance, $X$ is Hausdorff if and only if the image of $\Delta$ is closed in $X\times X$ [\PMlinkname{proof}{ASpaceMathnormalXIsHausdorffIfAndOnlyIfDeltaXIsClosed}].  If we know more about the product space than we do about $X$, it might be easier to check if $\operatorname{Im}\Delta$ is closed than to verify the Hausdorff condition directly.

When studying algebraic topology, the fact that we have a diagonal embedding for any space $X$ lets us define a bit of extra structure in cohomology, called the cup product.  This makes cohomology into a ring, so that we can bring additional algebraic muscle to bear on topological questions.

Another application from algebraic topology: there is something called an $H$-space, which is essentially a topological space in which you can multiply two points together.  The diagonal embedding, together with the multiplication, lets us say that the cohomology of an $H$-space is a Hopf algebra; this structure lets us find out lots of things about $H$-spaces by analogy to what we know about compact Lie groups.
%%%%%
%%%%%
\end{document}
