\documentclass[12pt]{article}
\usepackage{pmmeta}
\pmcanonicalname{ClosedSetInASubspace}
\pmcreated{2013-03-22 15:33:32}
\pmmodified{2013-03-22 15:33:32}
\pmowner{yark}{2760}
\pmmodifier{yark}{2760}
\pmtitle{closed set in a subspace}
\pmrecord{9}{37460}
\pmprivacy{1}
\pmauthor{yark}{2760}
\pmtype{Theorem}
\pmcomment{trigger rebuild}
\pmclassification{msc}{54B05}

% almost certainly you want these
\usepackage{amssymb}
\usepackage{amsmath}
\usepackage{amsfonts}
\usepackage{amsthm}

\usepackage{mathrsfs}

% used for TeXing text within eps files
%\usepackage{psfrag}
% need this for including graphics (\includegraphics)
%\usepackage{graphicx}
% for neatly defining theorems and propositions
%
% making logically defined graphics
%%%\usepackage{xypic}

% there are many more packages, add them here as you need them

% define commands here

\newcommand{\sR}[0]{\mathbb{R}}
\newcommand{\sC}[0]{\mathbb{C}}
\newcommand{\sN}[0]{\mathbb{N}}
\newcommand{\sZ}[0]{\mathbb{Z}}

 \usepackage{bbm}
 \newcommand{\Z}{\mathbbmss{Z}}
 \newcommand{\C}{\mathbbmss{C}}
 \newcommand{\F}{\mathbbmss{F}}
 \newcommand{\R}{\mathbbmss{R}}
 \newcommand{\Q}{\mathbbmss{Q}}

\newcommand*{\norm}[1]{\lVert #1 \rVert}
\newcommand*{\abs}[1]{| #1 |}

\newtheorem{thm}{Theorem}
\newtheorem{defn}{Definition}
\newtheorem{prop}{Proposition}
\newtheorem{lemma}{Lemma}
\newtheorem{cor}{Corollary}
\begin{document}
\PMlinkescapeword{closed}
\PMlinkescapeword{open}
\PMlinkescapeword{theorem}

% This is written from scratch with no external references.
% These kind of results seem to be needed to prove
%   http://planetmath.org/encyclopedia/ClosedSubsetsOfACompactSetAreCompact.html
% using the FIP.

In the following, let $X$ be a topological space. 

\begin{thm}
Suppose $Y\subseteq X$ is equipped with the subspace topology,
and $A\subseteq Y$.
Then $A$ is \PMlinkname{closed}{ClosedSet} in $Y$ if and only if
$A=Y\cap J$ for some closed set $J\subseteq X$.
\end{thm}

\begin{proof} 
If $A$ is closed in $Y$,
then $Y\setminus A$ is \PMlinkname{open}{OpenSet} in $Y$, 
and by the definition of the subspace topology,
$Y\setminus A = Y\cap U$ for some open $U\subseteq X$. 
Using \PMlinkname{properties of the set difference}{SetDifference}, 
we obtain
\begin{eqnarray*}
  A &=& Y\setminus (Y\setminus A) \\
    &=& Y\setminus(Y\cap U) \\
    &=& Y\setminus U \\
    &=& Y\cap U^\complement.
\end{eqnarray*}
On the other hand, if $A=Y\cap J$ for some closed $J\subseteq X$,
then $Y\setminus A = Y\setminus(Y\cap J) = Y\cap J^\complement$,
and so $Y\setminus A$ is open in $Y$,
and therefore $A$ is closed in $Y$.
\end{proof}

\begin{thm}
Suppose $X$ is a topological space, $C\subseteq X$ 
is a closed set equipped with the subspace topology, 
and $A\subseteq C$ is closed in $C$. 
Then $A$ is closed in $X$. 
\end{thm}

\begin{proof} This follows from the previous theorem:
since $A$ is closed in $C$, 
we have $A = C\cap J$ for some closed set $J\subseteq X$,
and $A$ is closed in $X$.
\end{proof}
%%%%%
%%%%%
\end{document}
