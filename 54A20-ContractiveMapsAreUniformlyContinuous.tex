\documentclass[12pt]{article}
\usepackage{pmmeta}
\pmcanonicalname{ContractiveMapsAreUniformlyContinuous}
\pmcreated{2013-03-22 13:46:28}
\pmmodified{2013-03-22 13:46:28}
\pmowner{mathcam}{2727}
\pmmodifier{mathcam}{2727}
\pmtitle{contractive maps are uniformly continuous}
\pmrecord{6}{34481}
\pmprivacy{1}
\pmauthor{mathcam}{2727}
\pmtype{Theorem}
\pmcomment{trigger rebuild}
\pmclassification{msc}{54A20}

\endmetadata

% this is the default PlanetMath preamble.  as your knowledge
% of TeX increases, you will probably want to edit this, but
% it should be fine as is for beginners.

% almost certainly you want these
\usepackage{amssymb}
\usepackage{amsmath}
\usepackage{amsfonts}

% used for TeXing text within eps files
%\usepackage{psfrag}
% need this for including graphics (\includegraphics)
%\usepackage{graphicx}
% for neatly defining theorems and propositions
%\usepackage{amsthm}
% making logically defined graphics
%%%\usepackage{xypic}

% there are many more packages, add them here as you need them

% define commands here

\newcommand{\sR}[0]{\mathbb{R}}
\newcommand{\sC}[0]{\mathbb{C}}
\newcommand{\sN}[0]{\mathbb{N}}
\newcommand{\sZ}[0]{\mathbb{Z}}
\begin{document}
{\bf Theorem} A contraction mapping is uniformly continuous.

{\bf Proof} Let $T:X\to X$ be a contraction mapping in a metric space
$X$ with metric $d$. Thus, for some $q\in [0,1)$, we have
for all $x,y\in X$,
$$ d(Tx,Ty)\le q d(x,y).$$
To prove that $T$ is uniformly continuous, let $\varepsilon>0$ be given.
There are two cases.
If $q=0$, our claim is trivial, since then for all $x,y\in X$,
 $$ d(Tx,Ty)=0<\varepsilon.$$
On the other hand, suppose $q\in(0,1)$. Then for all $x,y\in X$ with
$d(x,y)<\varepsilon/q$, we have
$$ d(Tx,Ty) \le q d(x,y) < \varepsilon.$$
In conclusion, $T$ is uniformly continuous. $\Box$

The result is stated without proof in \cite{rudin}, pp. 221.

\begin{thebibliography}{9}
 \bibitem{rudin}
 W. Rudin, \emph{Principles of Mathematical Analysis}, McGraw-Hill Inc., 1976.
\end{thebibliography}
%%%%%
%%%%%
\end{document}
