\documentclass[12pt]{article}
\usepackage{pmmeta}
\pmcanonicalname{TietzeExtensionTheorem}
\pmcreated{2013-03-22 13:35:30}
\pmmodified{2013-03-22 13:35:30}
\pmowner{matte}{1858}
\pmmodifier{matte}{1858}
\pmtitle{Tietze extension theorem}
\pmrecord{5}{34215}
\pmprivacy{1}
\pmauthor{matte}{1858}
\pmtype{Theorem}
\pmcomment{trigger rebuild}
\pmclassification{msc}{54D15}
\pmrelated{ApplicationsOfUrysohnsLemmaToLocallyCompactHausdorffSpaces}

% this is the default PlanetMath preamble.  as your knowledge
% of TeX increases, you will probably want to edit this, but
% it should be fine as is for beginners.

% almost certainly you want these
\usepackage{amssymb}
\usepackage{amsmath}
\usepackage{amsfonts}

% used for TeXing text within eps files
%\usepackage{psfrag}
% need this for including graphics (\includegraphics)
%\usepackage{graphicx}
% for neatly defining theorems and propositions
%\usepackage{amsthm}
% making logically defined graphics
%%%\usepackage{xypic}

% there are many more packages, add them here as you need them

% define commands here
\begin{document}
Let $X$ be a topological space. Then the following are equivalent:
\begin{enumerate}
\item $X$ is normal.
\item If $A$ is a closed subset in $X$, and $f\colon A\to [-1,1]$ is a
continuous function, then $f$ has a continuous 
\PMlinkescapetext{extension} to all of $X$.
(In other words, there is a continuous function $f^\ast\colon X\to [-1,1]$ such that
$f$ and $f^\ast$ coincide on $A$.)
\end{enumerate}

\emph{Remark:}
If $X$ and $A$ are as above, and $f\colon A\to(-1,1)$ is a continuous function, then $f$ has a continuous \PMlinkescapetext{extension} to all of $X$.

The present result can be found in \cite{mukherjea}.
\begin{thebibliography}{9}
\bibitem{mukherjea}
A. Mukherjea, K. Pothoven,
\emph{Real and Functional analysis},
Plenum press, 1978.
\end{thebibliography}
%%%%%
%%%%%
\end{document}
