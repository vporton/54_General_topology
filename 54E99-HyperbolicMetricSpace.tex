\documentclass[12pt]{article}
\usepackage{pmmeta}
\pmcanonicalname{HyperbolicMetricSpace}
\pmcreated{2013-03-22 17:11:29}
\pmmodified{2013-03-22 17:11:29}
\pmowner{Wkbj79}{1863}
\pmmodifier{Wkbj79}{1863}
\pmtitle{hyperbolic metric space}
\pmrecord{6}{39509}
\pmprivacy{1}
\pmauthor{Wkbj79}{1863}
\pmtype{Definition}
\pmcomment{trigger rebuild}
\pmclassification{msc}{54E99}
\pmclassification{msc}{54E35}
\pmclassification{msc}{20F06}
\pmdefines{$\delta$ hyperbolic}

\usepackage{amssymb}
\usepackage{amsmath}
\usepackage{amsfonts}
\usepackage{pstricks}
\usepackage{psfrag}
\usepackage{graphicx}
\usepackage{amsthm}
%%\usepackage{xypic}

\begin{document}
Let $\delta \ge 0$.  A metric space $(X,d)$ is \emph{$\delta$ hyperbolic} if, for any figure $ABC$ in $X$ that is a geodesic triangle with respect to $d$ and for every $P \in \overline{AB}$, there exists a point $Q \in \overline{AC} \cup \overline{BC}$ such that $d(P,Q) \le \delta$.

A \emph{hyperbolic metric space} is a metric space that is $\delta$ hyperbolic for some $\delta \ge 0$.

Although a metric space is hyperbolic if it is $\delta$ hyperbolic for some $\delta \ge 0$, one usually tries to find the smallest value of $\delta$ for which a hyperbolic metric space $(X,d)$ is $\delta$ hyperbolic.

A \PMlinkescapetext{simple} example of a hyperbolic metric space is the real line under the usual metric.  Given any three points $A,B,C \in \mathbb{R}$, we always have that $\overline{AB} \subseteq \overline{AC} \cup \overline{BC}$.  Thus, for any $P \in \overline{AB}$, we can take $Q=P$.  Therefore, the real line is 0 hyperbolic.  \PMlinkescapetext{Similar} reasoning can be used to show that every real tree is 0 hyperbolic.
%%%%%
%%%%%
\end{document}
