\documentclass[12pt]{article}
\usepackage{pmmeta}
\pmcanonicalname{CompletelyHausdorff}
\pmcreated{2013-03-22 14:16:03}
\pmmodified{2013-03-22 14:16:03}
\pmowner{PrimeFan}{13766}
\pmmodifier{PrimeFan}{13766}
\pmtitle{completely Hausdorff}
\pmrecord{15}{35717}
\pmprivacy{1}
\pmauthor{PrimeFan}{13766}
\pmtype{Definition}
\pmcomment{trigger rebuild}
\pmclassification{msc}{54D10}
\pmsynonym{completely Hausdorff space}{CompletelyHausdorff}
\pmsynonym{$T_{2\frac{1}{2}}$}{CompletelyHausdorff}
\pmsynonym{Urysohn space}{CompletelyHausdorff}
\pmrelated{HausdorffSpaceNotCompletelyHausdorff}

\endmetadata

\usepackage{graphicx}
%%%\usepackage{xypic} 
\usepackage{bbm}
\newcommand{\Z}{\mathbbmss{Z}}
\newcommand{\C}{\mathbbmss{C}}
\newcommand{\R}{\mathbbmss{R}}
\newcommand{\Q}{\mathbbmss{Q}}
\newcommand{\mathbb}[1]{\mathbbmss{#1}}
\newcommand{\figura}[1]{\begin{center}\includegraphics{#1}\end{center}}
\newcommand{\figuraex}[2]{\begin{center}\includegraphics[#2]{#1}\end{center}}
\newtheorem{dfn}{Definition}

\usepackage{amsmath}
\usepackage{amsthm}

\newtheorem{thm}{Theorem}
\newtheorem{defn}{Definition}
\newtheorem{prop}{Proposition}
\newtheorem{lemma}{Lemma}
\newtheorem{cor}{Corollary}
\begin{document}
\begin{defn} \cite{steen}
Let $(X,\tau)$ be a topological space. 
Suppose that for any two different points $x,y\in X, x\neq y$, 
we can find two disjoint neighborhoods 
\[U_x,V_y\in \tau,\qquad x\in U_x, y\in Y_y\] such that their 
closures are also disjoint: 
\[\overline{U_x}\cap \overline{V_y}=\emptyset.\]
Then we say that $(X,\tau)$ is a 
\emph{completely Hausdorff space} or a \emph{$T_{2\frac12}$ space}. 
\end{defn}

\subsubsection*{Notes}
A synonym for functionally Hausdorff space is 
   \emph{Urysohn space} \cite{steen}. 
Unfortunately, the definition of completely Hausdorff and $T_{2\frac12}$ 
are not as standard as one would like since. For example, the
term completely Hausdorff space is also used to mean 
a functionally Hausdorff space (e.g. \cite{willard}).
Nevertheless, in the present convention, we have the implication:
\[
  \mbox{functionally Hausdorff}
    \Rightarrow
  \mbox{completely Hausdorff}
    \Rightarrow 
  T_2=\mbox{Hausdorff},
\]
which suggests why the $T_{2\frac12}$ name have been used to 
denote both completely Hausdorff spaces and functionally Hausdorff spaces.

\begin{thebibliography}{9}
\bibitem{steen} L.A. Steen, J.A.Seebach, Jr.,
\emph{Counterexamples in topology},
Holt, Rinehart and Winston, Inc., 1970.
\bibitem{willard} S. Willard, \emph{General Topology},
Addison-Wesley Publishing Company, 1970.
\end{thebibliography}
%%%%%
%%%%%
\end{document}
