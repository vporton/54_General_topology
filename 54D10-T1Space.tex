\documentclass[12pt]{article}
\usepackage{pmmeta}
\pmcanonicalname{T1Space}
\pmcreated{2013-03-22 12:18:14}
\pmmodified{2013-03-22 12:18:14}
\pmowner{drini}{3}
\pmmodifier{drini}{3}
\pmtitle{T1 space}
\pmrecord{10}{31852}
\pmprivacy{1}
\pmauthor{drini}{3}
\pmtype{Definition}
\pmcomment{trigger rebuild}
\pmclassification{msc}{54D10}
\pmsynonym{T1}{T1Space}
\pmrelated{T0Space}
\pmrelated{T2Space}
\pmrelated{T3Space}
\pmrelated{RegularSpace}
\pmrelated{ASpaceIsT1IfAndOnlyIfEverySubsetAIsTheIntersectionOfAllOpenSetsContainingA}
\pmrelated{SierpinskiSpace}
\pmrelated{PropertyThatCompactSetsInASpaceAreClosedLiesStrictlyBetweenT1AndT2}

\endmetadata

%\usepackage{graphicx}
%%%\usepackage{xypic} 
\usepackage{bbm}
\newcommand{\Z}{\mathbbmss{Z}}
\newcommand{\C}{\mathbbmss{C}}
\newcommand{\R}{\mathbbmss{R}}
\newcommand{\Q}{\mathbbmss{Q}}
\newcommand{\mathbb}[1]{\mathbbmss{#1}}
\begin{document}
A topological space $(X,\tau)$ is said to be $T_1$ (or said to hold the $T_1$ axiom) if for all distinct points $x,y\in X$ ($x\neq y$), there exists an open set $U\in\tau$ such that $x\in U$ and $y\notin U$.

A space being $T_1$ is equivalent to the following statements:
\begin{itemize}
\item For every $x\in X$, the set $\{x\}$ is closed.
\item Every subset of $X$ is equal to the intersection of all the open sets that contain it.
\item Distinct points are separated.
\end{itemize}
%%%%%
%%%%%
\end{document}
