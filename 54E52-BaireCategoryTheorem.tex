\documentclass[12pt]{article}
\usepackage{pmmeta}
\pmcanonicalname{BaireCategoryTheorem}
\pmcreated{2013-03-22 12:43:32}
\pmmodified{2013-03-22 12:43:32}
\pmowner{Koro}{127}
\pmmodifier{Koro}{127}
\pmtitle{Baire category theorem}
\pmrecord{13}{33024}
\pmprivacy{1}
\pmauthor{Koro}{127}
\pmtype{Theorem}
\pmcomment{trigger rebuild}
\pmclassification{msc}{54E52}
\pmrelated{SardsTheorem}
\pmrelated{Meager}
\pmrelated{Residual}

\endmetadata

% this is the default PlanetMath preamble.  as your knowledge
% of TeX increases, you will probably want to edit this, but
% it should be fine as is for beginners.

% almost certainly you want these
\usepackage{amssymb}
\usepackage{amsmath}
\usepackage{amsfonts}

% used for TeXing text within eps files
%\usepackage{psfrag}
% need this for including graphics (\includegraphics)
%\usepackage{graphicx}
% for neatly defining theorems and propositions
%\usepackage{amsthm}
% making logically defined graphics
%%%\usepackage{xypic}

% there are many more packages, add them here as you need them

% define commands here

\newcommand{\Prob}[2]{\mathbb{P}_{#1}\left\{#2\right\}}
\begin{document}
In a non-empty complete metric space, any countable intersection of dense, open subsets is non-empty.

In fact, such countable intersections of dense, open subsets are dense.  So the theorem holds also for any non-empty open subset of a complete metric space.

\textbf{Alternative formulations:}
Call a set \emph{first category}, or a \emph{meagre} set, if it is a countable union of nowhere dense sets, otherwise \emph{second category}. The Baire category theorem is often stated as ``no non-empty complete metric space is of first category'', or, trivially, as ``a non-empty, complete metric space is of second category''. In short, this theorem says that every nonempty complete metric space is a Baire space.

In functional analysis, this important property of complete metric spaces forms the {b}asis for the proofs of the important principles of Banach spaces: the open mapping theorem and the closed graph theorem.

It may also be taken as giving a concept of ``small sets'', similar to sets of measure zero: a countable union of these sets remains ``small''.  However, the real line $\mathbb{R}$ may be partitioned into a set of measure zero and a set of first category; the two concepts are distinct.

Note that, apart from the requirement that the set be a complete metric space, all conditions and conclusions of the theorem are phrased topologically.  This ``metric requirement'' is thus something of a disappointment.  As it turns out, there are two ways to reduce this requirement.

First, if a topological space $\mathcal{T}$ is homeomorphic to a non-empty open subset of a complete metric space, then we can transfer the Baire property through the homeomorphism, so in $\mathcal{T}$ too any countable intersection of open dense sets is non-empty (and, in fact, dense).  The other formulations also hold in this case.

Second, the Baire category theorem holds for a locally compact, Hausdorff\footnote{Some authors only define a locally compact space to be a Hausdorff space; that is the sense required for this theorem.} topological space  $\mathcal{T}$.
%%%%%
%%%%%
\end{document}
