\documentclass[12pt]{article}
\usepackage{pmmeta}
\pmcanonicalname{AxiomaticDefinitionOfTheRealNumbers}
\pmcreated{2013-03-22 15:39:29}
\pmmodified{2013-03-22 15:39:29}
\pmowner{matte}{1858}
\pmmodifier{matte}{1858}
\pmtitle{axiomatic definition of the real numbers}
\pmrecord{17}{37591}
\pmprivacy{1}
\pmauthor{matte}{1858}
\pmtype{Definition}
\pmcomment{trigger rebuild}
\pmclassification{msc}{54C30}
\pmclassification{msc}{26-00}
\pmclassification{msc}{12D99}
\pmrelated{RealNumber}

\endmetadata

% this is the default PlanetMath preamble.  as your knowledge
% of TeX increases, you will probably want to edit this, but
% it should be fine as is for beginners.

% almost certainly you want these
\usepackage{amssymb}
\usepackage{amsmath}
\usepackage{amsfonts}
\usepackage{amsthm}

\usepackage{mathrsfs}

% used for TeXing text within eps files
%\usepackage{psfrag}
% need this for including graphics (\includegraphics)
%\usepackage{graphicx}
% for neatly defining theorems and propositions
%
% making logically defined graphics
%%%\usepackage{xypic}

% there are many more packages, add them here as you need them

% define commands here

\newcommand{\sR}[0]{\mathbb{R}}
\newcommand{\sC}[0]{\mathbb{C}}
\newcommand{\sN}[0]{\mathbb{N}}
\newcommand{\sZ}[0]{\mathbb{Z}}

 \usepackage{bbm}
 \newcommand{\Z}{\mathbbmss{Z}}
 \newcommand{\C}{\mathbbmss{C}}
 \newcommand{\F}{\mathbbmss{F}}
 \newcommand{\N}{\mathbbmss{N}}
 \newcommand{\R}{\mathbbmss{R}}
 \newcommand{\Q}{\mathbbmss{Q}}



\newcommand*{\norm}[1]{\lVert #1 \rVert}
\newcommand*{\abs}[1]{| #1 |}



\newtheorem{thm}{Theorem}
\newtheorem{defn}{Definition}
\newtheorem{prop}{Proposition}
\newtheorem{lemma}{Lemma}
\newtheorem{cor}{Corollary}
\begin{document}
\subsubsection*{Axiomatic definition of the real numbers}
The real numbers consist of a set $\R$ together with mappings
$+\colon \R\times \R\to \R$ and $\cdot\colon \R\times \R \to\R$ and a
relation $< \,\subseteq \R\times\R$ satisfying
 the following conditions:
\begin{enumerate}
\item $(\R,+)$ is an Abelian group:
\begin{enumerate}
\item For $a,b,c\in \R$, we have
\begin{eqnarray*}
  a+b&=&b+a, \\
(a+b)+c&=&a+(b+c),
\end{eqnarray*}
\item there exists an element $0\in \R$ such that $a+0=a$ for all $a\in \R$,
\item every $a\in \R$ has an inverse $(-a)\in\R$ such that $a+(-a)=0$. 
\end{enumerate}
\item $(\R\setminus\{0\}, \cdot)$ is an Abelian group: 
\begin{enumerate}
\item For $a,b,c\in \R$, we have
\begin{eqnarray*}
  a\cdot b&=&b\cdot a, \\
(a\cdot b)\cdot c&=&a\cdot (b\cdot c),
\end{eqnarray*}
\item there exists an element $1\in \R\setminus\{0\}$ such that $a\cdot 1=a$ for all $a\in \R$,
\item every $a\in \R\setminus\{0\}$ has an inverse $a^{-1}\in\R$ 
      such that $a^{-1}\cdot a=1$. 
\end{enumerate}
\item The operation $\cdot$ is distributive over  $+$: If $a,b,c\in \R$, then 
\begin{eqnarray*}
  a\cdot (b+c)&=a\cdot b + a\cdot c, \\
   (b+c)\cdot a&=b\cdot a + c\cdot a.
\end{eqnarray*}
\item $(\R,<)$ is a total order:
\begin{enumerate}
\item (transitivity) if $c\in\R$, $a<b$, and $b<c$, then $a<c$,
\item (trichotomy) precisely one of the below alternatives
hold:
\[
  a<b, \quad a=b, \quad b<a.
\]
\end{enumerate}
For convenience we make the following notational definitions:
$a>b$ means $b<a$, $a\le b$ means either $a<b$ or $a=b$, and $a\ge b$
means either $b<a$ or $a=b$.

\item The operations $+$ and $\cdot$ are compatible with the order $<$:
\begin{enumerate}
\item If $a$, $b$, $c\in\R$ and $a<b$, then $a+c<b+c$. 
\item If $a$, $b$, $c\in\R$ with $a<b$ and $0<c$, then $ac<bc$.
\end{enumerate}

\item $\R$ has the least upper bound property: If $A\subset \R$,
then an element $M\in \R$ is an \PMlinkescapetext{upper bound} for $A$ if
\[
  a<M, \mbox{ for all }\ a\in A.
\]
If $A$ is non-empty, we then say that $A$ is bounded from above. 
That $\R$ has the least upper bound property means that
if $A\subset \R$ is bounded from above, it has a least upper bound $m\in \R$. That is, 
$A$ has an upper bound $m$ such that if $M$ is any upper bound from $M$, 
then $m\le M$. 
\end{enumerate}

Here it should be emphasized that from the above we can not deduce that 
a set $\R$ with operations $+,\cdot,<$ exists. To settle this question such 
a set has to be explicitly constructed. However, this can be done in various ways, as
discussed on \PMlinkname{this page}{RealNumber}.
One can also show the above conditions uniquely determine the real numbers 
(up to an isomorphism). The proof of this can be found on 
\PMlinkname{this page}{EveryOrderedFieldWithTheLeastUpperBoundPropertyIsIsomorphicToTheRealNumbers}.


\subsubsection*{Basic properties}
In condensed form, the above conditions state that $\R$ is an ordered 
field with the least upper bound property. In particular $(\R,+,\cdot)$ 
is a ring, and $(\R\setminus\{0\},\cdot)$ is a group, and we have the following basic properties:

\begin{lemma}Suppose $a,b\in \R$. 
\begin{enumerate} 
\item The additive inverse $(-a)$ is unique \PMlinkname{(proof)}{UniquenessOfAdditiveIdentityInARing}.
\item The additive identity $0$ is unique \PMlinkname{(proof)}{UniquenessOfAdditiveIdentityInARing2}.
\label{minusProperty}
\item $(-1)\cdot a=(-a)$ \PMlinkname{(proof)}{1cdotAA}.
\item $(-a)\cdot(-b)=a\cdot b$ \PMlinkname{(proof)}{XcdotYXcdotY}.
\item $0\cdot a=0$ \PMlinkname{(proof)}{0cdotA0}
\item The multiplicative inverse $a^{-1}$ is unique \PMlinkname{(proof)}{UniquenessOfInverseForGroups}.
\item If $a,b$ are non-zero, then $(ab)^{-1}=b^{-1} a^{-1}$ \PMlinkname{(proof)}{InverseOfAProduct}.
\end{enumerate}
\end{lemma}


In view of property \ref{minusProperty}, we can write simply $-a$ 
instead of $(-1)\cdot a$ and $(-a)$.  

Because of the additive inverse of a real number is unique (by property 1 above), and $(-a)+a=a+(-a)=0$, we see that the additive inverse of $-a$ is $a$, or that $-(-a)=a$.  Similarly, if $a\ne 0$, then $a^{-1}\ne 0$ (or we'll end up with $1=aa^{-1}=a0=0$), and therefore by Property 6 above, $a^{-1}$ has a unique multiplicative inverse.  Since $aa^{-1}=a^{-1}a=1$, we see that $a$ is the multiplicative inverse of $a^{-1}$.  In other words, $(a^{-1})^{-1}=a$.

For $a,b\in \R$ let us also define $a-b=a+(-b)$, which is called 
the \emph{difference} of $a$ and $b$.
By commutativity, $a-b=-b+a$.\, It is also common to leave out the
multiplication symbol and simply write $ab=a\cdot b$.\, Suppose\, 
$a\in \R$\, and\, $b\in \R$\, is non-zero.\, Then \emph{$b$
\PMlinkname{divided}{Division} by $a$} is defined as 
$$ 
   \frac{a}{b} = ab^{-1}.
$$
In consequence, if\, $a,\,b,\,c,\,d\in \R$\, and $b,\,c,\,d$ are non-zero, then 
\begin{itemize}
\item $ \frac{\frac{a}{b}}{\frac{c}{d}} = \frac{bd}{ac}$,
\item $\frac{ab}{b}=a$.
\end{itemize}
For example, 
$$
 \frac{\frac{a}{b}}{\frac{c}{d}} = \frac{ab^{-1}}{cd^{-1}}=
ab^{-1}(cd^{-1})^{-1}
= ab^{-1} d c^{-1}=
\frac{ad}{bc}.
$$
%%%%%
%%%%%
\end{document}
