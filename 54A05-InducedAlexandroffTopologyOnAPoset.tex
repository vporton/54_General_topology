\documentclass[12pt]{article}
\usepackage{pmmeta}
\pmcanonicalname{InducedAlexandroffTopologyOnAPoset}
\pmcreated{2013-03-22 18:46:01}
\pmmodified{2013-03-22 18:46:01}
\pmowner{joking}{16130}
\pmmodifier{joking}{16130}
\pmtitle{induced Alexandroff topology on a poset}
\pmrecord{4}{41548}
\pmprivacy{1}
\pmauthor{joking}{16130}
\pmtype{Derivation}
\pmcomment{trigger rebuild}
\pmclassification{msc}{54A05}

\endmetadata

% this is the default PlanetMath preamble.  as your knowledge
% of TeX increases, you will probably want to edit this, but
% it should be fine as is for beginners.

% almost certainly you want these
\usepackage{amssymb}
\usepackage{amsmath}
\usepackage{amsfonts}

% used for TeXing text within eps files
%\usepackage{psfrag}
% need this for including graphics (\includegraphics)
%\usepackage{graphicx}
% for neatly defining theorems and propositions
%\usepackage{amsthm}
% making logically defined graphics
%%%\usepackage{xypic}

% there are many more packages, add them here as you need them

% define commands here

\begin{document}
Let $(X,\leq)$ be a poset. For any $x\in X$ define following subset:
$$(-\infty,x]=\{y\in X\ |\ y\leq x\}.$$
\textit{The induced Alexandroff topology} $\tau$ on $X$ is defined as a topology generated by $\{(-\infty,x]\}_{x\in X}$.

\textbf{Proposition 1.} $(X,\tau)$ is a $\mathrm{T}_{0}$, Alexandroff space.

\textit{Proof.} Let $x,y\in X$ be such that $x\neq y$. Note that this implies that $x\not\leq y$ or $y\not\leq x$ (because $\leq$ is antisymmetric). Therefore $x\not\in (-\infty,y]$ or $y\not\in (-\infty,x]$. Thus $(X,\tau)$ is $\mathrm{T}_{0}$.

Now in order to show that $(X,\tau)$ is Alexandroff it is enough to show that an arbitrary intersection of base sets is open. So assume that $\{x_i\}_{i\in I}$ is a subset of $X$ such that 
$$A=\bigcap_{i\in I}\,(-\infty,x_i]\neq\emptyset$$
and let $y\in A$. Then (since $\leq$ is transitive) it is clear that 
$$(-\infty,y]\subseteq A$$
and thus
$$\bigcap_{i\in I}\,(-\infty,x_i]=A=\bigcup_{y\in A}\,(-\infty,y]$$
and therefore the intersection is open, which completes the proof. $\square$

\textbf{Proposition 2.} Let $X,Y$ be posets and $f:X\to Y$ a function. Then $f$ preserves order if and only if $f$ is continuous in induced Alexandroff topologies.

\textit{Proof.} ,,$\Rightarrow$'' Assume that $f$ preserves order and let $A=(-\infty,y]\subseteq Y$ be an open base set. We wish to show that $f^{-1}(A)$ is open in $X$. So take any $x\in f^{-1}(A)$. Now if $y\leq x$, then $f(y)\leq f(x)$ (since $f$ preserves order) and thus $f(y)\in A$. Therefore $y\in f^{-1}(A)$. Since $y$ was arbitrary we obtain that for any $x\in f^{-1}(A)$ we have $(-\infty, x]\subseteq f^{-1}(A)$ and thus 
$$f^{-1}(A)=\bigcup_{x\in f^{-1}(A)}\, (-\infty, x],$$
which implies that $f^{-1}(A)$ is open.

,,$\Leftarrow$'' Assume that $f$ is continuous and let $y\leq x$ for some $x,y\in X$. Assume that $f(y)\not\leq f(x)$. Let $A=(-\infty,f(x)]$. Therefore $f(y)\not\in A$, but $A$ is open, so $f^{-1}(A)$ is open (because $f$ is continuous). Thus $(-\infty,x]\subseteq f^{-1}(A)$. But $y\leq x$, so $y\in f^{-1}(A)$. But this implies that $f(y)\in A$. Contradiction. $\square$
%%%%%
%%%%%
\end{document}
