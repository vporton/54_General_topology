\documentclass[12pt]{article}
\usepackage{pmmeta}
\pmcanonicalname{ProofOfUrysohnsLemma}
\pmcreated{2013-03-22 13:09:23}
\pmmodified{2013-03-22 13:09:23}
\pmowner{scanez}{1021}
\pmmodifier{scanez}{1021}
\pmtitle{proof of Urysohn's lemma}
\pmrecord{4}{33597}
\pmprivacy{1}
\pmauthor{scanez}{1021}
\pmtype{Proof}
\pmcomment{trigger rebuild}
\pmclassification{msc}{54D15}

\endmetadata

% this is the default PlanetMath preamble.  as your knowledge
% of TeX increases, you will probably want to edit this, but
% it should be fine as is for beginners.

% almost certainly you want these
\usepackage{amssymb}
\usepackage{amsmath}
\usepackage{amsfonts}

% used for TeXing text within eps files
%\usepackage{psfrag}
% need this for including graphics (\includegraphics)
%\usepackage{graphicx}
% for neatly defining theorems and propositions
%\usepackage{amsthm}
% making logically defined graphics
%%%\usepackage{xypic}

% there are many more packages, add them here as you need them

% define commands here
\begin{document}
First we construct a family $U_p$ of open sets of $X$ indexed by the
rationals such that if $ p < q$, then $\bar{U_p} \subseteq U_q$. These
are the sets we will use to define our continuous function.

Let $P = \mathbb{Q} \cap [0,1]$. Since $P$ is countable, we can use
induction (or recursive definition if you prefer) to define the sets
$U_p$. List the elements of $P$ is an infinite sequence in some way;
let us assume that $1$ and $0$ are the first two elements of this
sequence. Now, define $U_1 = X \backslash D$ (the complement of $D$ in
$X$). Since $C$ is a closed set of $X$ contained in $U_1$, by
normality of $X$ we can choose an open set $U_0$ such that $C
\subseteq U_0$ and $\bar{U_0} \subseteq U_1$.

In general, let $P_n$ denote the set consisting of the first $n$
rationals in our sequence. Suppose that $U_p$ is defined for all $p
\in P_n$ and
\begin{equation}
	\textrm{if} \ p < q, \textrm{then} \ \bar{U_p} \subseteq U_q.
\end{equation}
Let $r$ be the next rational number in the sequence. Consider $P_{n+1}
= P_n \cup \{r\}$.  It is a finite subset of $[0,1]$ so it inherits
the usual ordering $<$ of $\mathbb{R}$. In such a set, every element
(other than the smallest or largest) has an immediate predecessor and
successor. We know that $0$ is the smallest element and $1$ the
largest of $P_{n+1}$ so $r$ cannot be either of these.  Thus $r$ has
an immediate predecessor $p$ and an immediate successor $q$ in
$P_{n+1}$. The sets $U_p$ and $U_q$ are already defined by the
inductive hypothesis so using the normality of $X$, there exists an
open set $U_r$ of $X$ such that
\begin{displaymath}
	\bar{U_p} \subseteq U_r \ \textrm{and} \ \bar{U_r} \subseteq U_q.
\end{displaymath}We now show that (1) holds for every pair of elements in $P_{n+1}$. If
both elements are in $P_n$, then (1) is true by the inductive
hypothesis. If one is $r$ and the other $s \in P_n$, then if $s \le p$
we have
\begin{displaymath}
	\bar{U_s} \subseteq \bar{U_p} \subseteq U_r
\end{displaymath}
and if $s \ge q$ we have
\begin{displaymath}
	\bar{U_r} \subseteq U_q \subseteq U_s.
\end{displaymath}
Thus (1) holds for ever pair of elements in $P_{n+1}$ and therefore by
induction, $U_p$ is defined for all $p \in P$.

We have defined $U_p$ for all rationals in $[0,1]$. Extend this
definition to every rational $p \in \mathbb{R}$ by defining
\begin{displaymath}
	\begin{array}{ll}
		U_p = \emptyset & \textrm{if} \ p < 0 \\
		U_p = X & \textrm{if} \ p > 1.
	\end{array}
\end{displaymath}
Then it is easy to check that (1) still holds.

Now, given $x \in X$, define $\mathbb{Q}(x) = \{p : x \in U_p\}$. This
set contains no number less than $0$ and contains every number greater
than $1$ by the definition of $U_p$ for $p < 0$ and $p > 1$. Thus
$\mathbb{Q}(x)$ is bounded below and its infimum is an element in
$[0,1]$. Define
\begin{displaymath}
	f(x) = \textrm{inf} \ \mathbb{Q}(x).
\end{displaymath}

Finally we show that this function $f$ we have defined satisfies the
conditions of lemma.  If $x \in C$, then $x \in U_p$ for all $p \ge 0$
so $\mathbb{Q}(x)$ equals the set of all nonnegative rationals and
$f(x) = 0$. If $x \in D$, then $x \notin U_p$ for $p \le 1$ so
$\mathbb{Q}(x)$ equals all the rationals greater than 1 and $f(x) =
1$. 

To show that $f$ is continuous, we first prove two smaller results:

\indent (a) $x \in \bar{U_r} \Rightarrow f(x) \le r$

\emph{Proof}. If $x \in \bar{U_r}$, then $x \in U_s$ for all $s > r$
so $\mathbb{Q}(x)$ contains all rationals greater than $r$. Thus $f(x)
\le r$ by definition of $f$.

\indent (b) $x \notin U_r \Rightarrow f(x) \ge r$.

\emph{Proof}. If $x \notin U_r$, then $x \notin U_s$ for all $s < r$
so $\mathbb{Q}(x)$ contains no rational less than $r$. Thus $f(x) \ge
r$.

Let $x_0 \in X$ and let $(c,d)$ be an open interval of $\mathbb{R}$
containing $f(x)$.  We will find a neighborhood $U$ of $x_0$ such that
$f(U) \subseteq (c,d)$. Choose $p,q \in \mathbb{Q}$ such that
\begin{displaymath}
	c < p < f(x_0) < q < d.
\end{displaymath}
Let $U = U_q \backslash \bar{U_p}$. Then since $f(x_0) < q$, (b)
implies that $x \in U_q$ and since $f(x_0) > p$, (a) implies that $x_0
\notin \bar{U_p}$. Hence $x_0 \in U$.

Finally, let $x \in U$. Then $x \in U_q \subseteq \bar{U_q}$, so $f(x)
\le q$ by (a).  Also, $x \notin \bar{U_p}$ so $x \notin U_p$ and $f(x)
\ge p$ by (b). Thus
\begin{displaymath}
	f(x) \in [p,q] \subseteq (c,d)
\end{displaymath}
as desired. Therefore $f$ is continuous and we are done.
%%%%%
%%%%%
\end{document}
