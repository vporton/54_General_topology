\documentclass[12pt]{article}
\usepackage{pmmeta}
\pmcanonicalname{SectionFilter}
\pmcreated{2013-03-22 16:41:37}
\pmmodified{2013-03-22 16:41:37}
\pmowner{CWoo}{3771}
\pmmodifier{CWoo}{3771}
\pmtitle{section filter}
\pmrecord{4}{38906}
\pmprivacy{1}
\pmauthor{CWoo}{3771}
\pmtype{Definition}
\pmcomment{trigger rebuild}
\pmclassification{msc}{54A99}
\pmclassification{msc}{03E99}

\usepackage{amssymb,amscd}
\usepackage{amsmath}
\usepackage{amsfonts}

% used for TeXing text within eps files
%\usepackage{psfrag}
% need this for including graphics (\includegraphics)
%\usepackage{graphicx}
% for neatly defining theorems and propositions
\usepackage{amsthm}
% making logically defined graphics
%%\usepackage{xypic}
\usepackage{pst-plot}
\usepackage{psfrag}

% define commands here

\begin{document}
Let $X$ be a set and $(x_i)_{i\in D}$ a non-empty net in $X$.  For each $j\in D$, define $S(j):=\lbrace x_i\mid i\le j\rbrace$.  Then the set $$S:=\lbrace S(j)\mid j\in D\rbrace$$ is a filter basis: $S$ is non-empty because $(x_i)\neq \varnothing$, and for any $j,k\in D$, there is a $\ell$ such that $j\le \ell$ and $k\le \ell$, so that $S(\ell) \subseteq S(j)\cap S(k)$.

Let $\mathcal{A}$ be the family of all filters containing $S$.  $\mathcal{A}$ is non-empty since the filter generated by $S$ is in $\mathcal{A}$.  Order $\mathcal{A}$ by inclusion so that $\mathcal{A}$ is a poset.  Any chain $\mathcal{F}_1\subseteq \mathcal{F}_2\subseteq\cdots $ has an upper bound, namely, $$\mathcal{F}:=\bigcup_{i=1}^{\infty} \mathcal{F}_i.$$  By Zorn's lemma, $\mathcal{A}$ has a maximal element $\mathcal{X}$.  

\textbf{Definition}.  $\mathcal{X}$ defined above is called the \emph{section filter} of the net $(x_i)$ in $X$.

\textbf{Remark}.  A section filter is obviously a filter.  The name ``section'' comes from the elements $S(j)$ of $S$, which are sometimes known as ``sections'' of the net $(x_i)$.

%%%%%
%%%%%
\end{document}
