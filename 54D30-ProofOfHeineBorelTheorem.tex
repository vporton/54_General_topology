\documentclass[12pt]{article}
\usepackage{pmmeta}
\pmcanonicalname{ProofOfHeineBorelTheorem}
\pmcreated{2013-03-22 12:57:40}
\pmmodified{2013-03-22 12:57:40}
\pmowner{stevecheng}{10074}
\pmmodifier{stevecheng}{10074}
\pmtitle{proof of Heine-Borel theorem}
\pmrecord{23}{33328}
\pmprivacy{1}
\pmauthor{stevecheng}{10074}
\pmtype{Proof}
\pmcomment{trigger rebuild}
\pmclassification{msc}{54D30}

\endmetadata

% this is the default PlanetMath preamble.  as your knowledge
% of TeX increases, you will probably want to edit this, but
% it should be fine as is for beginners.

% almost certainly you want these
\usepackage{amssymb}
\usepackage{amsmath}
\usepackage{amsfonts}

% used for TeXing text within eps files
%\usepackage{psfrag}
% need this for including graphics (\includegraphics)
%\usepackage{graphicx}
% for neatly defining theorems and propositions
\usepackage{amsthm}
% making logically defined graphics
%%%\usepackage{xypic}

% there are many more packages, add them here as you need them
\usepackage{enumerate}

% define commands here
\newcommand{\real}{\mathbb{R}}
\newcommand{\rat}{\mathbb{Q}}
\newcommand{\nat}{\mathbb{N}}

\providecommand{\abs}[1]{\lvert#1\rvert}
\providecommand{\absW}[1]{\left\lvert#1\right\rvert}
\providecommand{\absB}[1]{\Bigl\lvert#1\Bigr\rvert}
\providecommand{\norm}[1]{\lVert#1\rVert}
\providecommand{\normW}[1]{\left\lVert#1\right\rVert}
\providecommand{\normB}[1]{\Bigl\lVert#1\Bigr\rVert}
\providecommand{\defnterm}[1]{\emph{#1}}
\begin{document}
To shorten the proofs, we will be using certain concepts and facts from general 
point-set topology, such as the very first assertion below.
Even if you are not familiar with them, 
these assertions are all easily proven from the definitions (left as an exercise). 

\begin{proof}[Compact implies closed and bounded]
Since $\real^n$ is a Hausdorff space, any compact subset $A$ has to be closed as well.
That $A$ is also bounded is very easily seen by taking
the open cover $\{ (-k, k)^n \}_{k \in \nat}$.
Passing to a finite cover, we see that $A$ is contained
in a bounded set.
\end{proof}

\begin{proof}[Reduction for closed and bounded implies compact]
To prove that if $A$ is closed and bounded implies it is compact,
it is only necessary to prove that the rectangle $[a,b]^n$ is compact.
For general $A$, since $A$ is bounded, it is contained in some such compact 
rectangle $[a, b]^n$.
Since $A$ is closed and contained in a compact set, $A$ is also compact.
\end{proof}

\begin{proof}[The case $n=1$: the closed interval is compact.]
Let $\mathcal{C}$ be an arbitrary cover of $[a, b]$ by open sets in $\real^1$.
Define 
\[
 S = \{ x \in [a,b] : \textrm{some finite subcollection of $\mathcal{C}$ covers $[a, x]$} \}\,.
\]
Set $t = \sup S$; then $a \leq t \leq b < \infty$.

We first note that the supremum $t$ is attained:
that is, there is a finite subcollection of $\mathcal{C}$ that covers $[a,t]$.
Obviously we can choose one open set $U$ from $\mathcal{C}$ that covers the point $t$.
This $U$ must also cover the open interval $(t-\epsilon, t+\epsilon)$ for some $\epsilon > 0$.
But by the definition of $t$, the interval $[a, t-\epsilon]$ has a finite subcover.
When this finite subcover is put together with $U$, we obtain a finite subcover for $[a, t]$.

It is easy to see that $t$ cannot be less than $b$.
For the same open set $U$ above covers the interval $(t, t+\epsilon)$, 
so we have a finite subcover for $[a, t+\epsilon/2]$.
If $t$ is the supremum then it has go to be at the right endpoint of the interval $[a, b]$, i.e.
$t = b$.
\end{proof}

For the case $n > 1$, apply 
\PMlinkname{Tychonoff's Theorem, in the finite case}{ProofOfTychonoffsTheoremInFiniteCase}, which states:
\begin{quote}
If $X_1, \dotsc, X_n$ are each compact,
then $X_1 \times \dotsm \times X_n$ in the product topology is compact.
\end{quote}
Here we set $X_i$ to be the closed intervals; then the general closed rectangle in $\real^n$
is compact.  It is not hard to see that the product topology for the closed rectangle is the same
as its subspace topology in the product topology $\real^n = \real \times \dotsm \times \real$.
And it is again a standard exercise to show that the product topology on $\real^n$ is the 
same as the norm topology on $\real^n$ as a vector space.

This completes the proof of the Heine-Borel theorem.

\section*{Proof by a bisection argument}
There is another proof of the Heine-Borel theorem for $\real^n$ without
resorting to Tychonoff's Theorem.  It goes by bisecting the rectangle along each of its sides.
At the first stage, we divide up the rectangle $A$ into $2^n$ subrectangles.
Suppose the open cover $\mathcal{C}$ of $A$ has no finite subcover.
Then one of the subrectangles --- call it $A_1$ ---
must have no finite subcover by $\mathcal{C}$.  We can subdivide $A_1$ into $2^n$ pieces; since $A_1$
has no finite subcover, one of the new subrectangles of $A_1$ also has no finite subcover.  And continue
dividing to get a nested sequence of rectangles $A \supset A_1 \supset A_2 \supset \dotsb$
whose side lengths approach zero, and possessing no finite subcover.

By the nested interval theorem, the ``limit rectangle'' 
$\bigcap_{i=1}^\infty A_i$ must consist of a sole point $x$,
and this obviously has a finite subcover by an open set $U \in \mathcal{C}$.
But $U$ must contain a small rectangle with centre $x$, which 
for $i$ large enough, contradicts $A_i$ having no finite subcover.

Of course, in both proofs of the Heine-Borel theorem, the completeness of the reals (the least upper bound property)
enters in an essential way.
%%%%%
%%%%%
\end{document}
