\documentclass[12pt]{article}
\usepackage{pmmeta}
\pmcanonicalname{StonevCechCompactification}
\pmcreated{2013-03-22 14:37:38}
\pmmodified{2013-03-22 14:37:38}
\pmowner{rspuzio}{6075}
\pmmodifier{rspuzio}{6075}
\pmtitle{Stone-\v{C}ech compactification}
\pmrecord{10}{36207}
\pmprivacy{1}
\pmauthor{rspuzio}{6075}
\pmtype{Definition}
\pmcomment{trigger rebuild}
\pmclassification{msc}{54D30}

% this is the default PlanetMath preamble.  as your knowledge
% of TeX increases, you will probably want to edit this, but
% it should be fine as is for beginners.

% almost certainly you want these
\usepackage{amssymb}
\usepackage{amsmath}
\usepackage{amsfonts}

% used for TeXing text within eps files
%\usepackage{psfrag}
% need this for including graphics (\includegraphics)
%\usepackage{graphicx}
% for neatly defining theorems and propositions
%\usepackage{amsthm}
% making logically defined graphics
%%%\usepackage{xypic}

% there are many more packages, add them here as you need them

% define commands here
\begin{document}
Stone-\v Cech compactification is a technique for embedding a Tychonoff topological space in a compact Hausdorff space.

Let $X$ be a Tychonoff space and let $C$ be the space of all continuous functions from $X$ to the closed interval $[0,1]$.  To each element $x \in X$, we may associate the evaluation functional $e_x \colon C \to [0,1]$ defined by $e_x (f) = f(x)$.  In this way, $X$ may be identified with a set of functionals.

The space $[0,1]^C$ of \emph{all} functionals from $C$ to $[0,1]$ may be endowed with the Tychonoff product topology.  Tychonoff's theorem asserts that, in this topology, $[0,1]^C$ is a compact Hausdorff space.  The closure in this topology of the subset of $[0,1]^C$ which was identified with $X$ via evaluation functionals is $\beta X$, the Stone-\v Cech compactification of $X$.
Being a closed subset of a compact Hausdorff space, $\beta X$ is itself a compact Hausdorff space.

This construction has the wonderful property that, for any compact Hausdorff space $Y$, every continuous function $f \colon X \to Y$ may be extended to a \emph{unique} continuous function $\beta f \colon \beta X \to Y$.
%%%%%
%%%%%
\end{document}
