\documentclass[12pt]{article}
\usepackage{pmmeta}
\pmcanonicalname{MobiusStrip}
\pmcreated{2013-03-22 12:55:28}
\pmmodified{2013-03-22 12:55:28}
\pmowner{Mathprof}{13753}
\pmmodifier{Mathprof}{13753}
\pmtitle{M\"obius strip}
\pmrecord{22}{33278}
\pmprivacy{1}
\pmauthor{Mathprof}{13753}
\pmtype{Definition}
\pmcomment{trigger rebuild}
\pmclassification{msc}{54B15}
\pmsynonym{M\"obius band}{MobiusStrip}
\pmrelated{KleinBottle}
\pmrelated{Torus}

\endmetadata

% this is the default PlanetMath preamble.  as your knowledge
% of TeX increases, you will probably want to edit this, but
% it should be fine as is for beginners.

% almost certainly you want these
\usepackage{amssymb}
\usepackage{amsmath}
\usepackage{amsfonts}

% used for TeXing text within eps files
%\usepackage{psfrag}
% need this for including graphics (\includegraphics)
\usepackage{graphicx}
% for neatly defining theorems and propositions
%\usepackage{amsthm}
% making logically defined graphics
%%%\usepackage{xypic} 

% there are many more packages, add them here as you need them

% define commands here
\begin{document}
A \emph{M\"{o}bius strip} is a non-orientiable 2-dimensional surface with a 1-dimensional boundary. It can be embedded in $\mathbb{R}^3$, but only has a single \PMlinkescapetext{side}.

We can parameterize the M\"{o}bius strip by
\[
x = r \cdot \cos{\theta}, \quad y = r \cdot \sin{\theta}, \quad z =
(r-2)\tan{\frac{\theta}{2}}.
\]
The M\"{o}bius strip is therefore a subset of the solid torus.

Topologically, the M\"{o}bius strip  is formed by taking a quotient space of $I^2 = [0,1] \times [0,1] \subset \mathbb{R}^2$. We do this by first letting $M$ be the partition of $I^2$ formed by the equivalence relation:
$$(1,x) \sim (0,1-x)\quad \mbox{where} \quad 0 \leq x \leq 1,$$ and every other point in $I^2$ is only related to itself.

By giving $M$ the quotient topology given by the quotient map $p: I^2 \to  M$ we obtain the M\"{o}bius strip.

Schematically we can represent this identification as follows:

\begin{center}
\includegraphics[scale=0.5]{mobius-2} \\
\tiny{Diagram 1: The identifications made on $I^2$ to make a M\"{o}bius strip. \\ We identify two opposite sides but with different orientations.}
\end{center}

Since the M\"{o}bius strip is homotopy equivalent to a circle, it has $\mathbb{Z}$\ as its fundamental group. It is not however, homeomorphic to the circle, although its boundary is.
%%%%%
%%%%%
\end{document}
