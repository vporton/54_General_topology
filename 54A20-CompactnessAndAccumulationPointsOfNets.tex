\documentclass[12pt]{article}
\usepackage{pmmeta}
\pmcanonicalname{CompactnessAndAccumulationPointsOfNets}
\pmcreated{2013-03-22 18:37:50}
\pmmodified{2013-03-22 18:37:50}
\pmowner{azdbacks4234}{14155}
\pmmodifier{azdbacks4234}{14155}
\pmtitle{compactness and accumulation points of nets}
\pmrecord{7}{41368}
\pmprivacy{1}
\pmauthor{azdbacks4234}{14155}
\pmtype{Theorem}
\pmcomment{trigger rebuild}
\pmclassification{msc}{54A20}
%\pmkeywords{net}
%\pmkeywords{compact}
%\pmkeywords{subnet}
%\pmkeywords{accumulation point}
%\pmkeywords{cluster point}
\pmrelated{Net}
\pmrelated{Compact}
\pmrelated{AccumulationPointsAndConvergentSubnets}

\endmetadata

%packages
\usepackage{amsmath,mathrsfs,amsfonts,amsthm}
%theorem environments
\theoremstyle{plain}
\newtheorem*{thm*}{Theorem}
\newtheorem*{lem*}{Lemma}
\newtheorem*{cor*}{Corollary}
\newtheorem*{prop*}{Proposition}
%delimiters
\newcommand{\set}[1]{\{#1\}}
\newcommand{\medset}[1]{\big\{#1\big\}}
\newcommand{\bigset}[1]{\bigg\{#1\bigg\}}
\newcommand{\Bigset}[1]{\Bigg\{#1\Bigg\}}
\newcommand{\abs}[1]{\vert#1\vert}
\newcommand{\medabs}[1]{\big\vert#1\big\vert}
\newcommand{\bigabs}[1]{\bigg\vert#1\bigg\vert}
\newcommand{\Bigabs}[1]{\Bigg\vert#1\Bigg\vert}
\newcommand{\norm}[1]{\Vert#1\Vert}
\newcommand{\mednorm}[1]{\big\Vert#1\big\Vert}
\newcommand{\bignorm}[1]{\bigg\Vert#1\bigg\Vert}
\newcommand{\Bignorm}[1]{\Bigg\Vert#1\Bigg\Vert}
\newcommand{\vbrack}[1]{\langle#1\rangle}
\newcommand{\medvbrack}[1]{\big\langle#1\big\rangle}
\newcommand{\bigvbrack}[1]{\bigg\langle#1\bigg\rangle}
\newcommand{\Bigvbrack}[1]{\Bigg\langle#1\Bigg\rangle}
\newcommand{\sbrack}[1]{[#1]}
\newcommand{\medsbrack}[1]{\big[#1\big]}
\newcommand{\bigsbrack}[1]{\bigg[#1\bigg]}
\newcommand{\Bigsbrack}[1]{\Bigg[#1\Bigg]}
%operators
\DeclareMathOperator{\Hom}{Hom}
\DeclareMathOperator{\Tor}{Tor}
\DeclareMathOperator{\Ext}{Ext}
\DeclareMathOperator{\Aut}{Aut}
\DeclareMathOperator{\End}{End}
\DeclareMathOperator{\Inn}{Inn}
\DeclareMathOperator{\lcm}{lcm}
\DeclareMathOperator{\ord}{ord}
\DeclareMathOperator{\rank}{rank}
\DeclareMathOperator{\tr}{tr}
\DeclareMathOperator{\Mat}{Mat}
\DeclareMathOperator{\Gal}{Gal}
\DeclareMathOperator{\GL}{GL}
\DeclareMathOperator{\SL}{SL}
\DeclareMathOperator{\SO}{SO}
\DeclareMathOperator{\ann}{ann}
\DeclareMathOperator{\im}{im}
\DeclareMathOperator{\Char}{char}
\DeclareMathOperator{\Spec}{Spec}
\DeclareMathOperator{\supp}{supp}
\DeclareMathOperator{\diam}{diam}
\DeclareMathOperator{\Ind}{Ind}
\DeclareMathOperator{\vol}{vol}

\begin{document}
\begin{thm*}
A topological space $X$ is compact if and only if every net in $X$ has an accumulation point.
\end{thm*}
\begin{proof}
Suppose $X$ is compact and let $(x_\alpha)_{\alpha\in A}$ be a net in $X$. For each $\alpha\in A$, put $E_\alpha=\set{x_\beta:\beta\geq\alpha}$; the collection $\set{\overline{E_\alpha}:\alpha\in A}$ of closed subsets of $X$ has the finite intersection property, for given $\alpha_1,\ldots,\alpha_n\in A$, because $A$ is directed, there exists $\beta\in A$ satisfying $\beta\geq\alpha_i$ for each $i\in\set{1,\ldots,n}$, so that $x_\beta\in\bigcap_{i=1}^n\overline{E_{\alpha_i}}$. Therefore, by compactness, $\bigcap_{\alpha\in A}\overline{E_\alpha}\neq\emptyset$; let $x$ be a point of this intersection. If $U$ is any open subset of $X$ and $\alpha\in A$, then because $x\in\overline{E_\alpha}$, $E_\alpha\cap U\neq\emptyset$, and thus there exists $\beta\geq\alpha\in A$ for which $x_\beta\in U$. It follows that $x$ is an accumulation point of $(x_\alpha)$. For the converse, assume that $X$ fails to be compact, and let $\set{U_i:i\in I}$ be an open cover of $X$ with no finite subcover. If $B$ is the set of finite subsets of $I$, then $B$ is directed by inclusion. For each set $S\in B$, let $x_S$ be a point in the complement of $\bigcup_{i\in S}U_i$. We contend that the net $(x_S)_{S\in B}$ has no accumulation points; indeed, given $x\in X$, we may select $i_0\in I$ such that $x\in U_{i_0}$; if $S\in B$ is such that $i_0\in S$, that is, if $S\geq\set{i_0}$, then by construction, $x_S\notin U_{i_0}$, establishing our contention. 
\end{proof}
\begin{cor*}
The following conditions on a topological space $X$ are equivalent:
\begin{enumerate}
\item $X$ is compact;
\item every net in $X$ has an accumulation point;
\item every net in $X$ has a convergent subnet;
\end{enumerate}
\end{cor*}
\begin{proof}
The preceding theorem establishes the equivalence of (1) and (2), while that of (2) and (3) is established in the entry on accumulation points and convergent subnets.
\end{proof}
%%%%%
%%%%%
\end{document}
