\documentclass[12pt]{article}
\usepackage{pmmeta}
\pmcanonicalname{SphericalMetric}
\pmcreated{2013-03-22 14:18:41}
\pmmodified{2013-03-22 14:18:41}
\pmowner{jirka}{4157}
\pmmodifier{jirka}{4157}
\pmtitle{spherical metric}
\pmrecord{6}{35775}
\pmprivacy{1}
\pmauthor{jirka}{4157}
\pmtype{Definition}
\pmcomment{trigger rebuild}
\pmclassification{msc}{54-00}
\pmclassification{msc}{30A99}
\pmdefines{spherical length}

% this is the default PlanetMath preamble.  as your knowledge
% of TeX increases, you will probably want to edit this, but
% it should be fine as is for beginners.

% almost certainly you want these
\usepackage{amssymb}
\usepackage{amsmath}
\usepackage{amsfonts}

% used for TeXing text within eps files
%\usepackage{psfrag}
% need this for including graphics (\includegraphics)
%\usepackage{graphicx}
% for neatly defining theorems and propositions
\usepackage{amsthm}
% making logically defined graphics
%%%\usepackage{xypic}

% there are many more packages, add them here as you need them

% define commands here
\theoremstyle{theorem}
\newtheorem*{thm}{Theorem}
\newtheorem*{lemma}{Lemma}
\newtheorem*{conj}{Conjecture}
\newtheorem*{cor}{Corollary}
\newtheorem*{example}{Example}
\newtheorem*{prop}{Proposition}
\theoremstyle{definition}
\newtheorem*{defn}{Definition}
\theoremstyle{remark}
\newtheorem*{rmk}{Remark}
\begin{document}
Suppose that $\hat{\mathbb{C}} := {\mathbb{C}} \cup \{ \infty \}$ is the extended complex plane (the Riemann sphere).

\begin{defn}
Suppose $\gamma \colon [0,1] \to \hat{\mathbb{C}}$ is a path in $\hat{\mathbb{C}}$.
The {\em spherical length} of $\gamma$ is defined as
\begin{equation*}
\ell (\gamma) :=
2 \int_\gamma \frac{\lvert dz \rvert}{1+\lvert z \rvert^2}
=
2 \int_0^1 \frac{\lvert \gamma'(t) \rvert}{1+\lvert \gamma(t) \rvert^2} dt.
\end{equation*}
\end{defn}

\begin{defn}
Let $z_1, z_2 \in \hat{\mathbb{C}}$, and let $\Gamma$ be the set of all paths
in $\hat{\mathbb{C}}$ from $z_1$ to $z_2$, then the distance from
$z_1$ to $z_2$ in the {\em spherical metric} is defined as
\begin{equation*}
\sigma(z_1,z_2) := \inf_{\gamma \in \Gamma} \ell(\gamma) .
\end{equation*} 
\end{defn}

More intuitivelly this is the shortest distance to travel from $z_1$ to
$z_2$ if we think of these points as being on the Riemann sphere, and we can
only travel on the Riemann sphere itself (we cannot ``drill'' a straight line
from $z_1$ to $z_2$).

\begin{thebibliography}{9}
\bibitem{Gamelin:complex}
Theodore~B.\@ Gamelin.
{\em \PMlinkescapetext{Complex Analysis}}.
Springer-Verlag, New York, New York, 2001.
\end{thebibliography}
%%%%%
%%%%%
\end{document}
