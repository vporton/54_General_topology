\documentclass[12pt]{article}
\usepackage{pmmeta}
\pmcanonicalname{FundamentalSystemOfEntourages}
\pmcreated{2013-03-22 16:29:55}
\pmmodified{2013-03-22 16:29:55}
\pmowner{mps}{409}
\pmmodifier{mps}{409}
\pmtitle{fundamental system of entourages}
\pmrecord{5}{38674}
\pmprivacy{1}
\pmauthor{mps}{409}
\pmtype{Definition}
\pmcomment{trigger rebuild}
\pmclassification{msc}{54E15}
\pmdefines{uniformity generated by}

% this is the default PlanetMath preamble.  as your knowledge
% of TeX increases, you will probably want to edit this, but
% it should be fine as is for beginners.

% almost certainly you want these
\usepackage{amssymb}
\usepackage{amsmath}
\usepackage{amsfonts}

% used for TeXing text within eps files
%\usepackage{psfrag}
% need this for including graphics (\includegraphics)
%\usepackage{graphicx}
% for neatly defining theorems and propositions
\usepackage{amsthm}
% making logically defined graphics
%%%\usepackage{xypic}

% there are many more packages, add them here as you need them

% define commands here
\newcommand{\UU}{\mathcal{U}}
\newcommand{\VV}{\mathcal{V}}
\newcommand{\BB}{\mathcal{B}}
\newtheorem*{lemma*}{Lemma}
\newtheorem{theorem}{Theorem}
\begin{document}
\PMlinkescapeword{similar}
\PMlinkescapeword{inverses}
\PMlinkescapeword{inverse}
\PMlinkescapeword{property}
\PMlinkescapeword{axioms}
\PMlinkescapeword{equivalent}

Let $(X,\UU)$ be a uniform space.  A subset $\BB\subseteq\UU$ is a \emph{fundamental system of entourages} for $\UU$ provided that each entourage in $\UU$ contains an element of $\BB$.

To see that each uniform space $(X,\UU)$ has a fundamental system of entourages, define
\[
\BB = \{ U\cap U^{-1} \colon U \in \UU \},
\]
where $U^{-1}$ denotes the inverse relation of $U$.  Since $\UU$ is closed under taking relational inverses and binary intersections, $\BB\subseteq\UU$.  By construction, each $U\in\UU$ contains the element of $U\cap U^{-1}\in\BB$.

There is a useful equivalent condition for being a fundamental system of entourages.  Let $\BB$ be a nonempty family of subsets of $X\times X$.  Then $\BB$ is a fundamental system of entourages of a uniformity on $X$ if and only if it \PMlinkescapetext{satisfies} the following axioms.
\begin{itemize}
\item
(B1) If $S$, $T\in\BB$, then $S\cap T$ contains an element of $\BB$.

\item
(B2) Each element of $\BB$ contains the diagonal $\Delta(X)$.

\item
(B3) For any $S\in\BB$, the inverse relation of $S$ contains an element of $\BB$.

\item
(B4) For any $S\in\BB$, there is an element $T\in\BB$ such that the relational composition $T\circ T$ is contained in $S$.
\end{itemize}

Suppose $\BB$ is a fundamental system of entourages for uniformities $\UU$ and $\VV$.  Then $\UU\subset\VV$.  To see this, suppose $S\in\UU$.  Since $\BB$ is a fundamental system of entourages for $\UU$, there is some element $B\in\BB$ such that $B\subset S$.  But $\BB\subset\VV$, so $B\in\VV$.  Hence by applying the fact that $\VV$ is closed under taking supersets we may conclude that $S\in\VV$.  So if $\BB$ is a fundamental system of entourages, it is a fundamental system for a unique uniformity $\UU$.  Thus it makes sense to call $\UU$ the \emph{uniformity generated by the fundamental system} $\BB$.

% ... more to come ...

\begin{thebibliography}{9}
\bibitem{TG}
Nicolas Bourbaki, {\it Elements of Mathematics: General Topology: Part 1}, Hermann, 1966.
\end{thebibliography}
%%%%%
%%%%%
\end{document}
