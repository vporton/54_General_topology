\documentclass[12pt]{article}
\usepackage{pmmeta}
\pmcanonicalname{Yhomeomorphism}
\pmcreated{2013-03-22 15:42:26}
\pmmodified{2013-03-22 15:42:26}
\pmowner{juanman}{12619}
\pmmodifier{juanman}{12619}
\pmtitle{y-homeomorphism}
\pmrecord{8}{37652}
\pmprivacy{1}
\pmauthor{juanman}{12619}
\pmtype{Definition}
\pmcomment{trigger rebuild}
\pmclassification{msc}{54C10}
\pmsynonym{crosscap slide}{Yhomeomorphism}
%\pmkeywords{homeomorphism}
\pmrelated{CrosscapSlide}

% this is the default PlanetMath preamble.  as your knowledge
% of TeX increases, you will probably want to edit this, but
% it should be fine as is for beginners.

% almost certainly you want these
\usepackage{amssymb}
\usepackage{amsmath}
\usepackage{amsfonts}

% used for TeXing text within eps files
%\usepackage{psfrag}
% need this for including graphics (\includegraphics)
%\usepackage{graphicx}
% for neatly defining theorems and propositions
%\usepackage{amsthm}
% making logically defined graphics
%%%\usepackage{xypic}

% there are many more packages, add them here as you need them

% define commands here
\begin{document}
The {\bf y-homeomorphism} also dubbed {\bf crosscap slide}, is an auto-homeomorphism (or self-homeomorphism) which can be defined only for 
{\bf non orientable surfaces} whose genus is greater than one. 

To define it we take a punctured Klein bottle $K_0=K\setminus{\rm int}\ D^2$ which can be consider as a pair of closed M\"obius bands $M_1,M_2$, one sewed in the other by perforating with a disk (being disjoint from $\partial M_1$) and then identify the boundary of the second with the boundary of that disk, in symbols:
$$ K_0=(M_1\setminus{\rm int}\ D^2)\cup_{\partial}M_2$$
where $\partial =\partial D^2=\partial M_2$. Other way to visualizing that, is by consider $K_0$ as the connected sum of ${\rm int}\ M_1$ with a 
projective plane ${\mathbb{R}}P^2$.

Now, thinking that the removed disk $D^2$ was located with its center at some point in the core of $M_1$, the second band, $M_2$ will have a pair  of points on that  part of the core in common with $\partial M_2$.

So, the  y-homeomorphism is defined by a isotopy  leaving the boundary $\partial M_1$ fixed by sliding the second band  $M_2$ one turn around the 
core of $M_1$ till the original position. The result is an automorphism of $K_0$ which maps $M_2$ into itself but reversing it.

To define this for genus greater than two just consider any other non orientable surface as a connected sum of a Kein bottle plus projective planes.

\begin{enumerate}
\item D.R.J. Chillingworth. {\it A finite set of generators for the 
ho\-meo\-to\-py group of a non-orientable surface}, Proc. Camb. Phil. Soc. 
65(1969), 409-430.

\item M. Korkmaz. {\it Mapping Class Groups of Non-orientable Surfaces}, Geometriae Dedicata 89 (2002), 109-133.

\item W.B.R. Lickorish. {\it Homeomorphisms of non-orientable two-manifolds}, 
Math. Proc. Camb. Phil. Soc. 59 (1963), 307-317.

\end{enumerate}
%%%%%
%%%%%
\end{document}
