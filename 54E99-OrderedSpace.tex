\documentclass[12pt]{article}
\usepackage{pmmeta}
\pmcanonicalname{OrderedSpace}
\pmcreated{2013-03-22 17:05:36}
\pmmodified{2013-03-22 17:05:36}
\pmowner{CWoo}{3771}
\pmmodifier{CWoo}{3771}
\pmtitle{ordered space}
\pmrecord{15}{39388}
\pmprivacy{1}
\pmauthor{CWoo}{3771}
\pmtype{Definition}
\pmcomment{trigger rebuild}
\pmclassification{msc}{54E99}
\pmclassification{msc}{06F20}
\pmclassification{msc}{06F30}
\pmsynonym{ordered topological space}{OrderedSpace}
\pmsynonym{topological ordered space}{OrderedSpace}
\pmsynonym{partially ordered space}{OrderedSpace}
\pmsynonym{partially ordered topological space}{OrderedSpace}
\pmrelated{OrderTopology}
\pmdefines{upper topology}
\pmdefines{lower topology}
\pmdefines{interval topology}
\pmdefines{upper semiclosed}
\pmdefines{lower semiclosed}
\pmdefines{semiclosed}
\pmdefines{pospace}
\pmdefines{totally ordered space}

\endmetadata

\usepackage{amssymb,amscd}
\usepackage{amsmath}
\usepackage{amsfonts}
\usepackage{mathrsfs}

% used for TeXing text within eps files
%\usepackage{psfrag}
% need this for including graphics (\includegraphics)
%\usepackage{graphicx}
% for neatly defining theorems and propositions
\usepackage{amsthm}
% making logically defined graphics
%%\usepackage{xypic}
\usepackage{pst-plot}
\usepackage{psfrag}

% define commands here
\newtheorem{prop}{Proposition}
\newtheorem{thm}{Theorem}
\newtheorem{ex}{Example}
\newcommand{\real}{\mathbb{R}}
\newcommand{\pdiff}[2]{\frac{\partial #1}{\partial #2}}
\newcommand{\mpdiff}[3]{\frac{\partial^#1 #2}{\partial #3^#1}}
\newcommand{\up}{\uparrow\!\!}
\newcommand{\down}{\downarrow\!\!}
\begin{document}
\textbf{Definition}.  A set $X$ that is both a topological space and a poset is variously called a \emph{topological ordered space}, \emph{ordered topological space}, or simply an \emph{ordered space}.  Note that there is no compatibility conditions imposed on $X$.  In other words, the topology $\mathcal{T}$ and the partial ordering $\le$ on $X$ operate independently of one another.  

If the partial order is a total order, then $X$ is called a \emph{totally ordered space}.  In some literature, a totally ordered space is called an ordered space.  In this entry, however, an ordered space is always a \emph{partially} ordered space.

One can construct an ordered space from a set with fewer structures.
\begin{enumerate}
\item
For example, any topological space is trivially an ordered space, with the partial order defined by $a\le b$ iff $a=b$.  But this is not so interesting.  A more interesting example is to take a $T_0$ space $X$, and define $a\le b$ iff $a\in \overline{\lbrace b\rbrace}$.  The relation so defined turns out to be a partial order on $X$, called the specialization order, making $X$ an ordered space.
\item
On the other hand, given any poset $P$, we can arbitrarily assign a topology on it, making it an ordered space, so that every poset is trivially an ordered space.  Again this is not very interesting.  
\item
A slightly more useful example is to take a poset $P$, and take $$\mathcal{L}(P):=\lbrace P-\up x\mid x\in P\rbrace,$$ the family of all set complements of principal upper sets of $P$, as the subbasis for the topology $\omega(P)$ of $P$.  The topology $\omega(P)$ so generated is called the \emph{lower topology} on $P$.  
\item
Dually, if we take $$\mathcal{U}(P):=\lbrace P-\down x\mid x\in P\rbrace,$$ as the subbasis, we get the \emph{upper topology} on $P$, denoted by $\nu(P)$.  
\item
In the lower topology $\omega(P)$ of $P$, if $y\in P-\up x$, then either $y< x$ (strict inequality) or $x\shortparallel y$ (incomparable with $x$).  If $x$ is an isolated element, then $P-\up x=P-\lbrace x\rbrace$.  This means that $\lbrace x\rbrace$ is a closed set.  Similarly, $\lbrace x\rbrace$ is closed in the upper topology $\nu(P)$.  

If $x$ is the top element of $P$, then $\lbrace x\rbrace$ is a closed set in $\omega(P)$, since $P-\up x=P-\lbrace x\rbrace$ is open.  Similarly $\lbrace x\rbrace$ is closed in $\nu(P)$ if $x$ is the bottom element in $P$.

If $P$ is totally ordered, there are no isolated elements.  As a result, we may write $P-\up x$ in a more familiar fashion: $(-\infty,x)$.  Similarly, $P-\down x$ may be written as $(x,\infty)$.
\item
Things get more interesting when we take the common refinement of $\omega(P)$ and $\nu(P)$.  What we end up with is called the \emph{interval topology} of $P$.  

When $P$ is totally ordered, the interval topology on $P$ has $$\mathcal{I}(P):=\lbrace (x,y)\mid x,y\in P\rbrace$$
as a subbasis, where $(x,y)$ denotes the \emph{open} poset interval, consisting of elements $a\in P$ such that $x<a<y$.  Since finite intersections of open poset intervals is a poset interval, an open set in $P$ can be written as an (arbitrary) union of open poset intervals.

As an example, the usual topology on $\mathbb{R}$ is precisely the interval topology generated by the linear order on $\mathbb{R}$.
\end{enumerate}

\textbf{Remark}.  It is a common practice in mathematics to impose special compatibility conditions on a structure having two inherent substructures so the substructures inter-relate, so that one can derive more interesting fruitful results.  This is true also in the case of an ordered space.  Let $X$ be an ordered space.  Below are some of the common conditions that can be imposed on $X$:
\begin{itemize}
\item
$X$ is said to be \emph{upper semiclosed} if $\up x$ is a closed set for every $x\in X$.  
\item 
Similarly, $X$ is \emph{lower semiclosed} if $\down x$ is closed in $X$.  
\item
$X$ is \emph{semiclosed} if it is both upper and lower semiclosed.  
\item
If $\le$, as a subset of $X\times X$, is closed in the product topology, then $X$ is called a \emph{pospace}.
\end{itemize}
Other structures, such as ordered topological vector spaces, \PMlinkname{topological lattices}{TopologicalLattice}, and \PMlinkname{topological vector lattices}{TopologicalVectorLattice} are ordered spaces with algebraic structures satisfying certain additional compatibility conditions.  Please click on the links for details.


\begin{thebibliography}{8}
\bibitem{ghklms} G. Gierz, K. H. Hofmann, K. Keimel, J. D. Lawson, M. W. Mislove, D. S. Scott, {\em Continuous Lattices and Domains}, Cambridge University Press, Cambridge (2003).
\end{thebibliography}
%%%%%
%%%%%
\end{document}
