\documentclass[12pt]{article}
\usepackage{pmmeta}
\pmcanonicalname{BaireSpace1}
\pmcreated{2013-03-22 16:04:09}
\pmmodified{2013-03-22 16:04:09}
\pmowner{mathcam}{2727}
\pmmodifier{mathcam}{2727}
\pmtitle{Baire space}
\pmrecord{5}{38124}
\pmprivacy{1}
\pmauthor{mathcam}{2727}
\pmtype{Definition}
\pmcomment{trigger rebuild}
\pmclassification{msc}{54E50}

\endmetadata

\usepackage{amssymb}
\newcommand{\N}[0]{\mathbb{N}}

\begin{document}
Baire space is the topological product of countably many copies of $\N$ with discrete topology.

It is a Polish space.

\begin{thebibliography}{9}
\bibitem{wiki} Wikipedia's entry on \PMlinkexternal{Baire space (set theory)}{http://en.wikipedia.org/wiki/Baire_space_(set_theory)}
\end{thebibliography} 

%%%%%
%%%%%
\end{document}
