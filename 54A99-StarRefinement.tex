\documentclass[12pt]{article}
\usepackage{pmmeta}
\pmcanonicalname{StarRefinement}
\pmcreated{2013-03-22 16:44:13}
\pmmodified{2013-03-22 16:44:13}
\pmowner{CWoo}{3771}
\pmmodifier{CWoo}{3771}
\pmtitle{star refinement}
\pmrecord{7}{38959}
\pmprivacy{1}
\pmauthor{CWoo}{3771}
\pmtype{Definition}
\pmcomment{trigger rebuild}
\pmclassification{msc}{54A99}
\pmdefines{star}
\pmdefines{star refine}
\pmdefines{barycentric refinement}

\endmetadata

\usepackage{amssymb,amscd}
\usepackage{amsmath}
\usepackage{amsfonts}
\usepackage{mathrsfs}

% used for TeXing text within eps files
%\usepackage{psfrag}
% need this for including graphics (\includegraphics)
%\usepackage{graphicx}
% for neatly defining theorems and propositions
\usepackage{amsthm}
% making logically defined graphics
%%\usepackage{xypic}
\usepackage{pst-plot}
\usepackage{psfrag}

% define commands here
\newtheorem{prop}{Proposition}
\newtheorem{thm}{Theorem}
\newtheorem{ex}{Example}
\newcommand{\real}{\mathbb{R}}
\begin{document}
Let $X$ be a set and $\mathscr{C}=\lbrace C_i\mid i\in I\rbrace$ be a cover of $X$ (we assume $C_i$ and $X$ are all subsets of some universe).  Let $A\subseteq X$.  The \emph{star} of $A$ (with respect to the cover $\mathscr{C}$) is defined as
$$\star(A,\mathscr{C}):=\bigcup \lbrace C_i\in \mathscr{C} \mid C_i\cap A\neq \varnothing \rbrace.$$
When $A$ is a singleton, we write $\star(x,\mathscr{C})=\star(\lbrace x\rbrace, \mathscr{C})$.  

\textbf{Properties of $\star$}
\begin{enumerate}
\item $A\subseteq \star(A,\mathscr{C})$.
\item If $A\subseteq B$, then $\star(A,\mathscr{C})\subseteq \star(B,\mathscr{C})$.
\item For any cover $\mathscr{C}$ of $X$, the sets $\mathscr{C}^{\star}:=\lbrace \star(C_i,\mathscr{C}) \mid C_i\in \mathscr{C}\rbrace$ and $\mathscr{C}^b:=\lbrace \star(x,\mathscr{C})\mid x\in X\rbrace$ are both covers of $X$.
\item $\mathscr{C}\preceq \mathscr{C}^b \preceq \mathscr{C}^{\star}$ ($\preceq$ denotes cover refinement).
\end{enumerate}

\textbf{Definitions}.  Let $\mathscr{C},\mathscr{D}$ be two covers of $X$.  If $\mathscr{C}^{\star} \preceq \mathscr{D}$, then we say that $\mathscr{C}$ is a \emph{star refinement} of $\mathscr{D}$, denoted by $\mathscr{C} \preceq^{\star} \mathscr{D}$.  If $\mathscr{C}^b \preceq \mathscr{D}$, then we say that $\mathscr{C}$ is a \emph{barycentric refinement} of $\mathscr{D}$, denoted by $\mathscr{C} \preceq^b \mathscr{D}$.

\textbf{Remark}.  By property 4 above, it is easy to see that 
$\mathscr{C} \preceq^{\star}\mathscr{D}\Rightarrow \mathscr{C} \preceq^b\mathscr{D}\Rightarrow \mathscr{C} \preceq \mathscr{D}$.

\begin{thebibliography}{9}
\bibitem{willard} S. Willard, \emph{General Topology},
Addison-Wesley, Publishing Company, 1970.
\end{thebibliography}
%%%%%
%%%%%
\end{document}
