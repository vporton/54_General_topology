\documentclass[12pt]{article}
\usepackage{pmmeta}
\pmcanonicalname{ClosureAxioms}
\pmcreated{2013-03-22 13:13:44}
\pmmodified{2013-03-22 13:13:44}
\pmowner{Koro}{127}
\pmmodifier{Koro}{127}
\pmtitle{closure axioms}
\pmrecord{9}{33697}
\pmprivacy{1}
\pmauthor{Koro}{127}
\pmtype{Definition}
\pmcomment{trigger rebuild}
\pmclassification{msc}{54A05}
\pmsynonym{Kuratowski's closure axioms}{ClosureAxioms}
\pmsynonym{Kuratowski closure axioms}{ClosureAxioms}
\pmrelated{Closure}
\pmdefines{closure operator}

\endmetadata

% this is the default PlanetMath preamble.  as your knowledge
% of TeX increases, you will probably want to edit this, but
% it should be fine as is for beginners.

% almost certainly you want these
\usepackage{amssymb}
\usepackage{amsmath}
\usepackage{amsfonts}

% used for TeXing text within eps files
%\usepackage{psfrag}
% need this for including graphics (\includegraphics)
%\usepackage{graphicx}
% for neatly defining theorems and propositions
%\usepackage{amsthm}
% making logically defined graphics
%%%\usepackage{xypic}

% there are many more packages, add them here as you need them

% define commands here
\begin{document}
A \emph{closure operator} on a set $X$ is an operator which assigns a set $A^c$ to each subset $A$ of $X$, and such that the following (Kuratowski's closure axioms) hold for any subsets $A$ and $B$ of $X$:
\begin{enumerate}
\item $\emptyset^c = \emptyset$;
\item $A\subset A^c$;
\item $(A^c)^c = A^c$;
\item $(A\cup B)^c = A^c\cup B^c.$
\end{enumerate}

The following theorem due to Kuratowski says that a closure operator characterizes a unique topology on $X$:

\textbf{Theorem.} Let $c$ be a closure operator on $X$, and let $\mathcal{T} = \{X-A: A\subseteq X,\; A^c=A\}$. Then $\mathcal{T}$ is a topology on $X$, and $A^c$ is the $\mathcal{T}$-closure of $A$ for each subset $A$ of $X$.
%%%%%
%%%%%
\end{document}
