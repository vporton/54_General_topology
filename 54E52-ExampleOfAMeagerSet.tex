\documentclass[12pt]{article}
\usepackage{pmmeta}
\pmcanonicalname{ExampleOfAMeagerSet}
\pmcreated{2013-03-22 17:07:08}
\pmmodified{2013-03-22 17:07:08}
\pmowner{Wkbj79}{1863}
\pmmodifier{Wkbj79}{1863}
\pmtitle{example of a meager set}
\pmrecord{4}{39420}
\pmprivacy{1}
\pmauthor{Wkbj79}{1863}
\pmtype{Example}
\pmcomment{trigger rebuild}
\pmclassification{msc}{54E52}
\pmrelated{ExamplesOfNowhereDenseSets}

\endmetadata

\usepackage{amssymb}
\usepackage{amsmath}
\usepackage{amsfonts}

\usepackage{psfrag}
\usepackage{graphicx}
\usepackage{amsthm}
%%\usepackage{xypic}

\begin{document}
Note that $\mathbb{Q}$ is meager in $\mathbb{R}$ under the usual topology.  Let $\{r_n\}_{n \in \mathbb{N}}$ be an enumeration of $\mathbb{Q}$.  Then $\displaystyle \mathbb{Q}=\bigcup_{n \in \mathbb{N}} \{r_n\}$ and, for every $n \in \mathbb{N}$, $\operatorname{int} \overline{\{r_n\}}=\operatorname{int} \{r_n\}=\emptyset$.
%%%%%
%%%%%
\end{document}
