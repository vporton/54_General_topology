\documentclass[12pt]{article}
\usepackage{pmmeta}
\pmcanonicalname{CantorSpace}
\pmcreated{2013-03-22 13:44:42}
\pmmodified{2013-03-22 13:44:42}
\pmowner{mathcam}{2727}
\pmmodifier{mathcam}{2727}
\pmtitle{Cantor space}
\pmrecord{14}{34439}
\pmprivacy{1}
\pmauthor{mathcam}{2727}
\pmtype{Definition}
\pmcomment{trigger rebuild}
\pmclassification{msc}{54E50}
%\pmkeywords{Cantor}
%\pmkeywords{Polish space}
%\pmkeywords{binary sequence}
%\pmkeywords{language}

% this is the default PlanetMath preamble.  as your knowledge
% of TeX increases, you will probably want to edit this, but
% it should be fine as is for beginners.

% almost certainly you want these
\usepackage{amssymb}
\usepackage{amsmath}
\usepackage{amsfonts}

% used for TeXing text within eps files
%\usepackage{psfrag}
% need this for including graphics (\includegraphics)
%\usepackage{graphicx}
% for neatly defining theorems and propositions
%\usepackage{amsthm}
% making logically defined graphics
%%%\usepackage{xypic}

% there are many more packages, add them here as you need them

% define commands here
\begin{document}
The {\it Cantor space}, denoted by $\mathbb{C}$ is the set of all infinite binary sequences with the product topology. It is a perfect Polish space. It is a compact subspace of the Baire space, the set of all infinite sequences of integers (again with the natural product topology).

{\bf References.}
\begin{itemize}
\item Moschovakis, Yiannis N.  \emph{Descriptive Set Theory}.  North-Holland Pub. Co.  1980, Amsterdam; New York.
\end{itemize}
%%%%%
%%%%%
\end{document}
