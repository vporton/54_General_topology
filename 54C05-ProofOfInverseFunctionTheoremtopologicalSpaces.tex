\documentclass[12pt]{article}
\usepackage{pmmeta}
\pmcanonicalname{ProofOfInverseFunctionTheoremtopologicalSpaces}
\pmcreated{2013-03-22 13:31:55}
\pmmodified{2013-03-22 13:31:55}
\pmowner{paolini}{1187}
\pmmodifier{paolini}{1187}
\pmtitle{proof of inverse function theorem (topological spaces)}
\pmrecord{5}{34125}
\pmprivacy{1}
\pmauthor{paolini}{1187}
\pmtype{Proof}
\pmcomment{trigger rebuild}
\pmclassification{msc}{54C05}

\endmetadata

% this is the default PlanetMath preamble.  as your knowledge
% of TeX increases, you will probably want to edit this, but
% it should be fine as is for beginners.

% almost certainly you want these
\usepackage{amssymb}
\usepackage{amsmath}
\usepackage{amsfonts}

% used for TeXing text within eps files
%\usepackage{psfrag}
% need this for including graphics (\includegraphics)
%\usepackage{graphicx}
% for neatly defining theorems and propositions
%\usepackage{amsthm}
% making logically defined graphics
%%%\usepackage{xypic}

% there are many more packages, add them here as you need them

% define commands here
\begin{document}
We only have to prove that whenever $A\subset X$ is an open set, then also $B=(f^{-1})^{-1}(A) = f(A) \subset Y$ is open ($f$ is an open mapping). Equivalently it is enough to prove that $B'=Y\setminus B$ is closed. 

Since $f$ is bijective we have $B' = Y\setminus B = f(X\setminus A)$

As $A'=X\setminus A$ is closed and since $X$ is compact $A'$ is compact too (this and the following are well know properties of compact spaces).
Moreover being $f$ continuous we know that also $B'=f(A')$ is compact. Finally since $Y$ is Hausdorff then $B'$ is closed.
%%%%%
%%%%%
\end{document}
