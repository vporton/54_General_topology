\documentclass[12pt]{article}
\usepackage{pmmeta}
\pmcanonicalname{ACompleteSubspaceOfAMetricSpaceIsClosed}
\pmcreated{2013-03-22 16:31:29}
\pmmodified{2013-03-22 16:31:29}
\pmowner{ehremo}{15714}
\pmmodifier{ehremo}{15714}
\pmtitle{a complete subspace of a metric space is closed}
\pmrecord{5}{38704}
\pmprivacy{1}
\pmauthor{ehremo}{15714}
\pmtype{Result}
\pmcomment{trigger rebuild}
\pmclassification{msc}{54E50}

% this is the default PlanetMath preamble.  as your knowledge
% of TeX increases, you will probably want to edit this, but
% it should be fine as is for beginners.

% almost certainly you want these
\usepackage{amssymb}
\usepackage{amsmath}
\usepackage{amsfonts}

% used for TeXing text within eps files
%\usepackage{psfrag}
% need this for including graphics (\includegraphics)
%\usepackage{graphicx}
% for neatly defining theorems and propositions
%\usepackage{amsthm}
% making logically defined graphics
%%%\usepackage{xypic}

% there are many more packages, add them here as you need them

% define commands here

\begin{document}
Let $X$ be a metric space, and let $Y$ be a complete subspace of $X$. Then $Y$ is closed.

{\bf Proof}

Let $x \in \overline Y$ be a point in the closure of $Y$. Then by the definition of closure, from each ball $B(x, \frac 1 n)$ centered in $x$, we can select a point $y_n \in Y$. This is clearly a Cauchy sequence in $Y$, and its limit is $x$, hence by the completeness of $Y$, $x \in Y$ and thus $Y = \overline Y$.
%%%%%
%%%%%
\end{document}
